\section*{Basic Exercises}

Compute the following linear combinations of vectors.
\begin{enumerate}
\item
  \begin{align*}
    7 \begin{bmatrix} 2 \\ -3 \\ -8 \\ 9 \end{bmatrix}
    + 3 \begin{bmatrix} 2 \\ -3 \\ -4 \\ -3 \end{bmatrix}
    - 4\begin{bmatrix} 5 \\ -1 \\ 5 \\ -7 \end{bmatrix}
    - 2\begin{bmatrix} -2 \\ -9 \\ -2 \\ -10 \end{bmatrix}
  \end{align*}
\end{enumerate}
For each of the following linear systems, determine an equivalent vector equation, i.e., one that has the same solution set as the given linear system.
You do not need to solve the vector equation.
\begin{enumerate}[resume]
\item
  \begin{align*}
    8x_{1} + 6x_{2} - 9x_{3} &= -5 \\
    4x_{1} + 3x_{2} + 9x_{3} + 2x_{4} &= -1 \\
    4x_{1} + 5x_{2} &= 9 \\
    3x_{2} + 8x_{3} - 3x_{4} &= 2
  \end{align*}
\end{enumerate}
For each of the following pairs of vectors $\mathbf v_1$ and $\mathbf v_2$, determine a vector that is in $\mathsf{span}\{\mathbf{v}_1, \mathbf{v}_2\}$ but not in $\mathsf{span}\{\mathbf{v}_1\}$ or $\mathsf{span}\{\mathbf{v}_2\}$, if possible.
Justify your answer.
In particular, if it is not possible, then explain why.
\begin{enumerate}[resume]
\item
  \begin{align*}
    \mathbf{v}_{1} = \begin{bmatrix} -7 \\ -1 \\ 0 \end{bmatrix}
    \quad
    \mathbf{v}_{2} = \begin{bmatrix} -8 \\ 9 \\ -7 \end{bmatrix}
  \end{align*}
\end{enumerate}
For each of the following vector equations, determine a general form solution.
If this equation has no solutions then write \textit{NO SOLUTION}.
\begin{enumerate}[resume]
\item
  \begin{align*}
    x_1\begin{bmatrix} 1 \\ -2 \\ 0 \end{bmatrix}
    + x_{2}\begin{bmatrix} -2 \\ 5 \\ 2 \end{bmatrix}
    + x_{3}\begin{bmatrix} -8 \\ 21 \\ 10 \end{bmatrix}
    = \begin{bmatrix} 16 \\ -38 \\ -12 \end{bmatrix}
  \end{align*}
\end{enumerate}
For each of the following collections of vectors, determine if the vector $\mathbf v_1$ is in the span of the remaining vectors. If it is, determine the corresponding dependence relation, i.e., write $\mathbf v_1$ as a linear combination of the remaining vectors.
\begin{enumerate}[resume]
\item
  \begin{align*}
    \mathbf v_1 = \begin{bmatrix} 0 \\ -3 \\ -20 \end{bmatrix}
    \qquad
    \mathbf v_2 = \begin{bmatrix} 1 \\ 1 \\ 3 \end{bmatrix}
    \qquad
    \mathbf v_3 = \begin{bmatrix} 1 \\ 2 \\ 8 \end{bmatrix}
  \end{align*}
\item
  \begin{align*}
    \mathbf{v}_{1} = \begin{bmatrix} 1 \\ 4 \\ -7 \end{bmatrix}
    \quad
    \mathbf{v}_{2} = \begin{bmatrix} 0 \\ 1 \\ -3 \end{bmatrix}
    \quad
    \mathbf{v}_{3} = \begin{bmatrix} 5 \\ 7 \\ -1 \end{bmatrix}
    \quad
    \mathbf{v}_{4} = \begin{bmatrix} 1 \\ 2 \\ -2 \end{bmatrix}
  \end{align*}
\item \faCalculator
  \begin{align*}
    \mathbf v_1 = \begin{bmatrix} 42 \\ -23 \\ 98 \\ 11 \\ -87 \end{bmatrix}
    \qquad
    \mathbf v_2 = \begin{bmatrix} 10 \\ -33 \\ -5 \\ -30 \\ -3 \end{bmatrix}
    \qquad
    \mathbf v_3 = \begin{bmatrix} -47 \\ -1 \\ -2 \\ -25 \\ 24 \end{bmatrix}
  \end{align*}
  \begin{align*}
    \mathbf v_4 = \begin{bmatrix} 31 \\ -34\\ 11 \\ 39 \\ 25 \end{bmatrix}
    \qquad
    \mathbf v_5 =\begin{bmatrix}-14 \\ 22 \\ 12 \\ 42 \\ 3\end{bmatrix}
  \end{align*}
\item \faCalculator
  \begin{align*}
    \mathbf v_1 = \begin{bmatrix} 87 \\ -19 \\ -24 \\ -61 \\ -79 \end{bmatrix}
    \qquad
    \mathbf v_2 = \begin{bmatrix} 10 \\ -33 \\ -5 \\ -30 \\ -3 \end{bmatrix}
    \qquad
    \mathbf v_3 = \begin{bmatrix} -47 \\ -1 \\ -2 \\ -25 \\ 24 \end{bmatrix}
  \end{align*}
  \begin{align*}
    \mathbf v_4 = \begin{bmatrix} 31 \\ -34\\ 11 \\ 39 \\ 25 \end{bmatrix}
    \qquad
    \mathbf v_5 =\begin{bmatrix}-14 \\ 22 \\ 12 \\ 42 \\ 3\end{bmatrix}
  \end{align*}
\end{enumerate}
For each of the following pairs of vectors $\mathbf v_1$ and $\mathbf v_2$, determine a linear equation whose point set is the $\mathrm{span}\{\mathbf v_1, \mathbf v_2\}$.
The linear equation you determine should have relatively prime integer coefficients (i.e., it should not be possible to divide the equation by an integer value and get a new equation with integer coefficients).
\begin{enumerate}[resume]
\item
  \begin{align*}
    \mathbf{v}_{1} = \begin{bmatrix} -2 \\ 1 \\ 2 \end{bmatrix}
    \quad
    \mathbf{v}_{2} = \begin{bmatrix} -1 \\ 0 \\ 2 \end{bmatrix}
  \end{align*}
\item
  \begin{align*}
    \mathbf v_1 = \begin{bmatrix} 1 \\ 0 \\ -2 \end{bmatrix}
    \qquad
    \mathbf v_2 = \begin{bmatrix} 1 \\ 1 \\ 1 \end{bmatrix}
  \end{align*}
\end{enumerate}

\section*{True/False}
\trueFalseHeader
\begin{enumerate}[resume]
\item For any two vectors $\mathbf v_1$ and $\mathbf v_2$ in $\mathbb
  R^n$, there is a vector $\mathbf u$ such that $\mathbf v_1 + \mathbf
  v_2 + \mathbf u = \mathbf 0$.
\item For any vectors $\mathbf v_1$, $\mathbf v_2$, and $\mathbf v_3$
  in $\mathbb R^n$, if $\mathbf v_1 \in \mathsf{span}\{\mathbf v_2,
  \mathbf v_3\}$, then $\mathbf v_2 \in \mathsf{span}\{\mathbf v_1,
  \mathbf v_3\}$.
\item The span of any two distinct nonzero vectors in $\mathbb R^3$ is a plane.
\item For any vector $\mathbf v_1$, $\mathbf v_2$, and $\mathbf v_3$ in $\mathbb R^n$, $\mathsf{span}\{\mathbf v_1, \mathbf v_2, \mathbf v_3\} = \mathsf{span}\{\mathbf v_1 + \mathbf v_3, \mathbf v_2\}$.
\end{enumerate}

\section*{More Difficult Problems}

\begin{enumerate}[resume]
\item
  Consider the vectors
  \begin{displaymath}
    \mathbf v_1 = \begin{bmatrix} 1 \\ 1 \\ 1 \\ 1 \end{bmatrix}
    \qquad
    \mathbf v_2 = \begin{bmatrix} 1 \\ 1 \\ 2 \\ 2 \end{bmatrix}
  \end{displaymath}
  Write vectors $\mathbf v_3$ and $\mathbf v_4$ such that $\mathbf v_3$ is not in $\mathsf{span}\{\mathbf v_1, \mathbf v_2\}$ and $\mathbf v_4$ is not in $\mathsf{span}\{\mathbf v_1, \mathbf v_2, \mathbf v_3\}$.
\end{enumerate}

\section*{Challenge Problems}

\begin{enumerate}[resume]
\item
  Find a linear equation in three variables whose point set is exactly
  \begin{align*}
    \left\{
    \mathbf v +
    \begin{bmatrix} 2 \\ -3 \\ 2 \end{bmatrix}
    \ \text{where} \
    \mathbf v \in
    \mathsf{span}
    \left\{
    \begin{bmatrix} 1 \\ 1 \\ -5 \end{bmatrix},
    \begin{bmatrix} 0 \\ 1 \\ -1 \end{bmatrix}
    \right\}
    \right\}
  \end{align*}
  In other words, every point in the point set of the equation can be expressed as the sum of $[2 \ \ (-3) \ \ 2]^T$ and a vector in the span of $[1 \ \ 1 \ \ (-5)]^T$ and $[0 \ \ 1 \ \ (-1)]^T$.
\item
  Consider the following linear equation.
  \begin{displaymath}
    x + y + z = 5
  \end{displaymath}
  The plane represented by this equation does not include the origin; such a plane is called \textit{affine}.
  Determine vectors $\mathbf b$, $\mathbf v_1$, and $\mathbf v_2$ such that every point in the plane represented by the above equation (that is, every point $(a, b, c)$ such that $a + b + c = 5$) can be written as $\mathbf b + \mathbf v$ where $\mathbf v$ is in $\mathsf{span} \{\mathbf v_1, \mathbf v_2\}$, i.e.,
  \begin{displaymath}
    \{(x, y, z) : x + y + z = 5\} = \{\mathbf b + \mathbf v : \mathbf v \in \mathsf{span}\{\mathbf v_1, \mathbf v_2\}\}.
  \end{displaymath}
\end{enumerate}
