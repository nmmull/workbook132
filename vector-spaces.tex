\section{Basic Exercises}


For each of the following exercises, justify your answer.
\begin{enumerate}
\item
  Given $A \in \R^{3 \times 6}$, determine value of $n$ such that
  $\nul A$ is a subspace of $\R^n$.
\item
  Given $A \in \R^{10 \times 13}$, determine the minimum dimension of
  $\nul A$.
\item
  Given $A \in \R^{7 \times 5}$ and $\rank A = 4$, determine
  $\dim(\nul A)$.
\item
  Determine if $v$ is in $\nul A$ where
  \begin{align*}
    A =
    \begin{bmatrix}
      -1 & 0 & 1 \\
      3 & 6 & 0\\
      5 & 7 & 2
    \end{bmatrix}
    \qquad
    \vv = \vThree 2 {-1} 2
  \end{align*}
\item Determine if $\vv$ is in $\col A$, where $\vv$ and $A$ are
  defined as in the previous exercise.
\item Without performing any row operations, determine $\rank A$ where
  \begin{align*}
    A =
    \begin{bmatrix}
      2 & 1 & -8 & 3 \\
      -1 & 3 & 4 & 2 \\
      3 & 2 & -12 & 5 \\
      1 & -2 & -4 & -1
    \end{bmatrix}
  \end{align*}
\end{enumerate}
For each of the following matrices, determine a basis for its column
space and a basis for its null space.
\begin{enumerate}[resume]
\item
  \begin{align*}
    \begin{bmatrix}
      1 & -5 \\
      -2 & 10
    \end{bmatrix}
  \end{align*}
\item
  \begin{align*}
    \begin{bmatrix}
      0 & 0 \\
      0 & 0
    \end{bmatrix}
  \end{align*}
\item
  \begin{align*}
    \begin{bmatrix}
      0 & -2 & 2 \\
      -2 & 3 & -9 \\
      -1 & -2 & -1
    \end{bmatrix}
  \end{align*}
\item
  \begin{align*}
    \begin{bmatrix}
      1 & -4 & 3 & -3 \\
      -2 & 8 & -6 & 7
    \end{bmatrix}
  \end{align*}
\item
  \begin{align*}
    \begin{bmatrix}
      1 & -4 & -3 \\
      -3 & 12 & 10 \\
      -2 & 8 & 8 \\
      -1 & 4 & 2
    \end{bmatrix}
  \end{align*}
\item
  The matrix $[ \mathbf a_1 \ \ \ \mathbf a_2 \ \ \ \mathbf a_3
    \ \ \ \mathbf a_4 \ \ \ \mathbf a_5 \ \ \ \mathbf a_6
    \ \ \ \mathbf a_7 \ \ \ \mathbf a_8 ]$ which is row-equivalent to
  the following matrix.
  \begin{align*}
    \begin{bmatrix}
      1 & 0 & 3 & 0 & 0 & 5 & 0 & -7 \\
      0 & 1 & 14 & 0 & 0 & 2 & 0 & 1 \\
      0 & 0  & 0  & 1 & 0 & 3 & 0 & 7 \\
      0 & 0  & 0  & 0 & 1 & 8 & 0 & 9 \\
      0 & 0  & 0  & 0 & 0 & 0 & 1 & 1 \\
      0 & 0  & 0  & 0 & 0 & 0 & 0 & 0 \\
    \end{bmatrix}
  \end{align*}
\item \faCalculator
  \begin{lstlisting}

  Matrix(
    [[1,   2,   4,   3,  -4,  1],
     [-3, -4, -10,  -8,  13, -3],
     [ 5,  6,  16,  10, -13,  9],
     [-7, -8, -22, -12,  13, -9],
     [13, 18,  44,  32, -47, 11]]
  )
  \end{lstlisting}
\end{enumerate}
For each of the following subspaces, determine a basis.
\begin{enumerate}[resume]
\item
  \begin{align*}
    \mathrm{span}\!
    \left
    \{\begin{bmatrix} 1 \\ 2 \\ 2 \end{bmatrix},
    \begin{bmatrix} 2 \\ 5 \\ 1 \end{bmatrix},
    \begin{bmatrix} -3 \\ -5 \\ -8 \end{bmatrix}
    \right\}
  \end{align*}
\item
  \begin{align*}
    \mathrm{span}\! \left\{
    \begin{bmatrix} 1 \\ 1 \\ 0 \end{bmatrix},
    \begin{bmatrix} -1 \\ 0 \\ -3 \end{bmatrix},
    \begin{bmatrix} 2 \\ -4 \\ 18 \end{bmatrix},
    \begin{bmatrix} -2 \\ -4 \\ 6 \end{bmatrix}
    \right\}
\end{align*}
\end{enumerate}
Determine the coordinate vector $[\mathbf u]_{\mathcal B}$ where
$\vu$ and $\mathcal B$ are defined below.
\begin{enumerate}[resume]
\item
  \begin{displaymath}
    \vu = \vTwo 8 {-12}
    \qquad
    \mathcal B = \left\{ \vTwo 1 {-1}, \vTwo {-3} 4 \right\}
  \end{displaymath}
\item
  \begin{displaymath}
    \vu = \vThree 4 1 3
    \qquad
    \mathcal B = \left\{
    \vThree 1 {-1} 0
    \vThree {-1} 2 1
    \vThree {-2} 2 1
    \right\}
  \end{displaymath}
\item
  \begin{displaymath}
    \vu = \vTwo {-17} {15}
    \qquad
    \mathcal B =
    \left\{
    \vTwo 1 {-5}
    \vTwo 3 {-1}
    \right\}
  \end{displaymath}
\item \faCalculator
  \begin{displaymath}
    \vu = \vFour 5 {29} {-80} {42}
    \qquad
    \mathcal B =
    \left\{
    \vFour {-11} {-19} {17} 9,
    \vFour 9 {17} {-20} {-2},
    \vFour {-3} 1 {-18} {18}
    \right\}
  \end{displaymath}
\end{enumerate}
Determine the change-of-basis matrix for the following bases.
\begin{enumerate}[resume]
\item
  \begin{displaymath}
    \left\{
    \vTwo 1 2,
    \vTwo {-3} {-5}
    \right\}
  \end{displaymath}
\item
\begin{displaymath}
\left\{
\vThree 1 {-3} {-2},
\vThree {-3} {10} {5},
\vThree {-2} 8 3
\right\}
\end{displaymath}
\end{enumerate}
For each of the following matrices, determine a vector that is
\textit{not} in its column space.
\begin{enumerate}[resume]
\item
  \begin{displaymath}
    \begin{bmatrix}
      1 & 0 & 0 & 0 \\
      0 & 1 & 2 & 0 \\
      0 & 0 & 0 & 1 \\
      0 & 0 & 0 & 0
    \end{bmatrix}
  \end{displaymath}
\item
  \begin{displaymath}
    \begin{bmatrix}
      1 & 1 & 5 \\
      -1 & 0 & -1 \\
      1 & 2 & 9
    \end{bmatrix}
  \end{displaymath}
\end{enumerate}

\section{True/False}

\trueFalseHeader

\begin{enumerate}[resume]
\item
  There is a unique basis for any subspace.
\item
  A $7 \times 4$ matrix $A$ may have $\mathrm{dim}(\mathrm{Nul}(A)) = 5$.
\item
  A $3 \times 6$ matrix $A$ may have $\mathrm{dim}(\mathrm{Nul}(A)) = 2$.
\item
  For any matrix $A \in \mathbb R^{n \times n}$, if $A$ is invertible then $\mathrm{rank}(A) = n$.
\item
  For any matrix $A \in \mathbb R^{m \times n}$, $\mathrm{Col}(A)$ is
  the same as the set of vectors $\mathbf{b}$ such that $A \mathbf{x}
  = \mathbf{b}$ has a solution.
\item
  A basis is a spanning set that is as large as possible.
\item
  A linear transformation $T: \mathbb{R}^3 \to \mathbb{R}^3$ that maps
  $\mathbb{R}^3$ to a plane has a trivial kernel (i.e., if
  $T(\mathbf{v}) = \mathbf{0}$, then $\mathbf{v} = \mathbf{0}$).
\item
  If $\cB$ is the standard basis for $R^n$, then for any $\vx \in
  \R^n$ we have that $[\vx]_{\cB} = \vx$
\end{enumerate}

\section{More Difficult Problems}

\begin{enumerate}[resume]
\item
  For a matrix $A \in \mathbb R^{m \times n}$, consider the set of
  vectors $\{\mathbf{x} \in \mathbb{R}^n : A\mathbf{x}=
  \mathbf{e_1}\}$ (\textit{Recall:} $\mathbf{e_1}$ is the first
  standard basis vector).
  \begin{enumerate}
  \item
    Determine if this set if closed under addition. Justify your
    answer.
  \item
    Determine if this set is closed under scaling. Justify your
    answer.
  \item
    Determine if this set is a subspace of $\mathbb R^n$. Justify your
    answer.
  \end{enumerate}
\item
  \begin{align*}
    \mathbf{v_1} = \begin{bmatrix}
      3  \\
      0  \\
      -1
    \end{bmatrix}, \,
    \mathbf{v_2} = \begin{bmatrix}
      -3  \\
      2  \\
      0
    \end{bmatrix}, \,
    \mathbf{v_3} = \begin{bmatrix}
      0  \\
      -2  \\
      -1
    \end{bmatrix}
\end{align*}
List all possible subsets of the above vectors that form a basis of
the subspace $\mathrm{span}\!\left\{\mathbf{v_1}, \, \mathbf{v_2}, \,
\mathbf{v_3} \right\}$.\footnote{The process we teach for determining
a basis gives one choice, but there may be multiple choices that are
suitable.}
\item Consider the following vectors.
\begin{align*}
\mathbf{v_1} = \begin{bmatrix}
      3   \\
      -1
\end{bmatrix}, \,
\mathbf{v_2} = \begin{bmatrix}
      -3  \\
      1
\end{bmatrix}, \,
\mathbf{v_3} = \begin{bmatrix}
      -1  \\
      1
\end{bmatrix}, \,
\mathbf{v_4} = \begin{bmatrix}
      8  \\
      -4
\end{bmatrix}
\end{align*}
List all possible subsets of the above vectors that form a basis of
the subspace $\mathrm{span}\!\left\{\mathbf{v_1}, \, \mathbf{v_2}, \,
\mathbf{v_3}, \, \mathbf{v_4} \right\}$.
\item Consider the 3-dimensional vector space of all quadratic
  polynomials $Q = \{a x^2 + b x + c \, | \, a,b,c \in \mathbb{R}\}$
  and consider the linear derivative map $\frac{d}{dx}: Q \to Q$
  defined by $\frac{d}{dx}(a x^2 + b x + c) = 0 x^2 + 2 a x + b$.
  \begin{enumerate}
  \item
    Using the standard basis given by $\{x^2, x, 1 \}$, determine a $3
    \times 3$ matrix $A$ that implements $\frac{d}{dx}$.
  \item
    Determine a basis for $\mathrm{Col}(A)$.
  \item
    Determine basis for $\mathrm{Nul}(A)$.
  \item
    Determine $\mathrm{rank}(A)$ and $\mathrm{dim}(\mathrm{Nul}(A))$.
  \end{enumerate}
\item
  Let $A$ be an $m \times n$ matrix and let $\vv$ be a vector in $\R^n$.
  Let $\vb$ denote the vector $A\vv$.
  \begin{enumerate}
  \item
    Show that if $\vw \in \nul A$, then $\vv + \vw$ is a solution to the equation $A\vx = \vb$.
  \item
    Show that if $\vb \not = 0$, then the solution set of $A\vx = \vb$ is not a subspace of $\R^n$.
  \item
    A set $H$ is an \textbf{affine} subspace of $\R^n$ if there is a subspace $U$ of $\R^n$ and a vector $\vo$ such that
    \begin{align*}
      H = \{\vu + \vo \ | \ \vu \in U\}
    \end{align*}
    Show that the solution set of $A\vx = \vb$ is an affine subspace of $\R^n$ if it is nonempty.
    In particular, choose a vector $\vo$ and subspace $U$.
  \end{enumerate}
\item
Determine the linear equation whose solution set is the column space
of the following matrix.
\begin{displaymath}
  \begin{bmatrix}
    1 & -4 & -10 \\
    -4 & 17 & 42 \\
    1 & -2 & -6
  \end{bmatrix}
\end{displaymath}
\item
  Let $A$ be a $4 \times 2024$ matrix where $\rank A = 3$. Further
  suppose that the LU-decomposition of $A$ has
  \begin{displaymath}
    L =
    \begin{bmatrix}
      1 & 0 & 0 & 0 \\
      -2 & 1 & 0 & 0 \\
      7 & 5 & 1 & 0 \\
      0 & 0 & 1 & 1
    \end{bmatrix}
  \end{displaymath}
  Determine the linear equation whose solution set is $\col A$.
\item
  Consider the following two matrices.
  \begin{displaymath}
    A =
    \begin{bmatrix}
      6 & -6 & -1 & 17 \\
      9 & -2 & 2 & 36 \\
      -10 & -13 & 8 & -45 \\
      -9 & 4 & -1 & -33 \\
      2 & 4 & 2 & 14
    \end{bmatrix}
    \qquad
    B =
    \begin{bmatrix}
      -8 & 5 & -16 & 1 \\
      2 & 11 & 4 & 16 \\
      3 & -2 & 6 & 3 \\
      2 & -10 & 4 & 0 \\
      8 & 4 & 16 & 9
    \end{bmatrix}
  \end{displaymath}
  Also consider the following set of vectors.
  \begin{displaymath}
    H = \{ \vu + \vv \ | \ \vu \in \col A \text{ and } \vv \in \col B \}
  \end{displaymath}
  That is, $H$ consists of all sums of pairs of vectors, one from
  $\col A$ and one from $\col B$.
  \begin{enumerate}
  \item
    Show that $H$ is a subspace of $\R^5$.
  \item \faCalculator \
    Determine the dimension of $H$.  Justify your answer.  You may use a
    computer but you must determine any matrix you reduce along with its
    reduced echelon form.
  \end{enumerate}
\end{enumerate}

\section{Challenge Problems}

\begin{enumerate}[resume]
\item
  The row space of a matrix $A \in \mathbb R^{m \times n}$ is the span
  of the rows of $A$, denoted $\mathrm{Row}(A)$. Show that
  $\mathrm{dim}(\mathrm{Row}(A)) + \mathrm{dim}(\mathrm{Nul}(A)) = n$.
\item
  In the vector space of all real-valued functions, find a basis for
  the subspaced spanned by $\{\sin t, \, \sin 2t, \, \sin t \cos t
  \}$.
\end{enumerate}
