\section{Basic Exercises}

Compute the matrix-vector multiplication $A\mathbf v$ where $A$ and $\mathbf v$ are given below.
If it is not possible to multiply $A$ with $\mathbf v$, then explain why.
\begin{enumerate}
\item
  \begin{align*}
    A =
    \begin{bmatrix}
      -10 & 6 & 2 & 8 \\
      -1 & 3 & 4 & 5 \\
      0 & -2 & 0 & -9
    \end{bmatrix}
    \quad
    \mathbf v = \begin{bmatrix} 4 \\ -5 \\ 3 \\ 1 \end{bmatrix}
\end{align*}
\item
  \begin{align*}
    A =
    \begin{bmatrix}
      6 & 1 & -8 & -3 \\
      5 & 0 & -9 & -4
    \end{bmatrix}
    \quad
    \mathbf v = \begin{bmatrix} -6 \\ 2 \end{bmatrix}
\end{align*}
\end{enumerate}
Determine a general form solution for the matrix equation $A\mathbf x = \mathbf b$ where $A$ and $\mathbf b$ are given below.
If this equation has no solutions then write \textit{NO SOLUTION}.
\begin{enumerate}[resume]
\item
  \begin{align*}
    A =
    \begin{bmatrix}
      1 & -3 & 1 & -9 \\
      1 & -2 & 0 & -5 \\
      -3 & 8 & -1 & 19 \\
      -2 & 4 & 0 & 10
    \end{bmatrix}
    \quad
    \mathbf b = \begin{bmatrix} 3 \\ 7 \\ -18 \\ -14 \end{bmatrix}
\end{align*}
\item \faCalculator
  \begin{align*}
    A =
    \begin{bmatrix}
      1 & 1 & 3 & 1 & -7 & 1 \\
      -5 & -4 & -14 & -3 & 24 & -8\\
      0 & -2 & -2 & -3 & 21 & 3 \\
      -6 & -3 & -15 & -2 & 11 & -8
    \end{bmatrix}
    \quad
    \mathbf b =
    \begin{bmatrix}
      7 \\ -16 \\ -20 \\ -27
    \end{bmatrix}
  \end{align*}
\end{enumerate}
For each of the following matrices, Determine if its columns have full span, i.e., given a matrix in $\mathbb R^{m \times n}$, determine if the columns span $\mathbb R^m$.
\begin{enumerate}
\item
  \begin{align*}
    \begin{bmatrix}
      1 & 2 \\
      2 & 5
    \end{bmatrix}
  \end{align*}
\item
  \begin{align*}
    \begin{bmatrix}
      1 & -5 & 4\\
      -1 & 6 & -3\\
      -2 & 13 & -7\\
    \end{bmatrix}
  \end{align*}
\item
  \begin{align*}
    \begin{bmatrix}
      1 & 2 & 0 & 4 \\
      -1 & -1 & 2 & 9 \\
      -1 & -2 & 1 & 1 \\
      0 & 2 & 6 & 36
    \end{bmatrix}
  \end{align*}
\item
  \begin{align*}
    \begin{bmatrix}
      8 & 3 & 8 \\
      8 & -4 & 4 \\
      -2 & 1 & -5 \\
      -5 & -7 & -10 \\
      -8 & 9 & 1
    \end{bmatrix}
  \end{align*}
\end{enumerate}

\section{True/False}

\trueFalseHeader

\begin{enumerate}[resume]
\item
  For $A \in \mathbb R^{m \times n}$ where $m > n$, it is not possible for the columns of $A$ to span $\mathbb R^m$.
\item
  For $A \in \mathbb R^{m \times n}$, if $A \mathbf x = \mathbf b$ is inconsistent for some vector $\mathbf b$, then it is not possible for the columns of $A$ to span $\mathbb R^m$.
\item
  For $A \in \mathbb R^{m \times n}$, if $A \mathbf x = \mathbf 0$ has infinitely many solutions, then the columns of $A$ span $\mathbb R^m$.
\item
  For $A \in \mathbb R^{m \times n}$, if $A \mathbf x = \mathbf 0$ has infinitely many solutions, then $\{ \mathbf v : A\mathbf v = \mathbf 0\} = \mathbb R^n$.
\item
  For matrices $A$ and $B$ in $\mathbb R^{m \times n}$, let $[A \ B] \in \mathbb R ^ {m \times 2n}$ be the matrix obtained by horizontally stacking $A$ and $B$.
  If $A\mathbf x = \mathbf b$ is consistent and $B\mathbf x = \mathbf b$ is consistent, then so is $[A \ B] \mathbf x = \mathbf b$.
\item
  For matrices $A$ and $B$ in $\mathbb R^{m \times n}$, if $[A \ B]\mathbf x = \mathbf b$ is consistent then $A\mathbf x = \mathbf b$ is consistent or $B \mathbf x = \mathbf b$ is consistent.
\item
  For $A \in \mathbb R^{m \times n}$, if $A \mathbf x = \mathbf b$ is inconsistent for some vector $\mathbf b$, then $A$ does not have a pivot position in every column.
\end{enumerate}

\section{More Difficult Problems}

\begin{enumerate}
\item
  \begin{displaymath}
    A
    \begin{bmatrix}
      1 \\ 1 \\ 1 \\ 1
    \end{bmatrix}
    =
    \begin{bmatrix}
      1 \\ 2 \\ 3 \\ 4
    \end{bmatrix}
  \end{displaymath}
  For each of the following shapes, determine a concrete matrix $A$ so that the above equation holds, where $\blacksquare$ represents a nonzero entry.
  {
    \newcommand{\bs}{\blacksquare}
    \begin{enumerate}
    \item
      \begin{displaymath}
        \begin{bmatrix}
          \bs&0&0&0 \\
          0&\bs&0&0 \\
          0&0&\bs&0 \\
          0&0&0&\bs
        \end{bmatrix}
      \end{displaymath}
    \item
      \begin{displaymath}
        \begin{bmatrix}
          \bs&0&0&0 \\
          \bs&\bs&0&0 \\
          \bs&\bs&\bs&0 \\
          \bs&\bs&\bs&\bs
        \end{bmatrix}
      \end{displaymath}
    \item
      \begin{displaymath}
        \begin{bmatrix}
          \bs&\bs&\bs&\bs \\
          \bs&\bs&\bs&0 \\
          \bs&\bs&0&0 \\
          \bs&0&0&0 \\
        \end{bmatrix}
      \end{displaymath}
    \end{enumerate}
  }
\item \faCalculator \
  Any three distinct points in the plane define a \textit{unique} quadratic equation.
  Given three points $(x_1, y_1)$, $(x_2, y_2)$ and $(x_3, y_3)$, the equation $y = ax^2 + bx + c$ that passes through these three points is given by the solution to the following matrix equation.
  \begin{align*}
    \begin{bmatrix}
      x_1^2 & x_1 & 1	\\
      x_2^2 & x_2 & 1	\\
      x_3^2 & x_3 & 1
    \end{bmatrix}
    \begin{bmatrix} a \\ b \\ c \end{bmatrix}
    =
    \begin{bmatrix} y_1 \\ y_2 \\ y_3 \end{bmatrix}
  \end{align*}
  Use it to determine the unique quadratic equation which passes through the points $(2, 13)$, $(3, 25)$ and $(-2, 5)$.
\item
  Determine the RREF of the following matrix in terms of $x$, $y$, assuming $x \not = 1$.
  \begin{align*}
    \begin{bmatrix}
      x^2 & x & 1 & y \\
      1 & 1 & 1 & 1 \\
      \left(\frac{x + 1}{2}\right)^2 & \frac{x + 1}{2} & 1 & \frac{y + 1}{2}
    \end{bmatrix}
  \end{align*}
  \textit{Hint:} Don't try to row reduce it. Think in terms of polynomial interpolation.
\end{enumerate}
\section{Challenge Problems}
