\section{Basic Exercises}

Determine if the following collections of vectors are linearly dependent.
If they are write a dependence relation, i.e., determine linear combination of the given vectors which sums to $\mathbf 0$.
\begin{enumerate}
\item
  \begin{align*}
    \mathbf{v}_{1} = \begin{bmatrix} 1 \\ -1 \end{bmatrix}
    \quad
    \mathbf{v}_{2} = \begin{bmatrix} -1 \\ 2 \end{bmatrix}
  \end{align*}
\item
  \begin{align*}
    \mathbf v_1 =
    \begin{bmatrix}
      1 \\ -1 \\ 2
    \end{bmatrix}
    \qquad
    \mathbf v_2 =
    \begin{bmatrix}
      -1 \\ 4 \\ -3
    \end{bmatrix}
    \qquad
    \mathbf v_3 =
    \begin{bmatrix}
      -3 \\ 9 \\ -8
    \end{bmatrix}
  \end{align*}
\item
  \begin{align*}
    \mathbf{v}_{1} = \begin{bmatrix} 1 \\ -2 \\ 1 \end{bmatrix}
    \quad
    \mathbf{v}_{2} = \begin{bmatrix} -3 \\ 7 \\ -2 \end{bmatrix}
    \quad
    \mathbf{v}_{3} = \begin{bmatrix} -6 \\ 15 \\ -3 \end{bmatrix}
    \quad
    \mathbf{v}_{4} = \begin{bmatrix} -1 \\ 3 \\ 1 \end{bmatrix}
  \end{align*}
\end{enumerate}

\section{True/False}
\trueFalseHeader
\begin{enumerate}[resume]
\item
  For $A \in \mathbb R^{m \times n}$, if the columns of $A$ are linearly dependent, then they do not span $\mathbb R^m$.
\end{enumerate}

\section{More Difficult Problems}

\begin{enumerate}[resume]
\item
  Consider four vectors $\mathbf v_1$, $\mathbf v_2$, $\mathbf
  v_3$ and $\mathbf v_4$ in $\mathbb R^4$ with the property that
  \begin{align*}
    \begin{bmatrix}
      \mathbf v_1 & \mathbf v_2 & \mathbf v_3 & \mathbf v_4
    \end{bmatrix}
    \sim
    \begin{bmatrix}
      1 & 1 & 3 & -2 \\
      0 & 4 & -4 & 12 \\
      0 & 0 & 3 & 3 \\
      0 & 0 & 0 & 0
    \end{bmatrix}
  \end{align*}
  \begin{enumerate}
  \item
    Determine if $\mathbf v_4 \in \mathrm{span}\{\mathbf v_1, \mathbf v_2, \mathbf v_3\}$.
    If so, express $\mathbf v_4$ as a linear combination of the vectors $\mathbf v_1$, $\mathbf v_2$ and $\mathbf v_3$.
  \item Deterime if vectors $\mathbf v_2$, $\mathbf v_3$, and $\mathbf v_4$ linearly independent.
    If so, justify your answer.
    If not, determine a dependence relation for these vectors.
  \end{enumerate}
\item
  Determine all values of $h$ for which the following set of vectors is linearly dependent.
  \begin{align*}
    \mathbf v_1 =
    \begin{bmatrix} 1 \\ 9 \\ -3 \end{bmatrix}
    \qquad
    \mathbf v_2 =
    \begin{bmatrix} 1 \\ 4 \\ 3 \end{bmatrix}
    \qquad
    \mathbf v_3 =
    \begin{bmatrix} -2 \\ h \\ -6 \end{bmatrix}
    \qquad
  \end{align*}
\item
  Determine three \textit{nonzero} vectors $\mathbf v_1$, $\mathbf v_2$, and $\mathbf v_3$ in $\mathbb R^3$ such that
  \begin{itemize}
  \item
    $\{\mathbf v_1, \mathbf v_2, \mathbf v_3\}$ is linearly dependent and
  \item
    $\mathbf v_1$ cannot be written as a linear combination of $\mathbf v_2$ and $\mathbf v_3$.
  \end{itemize}
\item \faCalculator \
  Consider the following matrix, presented as a SymPy array.
  \begin{lstlisting}

  import sympy
  a = sympy.Matrix(
      [[ 13,  -19,  19,  16,   5,   1,  10,   5,  15],
       [-11,   -7, -10,   7,   2,  -8, -10, -19,   6],
       [ -5,   10,  -7,   2,  -8,   2, -15, -16, -11],
       [ 17,  -13,   9,  13,  19,   8,  -3,  -9,   0],
       [  9,  -18,   5,   1,   4,  14,   9,   8,  -4],
       [  8,   14,  17,   5,  -6,   7, -13,   2,  12],
       [ 18,   12,  -7,   2, -10,  15, -12,   1, -12],
       [ 19,  -12,   1, -16,   2,  -6,  -4,  17,  15],
       [-19,   -6, -16, -20, -20,  -3,   7,   3,  14],
       [  6,    8, -15,   5,  -5,   8, -14,   5, -19]])
  \end{lstlisting}
  \begin{enumerate}
  \item Determine if the columns of this matrix span all of $\mathbb R^{10}$. Justify your answer.
  \item Determine if the columns of this matrix linearly independent. Justify your answer.
  \end{enumerate}
\item
  Consider the following collection of vectors.
  \begin{align*}
    &\mathbf v_1 =
    \begin{bmatrix} -10 \\ -6 \\ 2 \\ 0 \\ 8 \\ -3 \\ -4 \end{bmatrix}
    \qquad
    \mathbf v_2 =
    \begin{bmatrix} -3 \\ 3 \\ -8 \\ 5 \\ -2 \\ 7 \\ -3 \end{bmatrix}
    \qquad
    \mathbf v_3 =
    \begin{bmatrix} -20 \\ -12 \\ 4 \\ 0 \\ 16 \\ -6 \\ -8 \end{bmatrix}
    \qquad
    \mathbf v_4 =
    \begin{bmatrix} 6 \\ 1 \\ 3 \\ -1 \\ 10 \\ 7 \\ 4 \end{bmatrix}
  \end{align*}
  \begin{align*}
    &\mathbf v_5 =
    \begin{bmatrix} -2 \\ -10 \\ 10 \\ -5 \\ -3 \\ -10 \\ -5 \end{bmatrix}
    \qquad
    \mathbf v_6 =
    \begin{bmatrix} -2 \\ 3 \\ -9 \\ -2 \\ -9 \\ -2 \\ 7 \end{bmatrix}
    \qquad
    \mathbf v_7 =
    \begin{bmatrix} -13 \\ 25 \\ -22 \\ 13 \\ 24 \\ 41 \\ 15 \end{bmatrix}
  \end{align*}
  \begin{enumerate}
  \item
    Determine the first vector (i.e., the vector with the smallest index) which can be written as a linear combination as of the vectors which precede it.
    Express this vectors as a linear combination of the vectors which precede it.
    You should write your final solution using the vector names of the form $\mathbf v_i$.
  \item
    Determine a dependence relation for the entire set of vectors with the
    following properties:
    \begin{itemize}
    \item
      the coefficients of the dependence relation are relatively prime integers;
    \item
      the number of nonzero coefficients is maximum, i.e., there is no other dependence relation with a greater number of nonzero coefficients.
    \end{itemize}
  \end{enumerate}
\end{enumerate}

\section{Challenge Problems}

\begin{enumerate}[resume]
\item
  Determine all values of $h$ for which the following set of vectors is linearly dependent.
  \begin{align*}
    \mathbf v_1 =
    \begin{bmatrix} 1 \\ -1 \\ h \end{bmatrix}
    \qquad
    \mathbf v_2 =
    \begin{bmatrix} 4 \\ 0 \\ 2 \end{bmatrix}
    \qquad
    \mathbf v_3 =
    \begin{bmatrix} 13 \\ h \\ -1 \end{bmatrix}
    \qquad
  \end{align*}
\end{enumerate}
