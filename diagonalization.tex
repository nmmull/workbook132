\section{Basic Exercises}

Determine a diagonalization of the following matrix, if it exists.
You can leave the rightmost factor in the form $P^{-1}$, i.e., you
don't have to compute the inverse of $P$.
\begin{enumerate}
\item
  \begin{displaymath}
    \begin{bmatrix}
      9 & 4 \\
      -14 & -6
    \end{bmatrix}
  \end{displaymath}
\item
  \begin{displaymath}
    \begin{bmatrix}
      2 & -6 \\
      2 & -5
    \end{bmatrix}
  \end{displaymath}
\item
  \begin{displaymath}
    \begin{bmatrix}
      -6 & -7\\
      7 & 8
    \end{bmatrix}
  \end{displaymath}
\item
  \begin{displaymath}
    \begin{bmatrix}
      -2 & 0 & 0 \\
      2 & -1 & -1 \\
      6 & 4 & -5
    \end{bmatrix}
  \end{displaymath}
\item
\begin{displaymath}
  \begin{bmatrix}
    -1 & -3 & -6 \\
    -8 & -6 & -18 \\
    4 & 3 & 9
  \end{bmatrix}
\end{displaymath}
\textit{Hint:} The characteristic polynomial of the above matrix is
$\lambda^{3} - 2 \lambda^{2} - 3 \lambda$.
\item
  \begin{displaymath}
    \begin{bmatrix}
      2 & 0 & 0 & 0 \\
      -1 & 3 & 0 & 0 \\
      3 & -3 & 2 & 0 \\
      -3 & 18 & -6 & -1
    \end{bmatrix}
  \end{displaymath}
\item
  \begin{displaymath}
    \begin{bmatrix}
      -1 & -3 & 0 \\
      2 & 4 & 0 \\
      0 & 0 & 1
    \end{bmatrix}
  \end{displaymath}
\item \faCalculator
  \begin{displaymath}
    \begin{bmatrix}
      0 & 2 & 2 \\
      5 & -6 & -5 \\
      -9 & 12 & 11
    \end{bmatrix}
  \end{displaymath}
  In addition, define $P$ so that its entries are integers.
\end{enumerate}

\section{True/False}

\trueFalseHeader

\begin{enumerate}[resume]
\item
  All square matrices are diagonalizable.
\item
  Similar matrices have the same eigenvalues.
\item
  Similar matrices have the same eigenvectors.
\item
  If a matrix does not have $n$ distinct eigenvalues, then it is not
  diagonalizable.
\item
  If a matrix is diagonalizable, then it is invertible.
\item
  A diagonalization of a matrix $A$ if it exists, is unique.
\end{enumerate}

\section{More Difficult Problems}

\begin{enumerate}
\item \faCalculator \
  Consider the following matrix.
  \begin{displaymath}
    A =
    \begin{bmatrix}
      103 & -8 & -47 \\
      24 & 1 & -12 \\
      198 & -16 & -90
    \end{bmatrix}
  \end{displaymath}
  Determine a matrix $B$ such that $B^2 = A$. You can use a computer
  to find eigenvectors or to reduce matrices, but you must otherwise
  show your work and justify your answer.  \textit{Hint:} $A$ is
  diagonalizable.
\end{enumerate}

\section{Challenge Problems}

\begin{enumerate}[resume]
\item
  Consider an arbitrary $2 \times 2$ matrix of the following form.
  \begin{displaymath}
    A =
    \begin{bmatrix}
      a & b \\
      c & d
    \end{bmatrix}
  \end{displaymath}

  \begin{enumerate}
  \item
    Determine an expression for the characteristic polynomial of $A$
    in terms of $a$, $b$, $c$, and $d$.
  \item
    Recall that for a quadratic polynomial $p(x) = ix^2 + jx + k$, the
    discriminant $j^2 - 4ik$ tells us how many roots $p$ has, i.e.,
    $p$ has $0$, $1$ or $2$ roots if the discriminant is less than
    $0$, equal to $0$, or greater than $0$, respectively.  Use this to
    derive an expression $E$ in terms of $a$, $b$, $c$, and $d$ where
    $A$ has $0$, $1$, or $2$ eigenvalues if $E < 0$, $E = 0$, or $E >
    0$, respectively.
  \item Using the expression from the previous part, argue that
    \textit{every $2 \times 2$ matrix with positive entries has two
      distinct eigenvalues.}  Note that this implies every $2 \times
    2$ matrix with positive entries is diagonalizable.  \textit{Hint:}
    Try to write the expression from the previous part so that it is
    of the form `$(\square - \square)^2 + 4\square\square$' and reason
    about why this must be positive.
  \end{enumerate}
\end{enumerate}
