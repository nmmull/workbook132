\section*{Basic Exercises}

For each of the following linear transformations, its domain, codomain, and the matrix that implements it (i.e., determine the matrix $A$ such that the given transformation is $\mathbf x \mapsto A \mathbf x$).
\begin{enumerate}
\item
  \begin{displaymath}
    \begin{bmatrix}
      x_1 \\ x_2 \\ x_3
    \end{bmatrix}
    \mapsto
    \begin{bmatrix}
      x_1 + x_2 \\
      -2x_1 - x_3 \\
      x_1 + x_3
    \end{bmatrix}
  \end{displaymath}
\item
  \begin{align*}
    \begin{bmatrix} x_{1} \\ x_{2} \\ x_{3} \\ x_{4} \\ x_{5} \end{bmatrix}
    \mapsto
    \begin{bmatrix}
      -9x_{1} + 4x_{2} + 5x_{3} - 3x_{4} + 2x_{5} \\
      3x_{1} + 8x_{2} + 5x_{3} - 5x_{4} + 7x_{5} \\
      -x_{1} + 5x_{2} - x_{3} + 4x_{4} + 8x_{5}
    \end{bmatrix}
  \end{align*}
\item
  \begin{align*}
    \begin{bmatrix}
      x_1 \\ x_2 \\ x_3 \\ x_4
    \end{bmatrix}
    \mapsto
    \begin{bmatrix}
      -2x_2 + x_3 + x_4 \\ x_1 + x_3 \\ -3x_3 - 3x_4 \\ 9x_4
    \end{bmatrix}
  \end{align*}
\end{enumerate}
Let $T$ be a linear transformation with the following input-output behavior.
\begin{align*}
  T(\mathbf v_{1}) = \begin{bmatrix} -6 \\ 3 \\ -2 \\ -10 \end{bmatrix}
  \quad
  T(\mathbf v_{2}) = \begin{bmatrix} -5 \\ 1 \\ -2 \\ 9 \end{bmatrix}
  \quad
  T(\mathbf v_{3}) = \begin{bmatrix} 8 \\ -7 \\ 6 \\ 6 \end{bmatrix}
\end{align*}
Compute the following.
\begin{enumerate}[resume]
\item
  $T(\mathbf v_1 + \mathbf v_2)$
\item
  $T(-3\mathbf v_{1} - \mathbf v_{2} - 2\mathbf v_{3})$.
\end{enumerate}
Let $T$ be a linear transformation with the following input-output behavior.
\begin{align*}
  T
  \left(
  \begin{bmatrix}
    -3 \\ 9 \\ 5 \\ 4
  \end{bmatrix}
  \right)
  =
  \begin{bmatrix}
    -4 \\ 3 \\ 4
  \end{bmatrix}
  \qquad
  T
  \left(
  \begin{bmatrix}
    -9 \\ -5 \\ 0 \\ -7
  \end{bmatrix}
  \right)
  =
  \begin{bmatrix}
    0 \\ 1 \\ 2
  \end{bmatrix}
  \qquad
  T
  \left(
  \begin{bmatrix}
    9 \\ -1 \\ -2 \\ 4
  \end{bmatrix}
  \right)
  =
  \begin{bmatrix}
    3 \\ -3 \\ 7
  \end{bmatrix}
\end{align*}
Compute the following
\begin{enumerate}[resume]
\item \faCalculator \
  \begin{align*}
    T
    \left(
    \begin{bmatrix}
      30 \\ 0 \\ -7 \\ 22
    \end{bmatrix}
    \right)
  \end{align*}
\item \faCalculator \
  \begin{align*}
    T
    \left(
    \begin{bmatrix}
      1 \\ 0 \\ 0 \\ 0
    \end{bmatrix}
    \right)
  \end{align*}
\end{enumerate}
Determine the matrix that implements the linear transformation $T$, given its input-output behavior.
\begin{enumerate}[resume]
\item
  \begin{align*}
    T\left(\begin{bmatrix} 1 \\ -2 \end{bmatrix}\right)
    = \begin{bmatrix} -3 \\ -1 \end{bmatrix}
    \quad
    T\left(\begin{bmatrix} 2 \\ -3 \end{bmatrix}\right)
    = \begin{bmatrix} -4 \\ -1 \end{bmatrix}
  \end{align*}
\item
  \begin{align*}
    T\left(\begin{bmatrix} 1 \\ -2 \\ -3 \end{bmatrix}\right)
    = \begin{bmatrix} 1 \\ 1 \\ 1 \end{bmatrix}
    \quad
    T\left(\begin{bmatrix} 2 \\ -3 \\ -4 \end{bmatrix}\right)
    = \begin{bmatrix} 1 \\ 2 \\ 3 \end{bmatrix}
    \quad
    T\left(\begin{bmatrix} 2 \\ -4 \\ -5 \end{bmatrix}\right)
    = \begin{bmatrix} 2 \\ 1 \\ 3 \end{bmatrix}
  \end{align*}
\item
  \begin{align*}
    T\left(\begin{bmatrix} 2 \\ 1 \\ 0 \end{bmatrix} \right)
    = \begin{bmatrix} 3 \\ -3 \\ 4 \end{bmatrix}
    \qquad
    T\left(\begin{bmatrix} 1 \\ 1 \\ 0 \end{bmatrix} \right)
    = \begin{bmatrix} 8 \\ 0 \\ 2 \end{bmatrix}
    \qquad
    T\left(\begin{bmatrix} 2 \\ 1 \\ 2 \end{bmatrix} \right)
    = \begin{bmatrix} -1 \\ -1 \\ -1 \end{bmatrix}
  \end{align*}
\item
  \begin{align*}
    T \left(\begin{bmatrix} -7 \\ -3 \end{bmatrix} \right)
    = \begin{bmatrix} -4 \\ 3 \\ 4 \\ 0 \\ 1 \end{bmatrix}
    \qquad
    T \left(\begin{bmatrix} 9 \\ 4 \end{bmatrix} \right)
    = \begin{bmatrix} 13 \\ 0 \\ 0 \\ 8 \\ 3 \end{bmatrix}
  \end{align*}
\end{enumerate}
For each matrix, determine if its transformation is (a) one-to-one but not onto, (b) onto but not one-to-one, (c) both, or (d) neither. Justify your answer.
\begin{enumerate}[resume]
\item
  \begin{align*}
    \begin{bmatrix}
      1 & 2 & 1 & 3 \\
      1 & 3 & 1 & 3 \\
      1 & 5 & 1 & 4
    \end{bmatrix}
  \end{align*}
\item
  \begin{align*}
    \begin{bmatrix}
      1 & 1 & 2 \\
      2 & 3 & 5 \\
      -1 & -3 & -3
    \end{bmatrix}
  \end{align*}
\item
  \begin{align*}
    \begin{bmatrix}
      5 & 0 & 5 \\
      0 & 0 & 0 \\
      4 & 0 & 7
    \end{bmatrix}
  \end{align*}
\item
  \begin{align*}
    \begin{bmatrix}
      1 & 2 \\
      0 & 1 \\
      -2 & -2 \\
      1 & 3 \\
      1 & 4
    \end{bmatrix}
  \end{align*}
\item
  \begin{align*}
    \begin{bmatrix}
      1 \\ 1 \\ 1 \\ 1 \\ 1 \\ 1
    \end{bmatrix}
  \end{align*}
\item
  \begin{align*}
    \begin{bmatrix}
      2 & 0 & 0 \\
      0 & 1 & 3 \\
      0 & 0 & 3 \\
      0 & 0 & 3
    \end{bmatrix}
  \end{align*}
\item
  \begin{align*}
    \begin{bmatrix}
      1 & 2 & -1 & 6 & 9 \\
      2 & 5 & -5 & 13 & 22 \\
      -3 & -4 & -3 & -16 & -19
    \end{bmatrix}
  \end{align*}
\item
  \begin{align*}
    \begin{bmatrix}
      2 & 1 & 3 & 0 \\
      1 & 4 & 3 & 0 \\
      0 & 0 & 1 & 0
    \end{bmatrix}
  \end{align*}
\item
  \begin{align*}
    \begin{bmatrix}
      1 & -2 & 3 & -4 \\
      0 & 5 & -6 &  7 \\
      0 & 0 & -8 & 9 \\
      0 & 0 & 0 & -10
    \end{bmatrix}
  \end{align*}
\item
  \begin{align*}
    \begin{bmatrix}
      12 & 45 & -3 & 20 & 1 \\
      0 & 24 & 121 & 0 & -47 \\
      0 & 0 & 252 & 44 & 46 \\
      0 & 0 & 0 & 21 & -44 \\
      0 & 0 & 0 & 0 & 11
    \end{bmatrix}
  \end{align*}
\item
  \begin{align*}
    \begin{bmatrix}
      1 & 2 & 3 & 4 \\
      -13 & 6 & -39 & 4 \\
      2 & 9 & 6 & 7 \\
      5 & 4  & 15 & 16 \\
      0 & 1 & 0 & 65 \\
      1 & 1 & 1 & 1 \\
      -12 & 3 & -36 & 33 \\
      0 & 0 & 0 & 0
    \end{bmatrix}
  \end{align*}
\item
\end{enumerate}
Draw the image of the unit square under the following linear transformations.
\begin{enumerate}[resume]
\item
  \begin{align*}
    \begin{bmatrix} x_1 \\ x_2 \end{bmatrix}
    \mapsto
    \begin{bmatrix}
      -2 & -2 \\
      -1 & 0
    \end{bmatrix}
    \begin{bmatrix} x_1 \\ x_2 \end{bmatrix}
  \end{align*}
\end{enumerate}

\section*{True/False}

\trueFalseHeader

\begin{enumerate}[resume]
\item
  If $\mathbf x \mapsto A \mathbf x$ is one-to-one and onto, then $A$ is a square matrix.
\item
  If $\mathbf x \mapsto A \mathbf x$ is neither one-to-one or onto, and the reduced echelon form of $A$ only has $0$s and $1$s, then one of the columns of $A$ is $\mathbf 0$.
\item
  For any matrix $A \in \mathbb R^{m \times n}$ where $m > n$, it is not possible for the transformation $\mathbf x \mapsto A \mathbf x$ to be one-to-one.
\end{enumerate}


\begin{enumerate}[resume]
\item
  If $T : \mathbb R^m \to \mathbb R^n$ is a linear transformation and $T(\mathbf v) = A \mathbf v$ for all vectors $\mathbf v$ in the domain of $T$, then $A$ is an $m \times n$ matrix.
\item
  The matrix that implements a linear transformation is unique.
\end{enumerate}

\section*{More Difficult Problems}

\begin{enumerate}[resume]
\item
  Determine the matrix which implements the linear tranformation $T : \mathbb R^3 \to \mathbb R^3$ that reflects vectors across the $x_1x_2$-plane.
\item
  Determine the matrix which implements the linear transformation $T : \mathbb R^2 \to \mathbb R^4$ that repeats the input vector, e.g.,
  \begin{align*}
    T\left(\begin{bmatrix} 1 \\ 2 \end{bmatrix} \right)
    = \begin{bmatrix} 1 \\ 2 \\ 1 \\ 2 \end{bmatrix}
  \end{align*}
\item
  Determine the matrix which implements the linear transformation $T : \mathbb R^2 \to \mathbb R^2$ that reflects vectors across the line $y = x$.
\item
  Determine the matrix which implements the linear transformation $T : \mathbb R^2 \to \mathbb R^2$ that reflects vectors across the line $y = \tan(\frac{3\pi}{8})x$.
\item
  Determine the matrix which implements the transformation $T : \mathbb R^4 \to \mathbb R^4$ that swaps the first and second entries of its input, e.g.,
  \begin{align*}
    T \left(\begin{bmatrix} 1 \\ 2 \\ 3 \\ 4 \end{bmatrix} \right)
    = \begin{bmatrix} 2 \\ 1 \\ 3 \\ 4 \end{bmatrix}
  \end{align*}
\item
  Consider the following transformation.
  \begin{align*}
    \begin{bmatrix}
      x \\ y \\ z
    \end{bmatrix}
    \mapsto
    \begin{bmatrix}
      (x^3 + y^3 + z^3)^{1 / 3} \\ y + z
    \end{bmatrix}
  \end{align*}
  \begin{enumerate}
  \item
    Demonstrate that the above transformation is homogeneous.
  \item
    Determine if the above transformation is not linear.
    Justify your answer.
  \end{enumerate}
\item
  Show that the transformation $T: \mathbb R^4 \to \mathbb R^4$ given by
  \begin{displaymath}
    \begin{bmatrix}
      v_1 \\ v_2 \\ v_3 \\ v_4
    \end{bmatrix}
    \mapsto
    \begin{bmatrix}
      \min(v_1, 100) \\
      \min(v_2, 100) \\
      \min(v_3, 100) \\
      \min(v_4, 100) \\
    \end{bmatrix}
  \end{displaymath}
  is not linear.
\end{enumerate}

\section*{Challenge Problems}

\begin{enumerate}[resume]
\item
  Determine the matrices which implement the linear transformations that rotates vectors in $\mathbb R^3$ 120 degrees about
  \begin{align*}
    \mathrm{span}\left\{\begin{bmatrix}1 \\ 1 \\ 1\end{bmatrix}\right\}
  \end{align*}
  You should determine \textit{two} matrices, one for clockwise rotation and the other for counterclockwise rotation.
\item
  Determine the matrices which implement the linear transformations that reflects vectors across the plane defined by the linear equation $x + y + z = 0$, and then rotates vectors 60 degrees about
  \begin{align*}
    \mathrm{span}\left\{\begin{bmatrix}1 \\ 1 \\ 1\end{bmatrix}\right\}
  \end{align*}
  You should determine \textit{two} matrices, one for clockwise rotation and the other for counterclockwise rotation.
\item
  Determine the matrix which implements the linear transformation $T : \mathbb R^3 \to \mathbb R^3$ that reflects vectors across the plane defined by the linear equation $x + y + z = 0$.
\item
  Consider the following $\mathbb R^3$ rotation matrices.
  \begin{displaymath}
    A =
    \begin{bmatrix}
      \cos 45^\circ & -\sin 45^\circ & 0 \\
      \sin 45^\circ & \cos 45^\circ & 0 \\
      0 & 0 & 1
    \end{bmatrix}
    \qquad
    B =
    \begin{bmatrix}
      1 & 0 & 0 \\
      0 & \cos 180^\circ & -\sin 180^\circ \\
      0 & \sin 180^\circ & \cos 180^\circ
    \end{bmatrix}
  \end{displaymath}
  The matrix $A$ rotates vectors around the $x_3$-axis by 45 degrees, and $B$ rotates vectors around the $x_1$-axis by 180 degrees.
  \begin{enumerate}
  \item
    Determine $A^{-1}$.
  \item
    Calculate $ABA^{-1}$.
  \item
    Describe what the transformation implemented by $ABA^{-1}$ does geometrically.
  \end{enumerate}
\end{enumerate}
