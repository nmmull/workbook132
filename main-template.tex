\documentclass[10pt]{report}

%% % palatino
%% \usepackage{mathpazo}

%% % 1 inch margins
%% \usepackage[margin=1in]{geometry}

% better quotes
\usepackage{csquotes}

%% standard math tools
\usepackage{amsthm, amssymb, mathtools}

%% solution box
\newcommand{\widebox}[1]{\begin{center}\noindent\fbox{\parbox{0.75\linewidth}{#1}}\end{center}}
\newcommand{\solutionbox}[1]{\widebox{\textit{Solution.} {#1}}}

%% better bullets/enumeration
\usepackage{enumitem}
%% \renewcommand{\theenumi}{\Alph{enumi}}
%% \renewcommand{\labelitemi}{$\triangleright$}



\usepackage{hyperref}
\usepackage{cleveref}

\title{Problems and Exercises}
\author{\texttt{CAS CS 132}: \textit{Geometric Algorithms}}
\begin{document}
\maketitle
\tableofcontents

\chapter*{Preface}

TODO. write a preface.

\chapter{Linear Systems}

\section{Definitions}

\section{Basic Problems}

For each of the following linear systems, determine its coefficient matrix and its augmented matrix.
\begin{enumerate}
\item
  \begin{align*}
    2 x_{1} - 6 x_{2} &= -4 \\
    6 x_{1} + 8 x_{2} &= -7
  \end{align*}
\item
  \begin{align*}x_{1} + 8 x_{2} + 7 x_{3} &= -1 \\
    4 x_{1} - 9 x_{2} &= 8 \\
    7 x_{1} - 7 x_{2} - 3 x_{3} &= 10
  \end{align*}
\item
  \begin{align*}
    x_{1} - 2x_{2} - 2x_{3} &= 2 \\
    2x_{1} - 3x_{2} - 5x_{3} &= 2 \\
    -2x_{1} + 2x_{2} + 7x_{3} &= -1
  \end{align*}
\item
  \begin{align*}
    - 8 x_{1} + 6 x_{2} + x_{3} + 7 x_{4} + 10 x_{5} &= 8 \\
    - 4 x_{2} - 4 x_{3} - x_{4} + x_{5} &= -8 \\
    7 x_{1} + x_{2} - 2 x_{3} - 8 x_{4} + 7 x_{5} &= 9 \\
    - 4 x_{1} - 7 x_{2} + 5 x_{3} + 2 x_{4} - 9 x_{5} &= -4
  \end{align*}
\end{enumerate}
For each of the following linear systems, verify that the given point is a solution.
\begin{enumerate}[resume]
\item
  \begin{align*}
    x_{1} - 2x_{2} + x_{3} - 2x_{4} &= -9 \\
    x_{1} - x_{2} - x_{3} - 2x_{4} &= -10 \\
    -3x_{1} + 8x_{2} - 6x_{3} + 4x_{4} &= 21 \\
    2x_{2} - 7x_{3} + 7x_{4} &= 13 \\
  \end{align*}
  \begin{align*}
    s = (1, 3, 2, 3)\\
  \end{align*}
\end{enumerate}
For each of the following linear systems, demonstrate that it has a unique solution.
Additionally, determine the solution.
\begin{enumerate}[resume]
\item
  \begin{align*}
    3 x_{1} + 6 x_{2} &= -15 \\
    - x_{1} + x_{2} &= 5
  \end{align*}
\item
  \begin{align*}
    2x + 4y &= 29 \\
    -6x - 10y &= -75
  \end{align*}
\item
  \begin{align*}
    x + 3y - 3z &= -2 \\
    2x + 7y + 7z &= -5 \\
    x - 2y - z &= 3
  \end{align*}
\item
  \begin{align*}
    3x - 4y + 3z &= -9 \\
    6x + 7y - 3z &= 0 \\
    x + 10z &= -21
\end{align*}
\item
  \begin{align*}
    - 3 x_{1} + 6 x_{3} &= 12 \\
    3 x_{1} + 3 x_{2} - 12 x_{3} &= -21 \\
    2 x_{1} - 6 x_{2} + 10 x_{3} &= 16
  \end{align*}
\item
  \begin{align*}
    - 3 x_{1} + 6 x_{2} &= 9 \\
    3 x_{1} - 8 x_{2} &= -9 \\
    2 x_{2} + 6 x_{3} &= 18
  \end{align*}
\item
  \begin{align*}
    x_{1} - 2x_{2} - 2x_{3} &= -7 \\
    -x_{1} + 3x_{2} + 2x_{3} &= 10 \\
    2x_{1} - 6x_{2} - 3x_{3} &= -18
  \end{align*}
\item
  \begin{align*}
    - 3 x_{1} - 6 x_{3} + 9 x_{4} &= -6 \\
    - x_{1} + 2 x_{2} - 3 x_{4} &= 8 \\
    - 2 x_{1} - 9 x_{2} - 19 x_{3} + 45 x_{4} &= -61 \\
    x_{1} - 3 x_{2} + 5 x_{3} - 9 x_{4} &= 5
  \end{align*}
\end{enumerate}
For each of the following matrices, apply the given row operations from top to bottom.
\begin{enumerate}[resume]
\item
  \begin{align*}
    \begin{bmatrix}
      9 & 5 & -7 & -5 & -9 \\
      5 & -7 & 1 & -2 & -9 \\
      5 & 1 & -10 & 6 & -5 \\
      5 & 7 & -5 & 2 & 1
    \end{bmatrix}
  \end{align*}
  \begin{align*}
    R_{4} &\gets -R_{4} \\
    R_{2} &\gets R_{2} - 2R_{3} \\
    R_{2} &\gets R_{2} - 5R_{4} \\
    R_{3} &\gets R_{3} + 3R_{4} \\
    R_{3} &\leftrightarrow R_{2}
  \end{align*}
\end{enumerate}
For each of the following RREFs, determine a general form solution, i.e., find a general form solution for a linear system whose augmented matrix is row equivalent to the given matrix.
\begin{enumerate}[resume]
  \item
    \begin{align*}
      \begin{bmatrix}
        1 & 1 & 0 & 0 & 0 & -4 & 5 & -3 \\
        0 & 0 & 1 & 0 & 0 & 1 & 3 & -4 \\
        0 & 0 & 0 & 0 & 1 & 5 & -3 & 2 \\
        0 & 0 & 0 & 0 & 0 & 0 & 0 & 0 \\
        0 & 0 & 0 & 0 & 0 & 0 & 0 & 0
      \end{bmatrix}
    \end{align*}
\end{enumerate}
For each of the following matrices, determine its row-reduced echelon form.
You must include the intermediate matrices and row operations you used.
\begin{enumerate}[resume]
  \item
    \begin{align*}
      \begin{bmatrix}
        1 & -1 & -2 & 1 \\
        -1 & 2 & 4 & 0 \\
        2 & -3 & -6 & 2 \\
        -2 & 1 & 2 & -1
      \end{bmatrix}
    \end{align*}
\end{enumerate}
For each of the following linear systems, determine
\begin{itemize}
\item its augmented matrix;
\item the row-reduced echelon form of its augmented matrix;
\item whether it has \textbf{no solutions}, a \textbf{unique solution}, or \textbf{infinitely many solutions}.
\end{itemize}
You may use a calculator/computer (e.g., \texttt{a.rref()} using SymPy) to determine the RREF.
\begin{enumerate}[resume]
\item
  \begin{align*}
    x_1 - 2x_2 + 5x_3 &= 6 \\
    -2x_1 + 6x_2 - 11x_3 + 7x_4 &= -8 \\
    5x_1 - 10 x_2 + 25x_3 + 3x_4 &= 30
  \end{align*}
\item
  \begin{align*}
    x_1 - 3x_2 + 4x_3 &= 0 \\
    -x_1 + 6x_2 + x_3 &= 4 \\
    23x_2 + 5x_3 &= 9
  \end{align*}
\item
  \begin{align*}
    x_1 + 5x_2 + 3x_3 &= -4 \\
    x_1 + 6x_2 + 7x_3 &= -13 \\
    -2x_1 - 12 x_2 - 14 x_3 &= 25
  \end{align*}
\end{enumerate}
For each of the following pairs of matrices, determine a sequence of elementary row operations from the left matrix to the right matrix.
You must including the intermediate matrices and row operations used.
\begin{enumerate}[resume]
\item
  \begin{align*}
    \begin{bmatrix}
      -4 & -7 & 4 \\
      3 & 8 & 2 \\
      -10 & 1 & -9
    \end{bmatrix}
    \sim
    \begin{bmatrix}
      3 & 8 & 2 \\
      -14 & -6 & -5 \\
      -10 & 1 & -9
    \end{bmatrix}
  \end{align*}
\item
  \begin{align*}
    \begin{bmatrix}
      1 & 2 & -1 & 5 \\
      0 & 1 & 0 & 1 \\
      0 & 0 & 0 & 5 \\
      2 & 2 & 1 & 1
    \end{bmatrix}
    \sim
    \begin{bmatrix}
      3 & 4 & 0 & 6 \\
      4 & 4 & 2 & 2 \\
      0 & -1 & 0 & 4 \\
      0 & 1 & 0 & 1
    \end{bmatrix}
  \end{align*}
\end{enumerate}
For each the following linear systems, determine a general form solution for it by first determining the reduced echelon form of its augmented matrix.
If this system has no solutions then write \textit{NO SOLUTION}.
\begin{enumerate}[resume]
\item
  \begin{align*}
    x_{1} - 6x_{2} + x_{4} &= 2 \\
    2x_{1} - 12x_{2} + x_{3} - 4x_{4} &= 7 \\
    x_{1} - 6x_{2} - 2x_{3} + 13x_{4} &= -4
  \end{align*}
\end{enumerate}

\section{True/False}

Determine if each statement is \textbf{true} or \textbf{false} and justify your answer.
In particular, if the statement is false, provide a counterexample if possible.

\begin{enumerate}[resume]
\item
  Elementary row operations cannot change the solution set of a linear system.
\item
  There is a linear system with exactly three solutions.
\item
  If $A$ is the augmented matrix of an inconsistent linear system, and $B$ is a matrix such that $A \sim B$ (that is, $A$ and $B$ are row equivalent), then $B$ is the augmented matrix of an inconsistent linear system.
\item
  If $A \sim B$ and $A \sim C$ and $B$ and $C$ are in reduced echelon form, then $B = C$.
\item
  There is a unique sequence of row operations that reduces a given matrix to reduced echelon form.
\item
  If a general form solution of a linear system has a free variable, then the system must have infinitely many solutions.
\item
  A matrix may have different pivot positions depending on the sequence of row operations used to attain a matrix in echelon form.
\item
  A linear system over 3 variables and 2 equations must be consistent.
\item
  If the coefficient matrix of a linear system as more rows than columns, then the system must have infinitely many solutions.
\end{enumerate}

\section{More Difficult Problems}
\begin{enumerate}[resume]
\item
  Determine the slope-intercept form of the line equation which defines the intersection of the plane
  \begin{align*}
    2x + 3y + 3z = 6
  \end{align*}
  with the $xy$-plane.
\item
  For what values of the coefficient $h$ is the following system inconsistent?
  \begin{align*}
    x + 4y &= -1 \\
    3x - hy &= 7
  \end{align*}
  Is there a value of $h$ for which the above system has infinitely many solutions? Justify your answer.
\item
  Consider the following linear system with two unknown coefficients $h$ and $k$.
  \begin{align*}
    hx + 2y &= 1 \\
    3x + 9y &= k
  \end{align*}
  \begin{enumerate}
  \item Determine values of $h$ and $k$ so that the above linear system has no solutions.
  \item Determine values of $h$ and $k$ so that the above linear system has exactly one solution.
  \item Determine values of $h$ and $k$ so that the above linear system has infinitely many solutions.
  \end{enumerate}
\item
  Consider the following general form solution.
  \begin{align*}
    x_{1} &= -6 + 6x_{3} + 2x_{5} \\
    x_{2} &= 4 + 4x_{3} + 6x_{5} \\
    x_{3} &\text{ is free} \\
    x_{4} &= -4 + 5x_{5} \\
    x_{5} &\text{ is free}
  \end{align*}
  Determine a general form solution that describes the same solution set but in which $x_1$ is free.
\item
  Suppose you're investigating a claim that, out of five competing car companies who purchase engine parts from the same suppliers, one purportedly \textit{overestimated} their total spending by \textdollar 1,300,000.
  Companies are required to report their total spending, and you've been able to determine how many units (× 10,000) of each part that each company has purchased.
  You haven't been unable to determine how much each item costs per unit (it's an industry secret), but you can assume that each company pays the same amount per unit.
  Given the following data, which company is falsifying their records?
  Justify our answer.
  The total amount spent by each company in the table below is multiplied by \textdollar 100,000.
  You may use a calculator to determine any RREFs.
  \begin{center}
    \begin{tabular}{|c|r|r|r|r|r|r|}
      \hline
      Co. & Part 1 & Part 2 & Part 3 & Part 4 & Part 5 & Total Spent\\
      \hline
      A & 15 & 3 & 24 & 46 & 182 & 1013 \\
      B & 5 & 3 & 14 & 25 &100 & 552 \\
      C & 15 & 3 & 24 & 46 &188 & 1038 \\
      D & 5 & 0 &  5 & 10 & 40 & 225 \\
      E & 15 & 3 & 26 & 47 &190 & 1056 \\
      \hline
    \end{tabular}
  \end{center}
\end{enumerate}

\section{Challenge Problems}

\begin{enumerate}[resume]
\item
  Consider the following pair of linear systems.
  \begin{align*}
    ax + by &= c \\
    dx + ey &= f
  \end{align*}
  \begin{align*}
    ax + by - cz = 0 \\
    dx + ey - fz = 0
  \end{align*}
  \begin{enumerate}
  \item Demonstrate that if the first system has a solution, then so does the second one.
  \item
    Give \textbf{nonzero} values to $a$ through $f$ such that the second system has a solution, but the first does not.
    Present your solution as an augmented matrix for the first system, i.e., of the form
    \begin{displaymath}
      \begin{bmatrix}
        a & b & c \\
        d & e & f
      \end{bmatrix}
    \end{displaymath}
  \end{enumerate}
\item
  Consider the following pair of linear systems.
  \begin{align*}
    x_1 - 2x_2 &= 3 \\
    4x_1 + x_2 &= 21
  \end{align*}
  \begin{align*}
    10x_3 + 2x_4 &= x_1 \\
    (-8)x_3 + 9x_4 &= x_2
  \end{align*}
  \begin{enumerate}
  \item Solve the first system of linear equations (in $x_1$ and $x_2$) and write down the augmented matrix of the second system with the solutions of $x_1$ and $x_2$ substituted in.
  \item
    Determine the augmented matrix of a \textit{single} system of linear equations \textit{with all four equations} in the variables $x_1$, $x_2$, $x_3$, $x_4$.
    Describe the relationship between this matrix and the one in the previous part.
\end{enumerate}
\item
  Determine what must hold of $a$, $b$, $c$, $d$ so that the following linear system has exactly one solution.
  \begin{align*}
    ax + by &= 0 \\
    cx + dy &= 0
  \end{align*}
\item
  Determine an inconsistent linear system in 3 variables such that every pair of equations is consistent.
  That is, give values for $a$ through $l$ in the system
  \begin{align*}
    ax + by + cz = d \\
    ex + fy + gz = h \\
    ix + jy + kz = l
  \end{align*}
  such that the system is inconsistent, but each pair of equations forms a consistent system.
  Present your solution to each part as the an augmented matrix, i.e., of the form
  \begin{displaymath}
    \begin{bmatrix}
      a & b & c & d \\
      e & f & g & h \\
      i & j & k & l \\
    \end{bmatrix}
  \end{displaymath}
  \begin{enumerate}
  \item Achieve this with no more than 5 nonzero values for $a$ through $l$.
  \item Achieve this with all nonzero values for $a$ through $l$.
  \end{enumerate}
\item
  Consider an arbitrary linear system in three variables:
  \begin{align*}
    ax + by + cz = d \\
    ex + fy + gz = h \\
    ix + jy + kz = l
  \end{align*}
  Show that if $(s_x, s_y, s_z)$ and $(t_x, t_y, t_z)$ are solutions to the above system, then
  \begin{displaymath}
    \left(\frac{s_x + t_x}{2}, \frac{s_y + t_y}{2}, \frac{s_z + t_z}{2} \right)
  \end{displaymath}
  is also a solution.
  Describe why $(s_x, s_y, s_z) \not = (t_x, t_y, t_z)$ implies the system has infinitely many solutions.
\item
  One way to describe a line in 3 dimensions is to use a parameter $t$.
  That is, after fixing values $a_x$, $b_x$, $a_y$, $b_y$, $a_z$, and $b_z$, a line can be described as all points of the form
  \begin{displaymath}
    (a_xt + b_x, a_yt + b_y, a_zt + b_z)
  \end{displaymath}
  for any real value of $t$.
  Give a pair of 3 dimensional linear equations (each of which represents a plane in $\mathbb R^3$) whose intersection is exactly the line whose points are defined by the above parametric form.
  \textit{Hint:} The line above can be thought of as a system in the variables $x$, $y$, $z$, and $t$, e.g., one of its equations is $x - a_xt = b_x$. Write $t$ in terms of $x$ and substitute this value for $t$ into the other equations.

\end{enumerate}

\chapter{Vector Equations}

\chapter{Matrix-Vector Equations}

\chapter{Linear Independence}

\chapter{Linear Transformations}

\chapter{Matrix Algebra}

\chapter{LU Factorization}

\chapter{Markov Chains}

\chapter{Vector Spaces}

\chapter{Eigenvectors}

\chapter{Analytic Geometry}

\chapter{Least Squares}

\chapter{Singular Value Decomposition}

\end{document}
