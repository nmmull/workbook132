\section*{Basic Exercises}

\faCalculator \ For each pair of vectors $\vv_1$ and $\vv_2$, determine:
\begin{itemize}
\item the lengths of $\vv_1$ and $\vv_2$;
\item the angle between $\vv_1$ and $\vv_2$;
\item the distance between $\vv_1$ and $\vv_2$;
\item the unit length normalizations of $\vv_1$ and $\vv_2$.
\end{itemize}
You must simplify the expression as much as you can before giving the approximate result to a couple decimal places.
\begin{enumerate}
\item
  \begin{displaymath}
    \vv_1 = \vTwo 7 3
    \qquad
    \vv_2 = \vTwo {-1}  5
  \end{displaymath}
\item
  \begin{displaymath}
    \vv_1 = \vThree {-2} 4 {-5}
    \qquad
    \vv_2 = \vThree 2 2 {-2}
  \end{displaymath}
\item
  \begin{displaymath}
    \vv = \vFive 1 {-1} 5 7 4
    \qquad
    \vu = \vFive 3 1 3 {-1} 0
  \end{displaymath}
  Additionally, without doing any further calculations determine
  approximately the angle between $\vv$ and $\vv - \vu$.  Justify your
  answer.
%% \item
%%   \begin{displaymath}
%%     \vv_1 = \vFive 1 {-1} 5 7 4
%%     \qquad
%%     \vv_2 = \vFive 3 0 3 {-5} {-7}
%%   \end{displaymath}
\end{enumerate}
Determine if each of the following set of vectors is an orthogonal set.
\begin{enumerate}[resume]
\item
  \begin{displaymath}
    \left\{
    \vThree 9 {-18} {-18},
    \vThree {-2} {-14} {13}
    \right\}
  \end{displaymath}
\item
  \begin{displaymath}
    \left\{
    \vFour 0 4 4 0,
    \vFour {-2} 2 0 0,
    \vFour 2 1 {-1} {-6}
    \right\}
  \end{displaymath}
\end{enumerate}
Express the vector $\vv$ in terms of the given orthonormal basis
$\cB$.
\begin{enumerate}[resume]
\item
  \begin{displaymath}
    \vv = \vTwo{-8}{9}
    \qquad
    \cB =
    \left\{
    \vTwo{-\frac{\sqrt{2}}{2}}{\frac{\sqrt{2}}{2}},
    \vTwo{\frac{\sqrt{2}}{2}}{\frac{\sqrt{2}}{2}}
    \right\}
  \end{displaymath}
\item
  \begin{displaymath}
    \vv = \vThree 3 9 {-3}
    \qquad
    \cB =
    \left\{
    \vThree {\frac{\sqrt{2}}{2}} {\frac{\sqrt{2}}{2}} {0},
    \vThree {\frac{\sqrt{6}}{6}} {-\frac{\sqrt{6}}{6}} {-\frac{\sqrt{6}}{3}},
    \vThree {-\frac{\sqrt{3}}{3}} {\frac{\sqrt{3}}{3}} {-\frac{\sqrt{3}}{3}}
    \right\}
  \end{displaymath}
\end{enumerate}
Determine the projection of $\mathbf v$ onto the span of the given set of vectors.
\begin{enumerate}[resume]
\item
  \begin{displaymath}
    \vv = \vTwo 9 7
    \qquad
    U =
    \left\{
    \vTwo 7 {-9}
    \right\}
  \end{displaymath}
\item
  \begin{displaymath}
    \vv = \left[\begin{matrix}3\\3\\5\end{matrix}\right]
    \qquad
    U =
    \left\{
    \left[\begin{matrix}-2\\0\\-10\end{matrix}\right]
    \right\}
  \end{displaymath}
\item
  \begin{displaymath}
    \vv = \left[\begin{matrix}-3\\6\\2\end{matrix}\right]
    \qquad
    U =
    \left\{
    \left[\begin{matrix}- \frac{\sqrt{2}}{2}\\\frac{\sqrt{2}}{2}\\0\end{matrix}\right],
    \left[\begin{matrix}\frac{2}{3}\\\frac{2}{3}\\- \frac{1}{3}\end{matrix}\right]
    \right\}
  \end{displaymath}
\end{enumerate}
\section*{True/False}

\trueFalseHeader

\begin{enumerate}
\item
  If $\|\vu\|^2 + \|\vv\|^2 = \|\vu + \vv\|^2$, then $\vu$ and $\vv$
  are orthogonal.
\item
  For a square matrix $A$, vectors in $\col A$ are orthogonal to
  vectors in $\nul A$.
\item
  There can be a linear dependence relationship between vectors in an
  orthogonal set.
\item
  In $\R^4$, any set of vectors that has 5 members cannot be an
  orthogonal set.
\item
  Orthogonal matrices are invertible.
\item
  Orthogonal matrices have determinant 1.
\item
  For an $m \times n$ matrix with orthogonal columns, it may be the
  case that $m < n$.
\item
  The orthogonal projection of $\vy$ onto $\vv$ is the same as the
  orthogonal projection of $\vy$ onto $c \vv$ whenever $c \neq 0$.
\item
  For any vector $\vy \in \R^n$ and any subspace $W$ of $\R^n$, the
  vector $\vy - \proj_W \vy$ is orthogonal to $W$.
\end{enumerate}

\section*{More Difficult Problems}

\begin{enumerate}
\item
  Determine if the following six vectors form an orthogonal set.
  Justify your answer. \textit{Hint:} You do not need to explicitly
  check every pair for orthogonality, you can argue more generally.
  \begin{displaymath}
    \vv_1 = \begin{bmatrix} 1 \\ -2 \\ 1 \\ 0 \\ 0 \\ 0 \end{bmatrix}
    \qquad
    \vv_2 = \begin{bmatrix} 0 \\ 1 \\ 2 \\ 0 \\ 0 \\ 0 \end{bmatrix}
    \qquad
    \vv_3 = \begin{bmatrix} -5 \\ -2 \\ 1 \\ 0 \\ 0 \\ 0 \end{bmatrix}
  \end{displaymath}
  \begin{displaymath}
    \vv_4 = \begin{bmatrix} 0 \\ 0 \\ 0 \\ 3 \\ -3 \\ 0 \end{bmatrix}
    \qquad
    \vv_5 = \begin{bmatrix} 0 \\ 0 \\ 0 \\ 2 \\ 2 \\ -1 \end{bmatrix}
    \qquad
    \vv_6 = \begin{bmatrix} 0 \\ 0 \\ 0 \\ 1 \\ 1 \\ 4 \end{bmatrix}
  \end{displaymath}
\item Project $\vb$ onto $\vspan\{ \va_1, \va_2, \va_3 \}$.
  \begin{displaymath}
    \vb = \vFour 2 5 6 6
    \qquad
    \va_1 = \vFour 1 1 0 {-1}
    \qquad
    \va_2 = \vFour 1 0 1 1
    \qquad
    \va_3 = \vFour 0 {-1} 1 {-1}
  \end{displaymath}
\item Show that any $2 \times 2$ rotation matrix $R_\theta$ is an orthogonal matrix.
\item
  Show that any $3 \times 3$ rotation matrix $R_x^\theta$,
  $R_y^\theta$, $R_z^\theta$ (about the $x$-axis, $y$-axis, and $z$-axis,
  respectively) is an orthogonal matrix.
\item
  Determine the matrix that implements orthogonal projection onto
  $\vspan\{ [2 \ \ 3 \ \ 1]^T \}$.
\item Determine a matrix (in terms of $\vu_1$ and $\vu_2$) that
  implements orthogonal projection onto $\vspan \{ \vu_1, \vu_2 \}$
  where $\vu_1$ and $\vu_2$ are orthogonal.
\item
  Show that for any nonzero vector $\vv \in \R^n$, the set of vectors
  orthogonal to $\vv$ form a subspace of $\R^n$.
\item
  Determine a basis for the set of vectors orthogonal to the following vector.
  \begin{displaymath}
    \vThree 1 4 {-3}
  \end{displaymath}
\item
  Determine a basis for the set of vectors orthogonal to every vector in the solution
  set of the following linear equation.
  \begin{displaymath}
    2x_1 + 3x_2 - 4x_3 + 5x_4 = 0.
  \end{displaymath}
\item
  Consider the following pair of vectors.
  \begin{displaymath}
    \vu = \vFour 1 2 1 {-2}
    \qquad
    \vv = \vFour {-3} {-6} {-2} 2
  \end{displaymath}
  For this problem, we will be using theorem that isn't too difficult
  to show: every vector in $\nul A^T$ is orthogonal to every vector in
  $\col A$.  Use this to determine two linearly independent vector
  with integer entries that are orthogonal to every vector in $\vspan
  \{ \vu , \vv \}$.
\item
  For this problem, we will use another theorem that's more difficult
  to prove: $\rank A + \dim(\nul A^T) = m$ for any matrix $A \in \R^{m
    \times n}$.  Use this to show that $\rank A = \rank A^T$.
\item
  Consider the following vectors.\footnote{This problem goes over the
  \textit{Gram-Schmidt process}, an algorithm for converting
  a basis into an orthogonal basis.}
  \begin{displaymath}
    \vv_1 = \vFour 2 1 0 {-1}
    \qquad
    \vv_2 = \vFour 3 0 0 0
    \qquad
    \vv_3 = \vFour {-4} 9 0 {-11}
    \qquad
    \vu = \vFour 2 {-2} 0 {-5}
  \end{displaymath}
  \begin{enumerate}
  \item
    Find the component of $\vv_2$ orthogonal to $\vv_1$ (that is, find
    the vector $\vz$ such that $\vv_2 = \hat{\vv_2} + \vz$ where
    $\hat{\vv_2}$ is the orthogonal projection of $\vv_2$ onto
    $\vv_1$). We will refer to this as $\vv_2'$ below.
  \item
    Find the component of $\vv_3$ orthogonal to $\vv_1$. We will refer
    to this as $\vv_3'$ below.
  \item
    Find the component of $\vv_3'$ orthogonal to $\vv_2'$. We will
    refer to this as $\vv_3''$ below.
  \item
    Demonstrate that $\{\vv_1, \vv_2', \vv_3''\}$ is an orthogonal
    set.
  \item
    Find $[\mathbf u]_{\cB}$ where $\cB = \{\vv_1, \vv_2', \vv_3''\}$
    without any row reductions.
  \end{enumerate}
\item
  Repeat the previous problem with the following vectors.
  \begin{displaymath}
    \vv_1 = \vFour 1 0 1 1
    \qquad
    \vv_2 = \vFour 0 1 1 2
    \qquad
    \vv_3 = \vFour 1 0 2 {-2}
    \qquad
    \vu = \vFour2 {-2} 1 {-5}
  \end{displaymath}
\item
  \begin{displaymath}
    A =
    \begin{bmatrix}
      1 & 2 & -5 \\
      1 & 1 & 3 \\
      8 & 14 & -6 \\
      1 & 2 & 1
    \end{bmatrix}
    \qquad
    B =
    \begin{bmatrix}
      1 & -4 & -2 & 9 & 0 \\
      0 & 2 & 2 & -6 & 1 \\
      3 & -5 & 1 & 9 & 5 \\
      0 & 1 & 1 & -2 & 1
    \end{bmatrix}
  \end{displaymath}
  \begin{enumerate}
  \item
    \faCalculator \ Determine approximately the matrix that implements
    orthogonal projection onto $\col A$.
  \item
    \faCalculator \ Determine approximately the matrix which
    implements orthogonal projection onto $\col B$.  \textit{Hint:}
    First determine $\rank B$.
  \item
    Compare the matrices from the previous two parts.  Explain what
    this implies about the relationship between $\col A$ and $\col B$.
    Justify your answer.
  \end{enumerate}
\end{enumerate}

\section*{Challenge Problems}

\begin{enumerate}[resume]
\item
  Show that the row vectors of an orthogonal matrix form an
  orthonormal set.
\item
  Show that the product of two orthogonal matrices is orthogonal.
\item
  Show that $\vv \in \nul A^T$ if and only if $\vv$ is orthogonal to every
  vector in $\col A$ for any matrix $A \in \R^{m \times n}$ and vector
  $\vv \in \R^m$.
\item
  Show that $\rank A + \dim(\nul A^T) = m$ for any matrix $A \in \R^{m
    \times n}$.
\end{enumerate}
