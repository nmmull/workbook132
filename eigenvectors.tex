\section{Basic Exercises}

Determine if $\mathbf v$ is an eigenvector of $A$.
If it is, find its corresponding eigenvalue.
\begin{enumerate}
\item
  \begin{displaymath}
    A =
    \begin{bmatrix}
      6 & 1 \\
      -3 & 2
    \end{bmatrix}
    \qquad
    \vv = \vTwo {-1} 1
  \end{displaymath}
\item
  \begin{displaymath}
    A =
    \begin{bmatrix}
      -10 & -3 & -5 \\
      5 & -5 & -3 \\
      5 & 7 & -7
    \end{bmatrix}
    \qquad
    \mathbf v = \vThree 3 0 {-5}
\end{displaymath}
\end{enumerate}
Determine if $\lambda$ is an eigenvalue of $A$.
If it is, find a basis for the corresponding eigenspace.
\begin{enumerate}[resume]
\item
  \begin{displaymath}
    A =
    \begin{bmatrix}
      -2 & 2 \\
      -1 & -4
    \end{bmatrix}
    \qquad
    \lambda = -3
  \end{displaymath}
\item
  \begin{displaymath}
    A =
    \begin{bmatrix}
      5 & -6 & 2 \\
      1 & -2 & 2 \\
      -1 & 6 & 2
    \end{bmatrix}
    \qquad
    \lambda = 4
  \end{displaymath}
\item
  \begin{displaymath}
    A =
    \begin{bmatrix}
      4 & -42 \\
      1 & -9
    \end{bmatrix}
    \qquad
    \lambda = -3
  \end{displaymath}
\item
  \begin{displaymath}
    A =
    \begin{bmatrix}
      3 & -8 & 8 \\
      8 & -13 & 8 \\
      2 & -2 & -3
    \end{bmatrix}
    \qquad
    \lambda = -5
  \end{displaymath}
\item
  \begin{displaymath}
    A =
    \begin{bmatrix}
      3 & -8 & 8 \\
      8 & -13 & 8 \\
      2 & -2 & -3
    \end{bmatrix}
    \qquad
    \lambda = 4
  \end{displaymath}
\end{enumerate}
For the following matrices, determine all eigenvalues and bases for the corresponding eigenspaces.
\begin{enumerate}[resume]
\item
  \begin{displaymath}
    \begin{bmatrix}
      -3 & 2 \\
      -10 & 6
    \end{bmatrix}
  \end{displaymath}
\item
  \begin{displaymath}
    \begin{bmatrix}
      1 & 16 & -12 \\
      0 & -3 & -2 \\
      0 & 0 & -4
    \end{bmatrix}
  \end{displaymath}

\item
  \begin{displaymath}
    \begin{bmatrix}
      3 & 0 & 0 \\
      -1 & 3 & 0 \\
      2 & 4 & 2
    \end{bmatrix}
  \end{displaymath}
\end{enumerate}
Calculate the determinant of the following matrices.
\begin{enumerate}[resume]
\item
  \begin{displaymath}
    \begin{bmatrix}
      -1 & 1 & -1 \\
      3 & 3 & 3 \\
      1 & -3 & -2
    \end{bmatrix}
  \end{displaymath}
\item
  \begin{displaymath}
    \begin{bmatrix}
      3 & -3 & 0 \\
      0 & 3 & -1 \\
      2 & 0 & -1
    \end{bmatrix}
  \end{displaymath}
\end{enumerate}
For each of the following matrices, calculate the determinant of its
inverse.
\begin{enumerate}[resume]
\item
  \begin{align*}
    \begin{bmatrix}
      1 & 3 & 4 \\
      -5 & -4 & -3 \\
      2 & 0 & -5
    \end{bmatrix}
  \end{align*}
\end{enumerate}
For each of the following matrices, determine its characteristic
polynomial.  Use the characteristic polynomial to calculate its
determinant.
\begin{enumerate}[resume]
\item
  \begin{displaymath}
    \begin{bmatrix}
      1 & -1\\
      -1 & 3 \\
    \end{bmatrix}
  \end{displaymath}
\item
  \begin{displaymath}
    \begin{bmatrix}
      1 & 0 & 0 & 0 & 0 \\
      -1 & 5 & 0 & 0 & 0 \\
      2 & 6 & 3 & 0 & 0 \\
      10 & -15 & 3 & 4 & 0 \\
      -1 & 5 & 2 & 5 & 5
    \end{bmatrix}
  \end{displaymath}
\item
  \begin{displaymath}
    \begin{bmatrix}
      1 & 0 & 2 & 10 & 5 \\
      0 & 0 & 5 & -3 & 15 \\
      0 & 0 & 16 & 6 & -1 \\
      0 & 0 & 0 & 1 & 5\\
      0 & 0 & 0 & 7 & 4
    \end{bmatrix}
  \end{displaymath}
\item
  \begin{displaymath}
    \begin{bmatrix}
      1 & 0 & 0 \\
      1 & 2 & 5 \\
      0 & 1 & 3
    \end{bmatrix}
  \end{displaymath}
\end{enumerate}

\section{True/False}

\trueFalseHeader

\begin{enumerate}[resume]
\item
  Every eigenspace of $A$ is a null space (potentially of some other
  matrix).
\item
  Every matrix has at least one eigenvector.
\item
  Every matrix $A \in \mathbb R^{n \times n}$ has an eigenbasis, i.e.,
  a set of eigenvectors that form a basis for $\mathbb{R}^n$.
\item
  Only square matrices can have eigenvectors.
\item
  If $0$ is an eigenvalue of $A$, then $A$ is not invertible.
\item
  The determinant of a matrix does not change under elementary row
  operations.
\item
  If $\det (A^2) = 1$, then $\det (A) = 1$.
\item
  A matrix with characteristic polynomial $(x+2)^3$ has a single
  eigenspace with dimension $3$.
\item
  For any square matrices $A$ and $B$, if $A \sim B$ then $\det(A) =
  \det(B)$.
\item
  $\det(5A) = 5\det A$ for any square matrix $A$.
\item
  $\det(A^TA) \geq 0$ for any square matrix $A$.
\item
  $\det(A + B) = \det(A) + \det(B)$ for any square matrices $A$ and
  $B$.
\item
  If $A$ is invertible, then $\det(ABA^{-1}) = \det(B)$.
\end{enumerate}

\section{More Difficult Problems}

\begin{enumerate}[resume]
\item
  Let $A$ be the following matrix.
  \begin{displaymath}
    \begin{bmatrix}
      -17 & 28 & 14 \\
      -7 & 11 & 7 \\
      -7 & 14 & 4
    \end{bmatrix}
  \end{displaymath}
  \begin{enumerate}
  \item
    Determine if the following vectors are eigenvectors of $A$. For
    the ones that are, find their associated eigenvalues.
    \begin{displaymath}
      \vv_1 = \vThree 1 1 1
      \qquad
      \vv_2 = \vThree 2 2 0
      \qquad
      \vv_3 = \vThree 2 1 1
    \end{displaymath}
  \item
    Show that $-3$ is an eigenvalue of $A$ without doing any row
    operations.  \textit{Hint:} Use the invertible matrix theorem.
  \item
    Find a basis for the eigenspace of $A$ corresponding to the
    eigenvalue $-3$.
\end{enumerate}
\item Let $A$ and $B$ be square matrices such that $\det A = 3.5$ and
  $\det B = -2$.  Compute the determinant of the matrix
  $B(AB)^{-1}(AB)^TA$.
\item Consider the following reduction sequence
  \begin{align*}
    R_1 &\gets R_1 + R_2 \\
    \mathsf{swap}&(R_2, R_3) \\
    R_3 &\gets R_3 + 5R_4 \\
    R_2 &\gets -3R_2 \\
    R_5 &\gets R_5 - 10R_3 \\
    R_5 &\gets R_5 / 11 \\
    \mathsf{swap}&(R_5, R_3) \\
    \mathsf{swap}&(R_1, R_2) \\
    R_4 &\gets R_4 + R_1 \\
    R_2 &\gets 5R_2 \\
    R_1 &\gets -R_1
  \end{align*}
  Suppose that $A \in \R^{5 \times 5}$ reduces to $U$ by this sequence
  of reductions, where $U$ is in \textit{reduced} echelon form.  Given
  $\rank A = 5$, determine $\det A$.
\item
  Let $A$ be as in the previous problem. Given $\rank A = 4$, determine $\det A$.
\item
  The characteristic polynomial of the following matrix is
  $-\lambda^3+5\lambda^2-6\lambda$.
  \begin{displaymath}
    \begin{bmatrix}
      0 & 3  & -3 \\
      -2 & 6 & -4 \\
      -2 & 3 & -1
    \end{bmatrix}
  \end{displaymath}
  Find the eigenvalues and bases for the eigenspaces.
\item
  For each of the following statements, find a counterexample by giving
  explicit $2 \times 2$ matrices $A$ and $B$ which falsify the
  statement.  Justify your answer.
  \begin{enumerate}
  \item
    $\det(A + B) = \det(A) + \det(B)$ for any matrices $A$ and $B$ in
    $\R^{n\times n}$.
  \item
    For any square matrix $A$, if $\det(A - \lambda I) = (\lambda - 1)^2$,
    then $\dim(\nul (A - I)) = 2$.
  \item
    For any matrices $A$ and $B$ in $\R^{n \times n}$, if $\det(A -
    \lambda I) = \det(B - \lambda I)$ (i.e., they have the same
    characteristic polynomial) then $A$ is similar to $B$.
  \end{enumerate}
\item
  Let $A$ be a $5 \times 5$ matrix with characteristic polynomial
  $(\lambda - 2)^2 (\lambda + 1)^2 \lambda$ and $\rank(A - 2I) =
  \rank(A + I) = 3$.
  \begin{enumerate}
  \item
    Determine $\rank A$. Justify your answer.
  \item
    Determine if $A$ is diagonalizable. Justify your answer.
  \end{enumerate}
\item
  Suppose that $AP = PD$ for a square matrix $A$, diagonal matrix $D$,
  and arbitrary $m \times n$ matrix $P$. Show that the nonzero columns
  of $P$ are eigenvectors of $A$ and find their corresponding
  eigenvalues in terms of entries of $D$.
\item
  Consider the following pair of vectors.
  \begin{align*}
    \vv_1 = \vFour 1 {-2} {-1} 3
    \qquad
    \vv_2 = \vFour 3 {-6} {-2} 2
  \end{align*}
  \begin{enumerate}
  \item
    Determine the matrix equations whose solution set consists exactly
    of those vectors which are orthogonal to $\vv_1$ and $\vv_2$.
  \item
    Determine a general-form solution for the matrix equation from the
    previous part.
  \item
    Determine the dimension of the subspace of vectors orthogonal to
    both $\vv_1$ and $\vv_2$
  \end{enumerate}
\item
  Show that the angle between two vectors does not depend on their
  length. In other words, show that for any pair of vectors $\vu$ and
  $\vv$ in $\R^n$, the angle between them is equal to the angle
  between $a\vu$ and $b\vv$ for any nonzero real numbers $a$ and $b$.
\item
  Suppose the eigenvalues of a $3 \times 3$ matrix $A$ are $\lambda_1
  = 2$, $\lambda_2 = 1/2$, and $\lambda_3 = 1/4$ with corresponding
  eigenvectors:
  \begin{displaymath}
    \vv_1 = \vThree 1 {-1} 3
    \qquad
    \vv_2 = \vThree {-3} 4 9
    \qquad
    \vv_3 = \vThree = 2 {-2} 4
  \end{displaymath}
  Given a starting state $\vx_0 = [8, -9, 6]^T$, give a closed form
  expression for the state after $k$ iterations: $A^k \vx_0$. Describe
  what happens as $k \to \infty$.
\item
  Recall that counterclockwise rotation by angle $\theta$ in the plane
  can be implemented by the matrix below.
  \begin{align*}
    R_\theta = \begin{bmatrix}
      \cos \theta & -\sin \theta \\
      \sin \theta & \cos \theta
    \end{bmatrix}
  \end{align*}
  Calculate $\det (R_\theta)$.
  \textit{Hint.} The value does not depend on $\theta$.
\item
  Let $R_\theta$ be as in the previous problem.
  \begin{enumerate}
  \item
    Determine the characteristic polynomial of $R_\theta$ as a function
    of $\theta$.
  \item
    Determine the values of $\theta$ for which $R_\theta$ has eigenvalues.
    For those values, also determine bases for the corresponding eigenspaces.
  \end{enumerate}
\item
  Give two $2 \times 2$ matrices where every nonzero vector is an
  eigenvector.
\item \faCalculator \
  Let $A$ be a matrix such that
  \begin{displaymath}
    A \vThree 1 1 {-3} = \vThree {-1} {-1} 3
    \qquad
    A \vThree 0 1 1 = \vThree 0 1 1
    \qquad
    A \vThree 2 {-2} {-9} = \vThree 4 {-4} {-18}
  \end{displaymath}
  Without determining $A$, determine the vector
  \begin{displaymath}
    A^5 \vThree 3 0 {-11}
  \end{displaymath}
  You may use a computer, but you must show your work and justify your
  answer.
\item
  Let $A$ be the matrix from the previous part.
  Consider the basis
  \begin{displaymath}
    \cB =
    \left\{
    \vThree 1 1 {-3},
    \vThree 0 1 1,
    \vThree 2 {-2} {-9}
    \right\}
  \end{displaymath}
  Determine the matrix $C$ such that $C[\vv]_{\cB} = A \vv$ for all $\vv \in \R^3$.
\item
Determine if the following matrix has an eigenbasis. Justify your
answer.
\begin{displaymath}
  \begin{bmatrix}
    1 & 2 & -4 & 1 \\
    0 & 1 & 4 & 1 \\
    0 & 0 & 0 & 3 \\
    0 & 0 & 0 & 2
  \end{bmatrix}
\end{displaymath}
\end{enumerate}

\section{Challenge Problems}

\begin{enumerate}[resume]
\item
  Show that a pair of eigenvectors $\vv_1$ and $\vv_2$ with
  distinct eigenvalues $\lambda_1$ and $\lambda_2$ is linearly
  independent.
\item
  Show that the determinant formula from lecture is correct.
  \textit{Hint:} Consider applying elementary row operations via left
  multiplication by elementary row matrices, and use the fact that
  \begin{displaymath}
    \det (E_k \ldots E_2 E_1 A) = \det E_k \ldots \det E_2 \det E_1 \det A
  \end{displaymath}
\item
  Consider the following matrix.
  \begin{displaymath}
    A =
    \begin{bmatrix}
      1 & 1 \\
      1 & 0
    \end{bmatrix}
  \end{displaymath}
  \begin{enumerate}
  \item
    Verify that the following vectors form an eigenbasis of $A$.  Also
    determine the eigenvalues for each eigenvector.
    \begin{displaymath}
      \vv_1 = \vTwo {\frac{1 + \sqrt{5}}{2}} 1
      \qquad
      \vv_2 = \vTwo {\frac{1 - \sqrt{5}}{2}} 1
    \end{displaymath}
  \item
    Write the vector $\ve_1$ in terms of the eigenbasis you found.  In
    other words, determine $\alpha_1$ and $\alpha_2$ such that
    \begin{displaymath}
      \begin{bmatrix}
        1 \\ 0
      \end{bmatrix}
      =
      \alpha_1\vv_1 + \alpha_2 \vv_2
    \end{displaymath}
    \textit{Hint:} Calcuate $\vv_1 - \vv_2$.
  \item
    Write down a closed-form solution for the linear dynamical system
    determined by $A$ with initial vector $\ve_1$.
  \item \faCalculator \
    If you look at the formula given by the second component of your
    closed-form solution from the previous part, this gives a
    \textit{non-recursive} definition for Fibonacci numbers.  Write
    down this formula and use it to calculate $F_{20}$, the $20$th
    fibonacci number (where $F_0 = 0$ and $F_1 = 1$).
  \end{enumerate}
\item
  A Jordan block $J$ is a square matrix in $\R^{n \times n}$ of the following form.
  \begin{displaymath}
    \begin{bmatrix}
      \alpha & 1 & & \\
      & \alpha & \ddots & \\
      & & \ddots & 1 \\
      & & & \alpha
    \end{bmatrix}
  \end{displaymath}
  where $\alpha \in \R$.
  \begin{enumerate}
  \item
    Determine the characteristic polynomial (in terms of $n$) of $J$.
  \item
    Determine the dimension of the eigenspace of $J$ for the
    eigenvalue $\alpha$.
  \item
    Construct six matrices, each with the characteristic polynomial
    $(\lambda - 3)^3(\lambda + 2)^2$, that achieve all possible
    combinations of eigenspace dimensions for $\lambda = 3$ (1,2, or
    3) and for $\lambda = 2$ (1 or 2).
  \end{enumerate}
\end{enumerate}
