\documentclass[10pt]{report}

% palatino
\usepackage{mathpazo}

% 1 inch margins
\usepackage[margin=1in]{geometry}

% line spacing
\usepackage{setspace}
\setstretch{1.15}

% better quotes
\usepackage{csquotes}

%% standard math tools
\usepackage{amsthm, amssymb, mathtools}

%% solution box
\newcommand{\widebox}[1]{\begin{center}\noindent\fbox{\parbox{0.75\linewidth}{#1}}\end{center}}
\newcommand{\solutionbox}[1]{\widebox{\textit{Solution.} {#1}}}

%% better bullets
%% \renewcommand{\theenumi}{\Alph{enumi}}
\renewcommand{\labelitemi}{$\triangleright$}

\usepackage{hyperref}
\usepackage{cleveref}

\title{Problems and Exercises}
\author{\texttt{CAS CS 132}: \textit{Geometric Algorithms}}
\begin{document}
\maketitle
\tableofcontents

\chapter*{Preface}

TODO. write a preface.

\chapter{Linear Systems}

\section{Definitions}

\section{Basic Problems}

\begin{enumerate}
\item % id: determine_coefficient_augmented_matrix
% seed: 2009837464
Determine the coefficient matrix and the augmented matrix of the following linear system.
\begin{align*}2 x_{1} - 6 x_{2} &= -4 \\6 x_{1} + 8 x_{2} &= -7\end{align*}

\item % id: determine_coefficient_augmented_matrix
% seed: 3209557267
Determine the coefficient matrix and the augmented matrix of the following linear system.
\begin{align*}x_{1} + 8 x_{2} + 7 x_{3} &= -1 \\4 x_{1} - 9 x_{2} &= 8 \\7 x_{1} - 7 x_{2} - 3 x_{3} &= 10\end{align*}

\item % id: determine_coefficient_augmented_matrix
% seed: 2878334791
Determine the coefficient matrix and the augmented matrix of the following linear system.
\begin{align*}- 8 x_{1} + 6 x_{2} + x_{3} + 7 x_{4} + 10 x_{5} &= 8 \\- 4 x_{2} - 4 x_{3} - x_{4} + x_{5} &= -8 \\7 x_{1} + x_{2} - 2 x_{3} - 8 x_{4} + 7 x_{5} &= 9 \\- 4 x_{1} - 7 x_{2} + 5 x_{3} + 2 x_{4} - 9 x_{5} &= -4\end{align*}

\item % id: determine_unique_solution_linear_system
% seed: 3094789175
Demonstrate that the following linear system has a unique solution. Also determine the solution.
\begin{align*}3 x_{1} + 6 x_{2} &= -15 \\- x_{1} + x_{2} &= 5\end{align*}

\item % id: determine_unique_solution_linear_system
% seed: 425464091
Demonstrate that the following linear system has a unique solution. Also determine the solution.
\begin{align*}- 3 x_{1} + 6 x_{3} &= 12 \\3 x_{1} + 3 x_{2} - 12 x_{3} &= -21 \\2 x_{1} - 6 x_{2} + 10 x_{3} &= 16\end{align*}

\item % id: determine_unique_solution_linear_system
% seed: 2573908387
Demonstrate that the following linear system has a unique solution. Also determine the solution.
\begin{align*}- 3 x_{1} + 6 x_{2} &= 9 \\3 x_{1} - 8 x_{2} &= -9 \\2 x_{2} + 6 x_{3} &= 18\end{align*}

\item % id: determine_unique_solution_linear_system
% seed: 4148622710
Demonstrate that the following linear system has a unique solution. Also determine the solution.
\begin{align*}- 3 x_{1} - 6 x_{3} + 9 x_{4} &= -6 \\- x_{1} + 2 x_{2} - 3 x_{4} &= 8 \\- 2 x_{1} - 9 x_{2} - 19 x_{3} + 45 x_{4} &= -61 \\x_{1} - 3 x_{2} + 5 x_{3} - 9 x_{4} &= 5\end{align*}

\end{enumerate}

\section{True/False}

\section{More Difficult Problems}

\section{Challenge Problems}

\section*{True/False}

\begin{enumerate}
\item Elementary row operations cannot change the solution set of a
  linear system.
\item There is a linear system with exactly three solutions.
\item If $A$ is the augmented matrix of an inconsistent linear system,
  and $B$ is a matrix such that $A \sim B$ (that is, $A$ and $B$ are
  row equivalent), then $B$ is the augmented matrix of an inconsistent
  linear system.
\item If $A \sim B$ and $A \sim C$ and $B$ and $C$ are in reduced
  echelon form, then $B = C$.
\item There is a unique sequence of row operations that reduces a
  given matrix to reduced echelon form.
\item If a general form solution of a linear system has a free
  variable, then the system must have infinitely many solutions.
\item A matrix may have different pivot positions depending on the
  sequence of row operations used to attain a matrix in echelon form.
\item A linear system over 3 variables and 2 equations must be
  consistent.
\item If the coefficient matrix of a linear system as more rows than
  columns, then the system must have infinitely many solutions.
\end{enumerate}

\section*{More Difficult Problems}

\begin{enumerate}
\item
  For what values of the coefficient $h$ is the following system
  inconsistent?
  \begin{align*}
    x + 4y &= -1 \\
    3x - hy &= 7
  \end{align*}
  Is there a value of $h$ for which the above system has infinitely
  many solutions? Justify your answer.
\item
  Consider the following linear system with two unknown coefficients
  $h$ and $k$.
  \begin{align*}
    hx + 2y &= 1 \\
    3x + 9y &= k
  \end{align*}
  \begin{enumerate}
  \item Determine values of $h$ and $k$ so that the above linear
    system has no solutions.
  \item Determine values of $h$ and $k$ so that the above linear
    system has exactly one solution.
  \item Determine values of $h$ and $k$ so that the above linear
    system has infinitely many solutions.
  \end{enumerate}
\end{enumerate}

\section*{Challenge Problems}

\begin{enumerate}
\item <<problem10>>
\item Determine what must hold of $a$, $b$, $c$, $d$, $f$, and $g$ so
  that the following system is inconsistent.
  \begin{align*}
    ax + by &= f \\
    cx + dy &= g
  \end{align*}
\end{enumerate}

\chapter{Vector Equations}

\chapter{Matrix-Vector Equations}

\chapter{Linear Independence}

\chapter{Linear Transformations}

\chapter{Matrix Algebra}

\chapter{LU Factorization}

\chapter{Markov Chains}

\chapter{Vector Spaces}

\chapter{Eigenvectors}

\chapter{Analytic Geometry}

\chapter{Least Squares}

\chapter{Singular Value Decomposition}

\end{document}
