\documentclass[10pt]{report}
\usepackage{workbook132}

\DeclareMathOperator{\col}{\mathrm{Col}}
\DeclareMathOperator{\row}{\mathrm{Row}}
\DeclareMathOperator{\nul}{\mathrm{Nul}}
\DeclareMathOperator{\rank}{\mathrm{rank}}
\DeclareMathOperator{\proj}{\mathrm{proj}}
\DeclareMathOperator{\vspan}{\mathrm{span}}
\DeclareMathOperator{\argmin}{\mathrm{argmin}}
\DeclareMathOperator{\argmax}{\mathrm{argmax}}

\newcommand{\R}{\mathbb R}
\newcommand{\cB}{\mathcal B}

\newcommand{\va}{\mathbf a}
\newcommand{\vb}{\mathbf b}
\newcommand{\vc}{\mathbf c}
\newcommand{\ve}{\mathbf e}
\newcommand{\vo}{\mathbf o}
\newcommand{\vu}{\mathbf u}
\newcommand{\vv}{\mathbf v}
\newcommand{\vw}{\mathbf w}
\newcommand{\vx}{\mathbf x}
\newcommand{\vy}{\mathbf y}
\newcommand{\vz}{\mathbf z}
\newcommand{\vzero}{\mathbf 0}

\newcommand{\vTwo}[2]{
  \ensuremath{
    \begin{bmatrix}
      #1 \\ #2
    \end{bmatrix}
  }
}
\newcommand{\vThree}[3]{
  \ensuremath{
    \begin{bmatrix}
      #1 \\ #2 \\ #3
    \end{bmatrix}
  }
}
\newcommand{\vFour}[4]{
  \ensuremath{
    \begin{bmatrix}
      #1 \\ #2 \\ #3 \\ #4
    \end{bmatrix}
  }
}
\newcommand{\vFive}[5]{
  \ensuremath{
    \begin{bmatrix}
      #1 \\ #2 \\ #3 \\ #4 \\ #5
    \end{bmatrix}
  }
}
\newcommand{\vSix}[6]{
  \ensuremath{
    \begin{bmatrix}
      #1 \\ #2 \\ #3 \\ #4 \\ #5 \\ #6
    \end{bmatrix}
  }
}

\newcommand{\trueFalseHeader}{
  Determine if each statement is \textbf{true} or \textbf{false} and
  justify your answer.  In particular, if the statement is false,
  provide a counterexample if possible.
}


\title{Problems and Exercises}
\author{\texttt{CAS CS 132}: \textit{Geometric Algorithms}}
\begin{document}
\maketitle
\tableofcontents

\chapter*{Preface}

The following is a collection of linear algebraic problems and
exercises for the course \texttt{CAS CS 132}: \textit{Geometric
  Algorithms} at Boston University.  As a student of this course, you
will hopefully work through nearly every problem in this document.
The problems are organized by topic and difficulty.  With regard to
difficulty there are four kinds of problems:
\begin{itemize}
\item Basic Exercises
\item True/False
\item More Difficult Problems
\item Challenge Problems
\end{itemize}
You should think of basic exercises as problems that are most like
problems that appear on quizzes, and the true/false and difficult
problems as those that are most like problems that appear on exams
(several of these problems and exercises have appeared on quizzes and
exams in the past).  The challenge problems are for those of who want
to think a bit harder, and are most like problems that may appear as
extra credit on exams.

Problems that require a computer or a calculator are labelled with the
symbol \faCalculator.  Otherwise, the expectation is that you should
be able to solve the problem by hand.

Solutions to these problems will never be made public, except during a
given semester when a problem is given in an assignment.  Also, during
the semester a student of the course is welcome to request a partial
solution any problem via our course public forum (usually Piazza).

Even if it's not explicitly stated, \textbf{you should always show
  your work and justify your answer for every problem.}  Happy
exercising/problem solving!

% ======================================================================
% ======================================================================
% ======================================================================

\chapter{Linear Systems}

Determine the coefficient matrix and augmented matrix for each of the
following linear systems.
\begin{enumerate}
\item
  \begin{align*}
    2 x_{1} - 6 x_{2} &= -4 \\
    6 x_{1} + 8 x_{2} &= -7
  \end{align*}
\item
  \begin{align*}
    x_{1} + 8 x_{2} + 7 x_{3} &= -1 \\
    4 x_{1} - 9 x_{2} &= 8 \\
    7 x_{1} - 7 x_{2} - 3 x_{3} &= 10
  \end{align*}
\item
  \begin{align*}
    x_{1} - 2x_{2} - 2x_{3} &= 2 \\
    2x_{1} - 3x_{2} - 5x_{3} &= 2 \\
    -2x_{1} + 2x_{2} + 7x_{3} &= -1
  \end{align*}
\item
  \begin{align*}
    - 8 x_{1} + 6 x_{2} + x_{3} + 7 x_{4} + 10 x_{5} &= 8 \\
    - 4 x_{2} - 4 x_{3} - x_{4} + x_{5} &= -8 \\
    7 x_{1} + x_{2} - 2 x_{3} - 8 x_{4} + 7 x_{5} &= 9 \\
    - 4 x_{1} - 7 x_{2} + 5 x_{3} + 2 x_{4} - 9 x_{5} &= -4
  \end{align*}
\end{enumerate}
For each of the following linear systems, verify that the given point
$s$ is a solution.
\begin{enumerate}[resume]
\item
  \begin{align*}
    x_{1} - 2x_{2} + x_{3} - 2x_{4} &= -9 \\
    x_{1} - x_{2} - x_{3} - 2x_{4} &= -10 \\
    -3x_{1} + 8x_{2} - 6x_{3} + 4x_{4} &= 21 \\
    2x_{2} - 7x_{3} + 7x_{4} &= 13 \\
    \\
    s = (1, 3, 2, 3)
  \end{align*}
\end{enumerate}
Demonstrate that each of the following linear systems has a unique
solution.  Additionally, determine the solution.
\begin{enumerate}[resume]
\item
  \begin{align*}
    3 x_{1} + 6 x_{2} &= -15 \\
    - x_{1} + x_{2} &= 5
  \end{align*}
\item
  \begin{align*}
    2x + 4y &= 29 \\
    -6x - 10y &= -75
  \end{align*}
\item
  \begin{align*}
    x + 3y - 3z &= -2 \\
    2x + 7y + 7z &= -5 \\
    x - 2y - z &= 3
  \end{align*}
\item
  \begin{align*}
    3x - 4y + 3z &= -9 \\
    6x + 7y - 3z &= 0 \\
    x + 10z &= -21
\end{align*}
\item
  \begin{align*}
    - 3 x_{1} + 6 x_{3} &= 12 \\
    3 x_{1} + 3 x_{2} - 12 x_{3} &= -21 \\
    2 x_{1} - 6 x_{2} + 10 x_{3} &= 16
  \end{align*}
\item
  \begin{align*}
    - 3 x_{1} + 6 x_{2} &= 9 \\
    3 x_{1} - 8 x_{2} &= -9 \\
    2 x_{2} + 6 x_{3} &= 18
  \end{align*}
\item
  \begin{align*}
    x_{1} - 2x_{2} - 2x_{3} &= -7 \\
    -x_{1} + 3x_{2} + 2x_{3} &= 10 \\
    2x_{1} - 6x_{2} - 3x_{3} &= -18
  \end{align*}
\item
  \begin{align*}
    - 3 x_{1} - 6 x_{3} + 9 x_{4} &= -6 \\
    - x_{1} + 2 x_{2} - 3 x_{4} &= 8 \\
    - 2 x_{1} - 9 x_{2} - 19 x_{3} + 45 x_{4} &= -61 \\
    x_{1} - 3 x_{2} + 5 x_{3} - 9 x_{4} &= 5
  \end{align*}
\end{enumerate}
For each of the following matrices, apply the given row operations
from top to bottom.
\begin{enumerate}[resume]
\item
  \begin{displaymath}
    \begin{bmatrix}
      -2 & 7 & 9 \\
      7 & -9 & -9 \\
      4 & -1 & 9
    \end{bmatrix}
  \end{displaymath}
  \begin{align*}
    R_{1} &\gets -5R_{1} \\
    R_{1} &\gets R_{1} - 4R_{2} \\
    R_{2} &\gets R_{2} + 3R_{1}
  \end{align*}
\item
  \begin{displaymath}
    \begin{bmatrix}
      -4 & 1 & 3 \\
      -7 & 0 & 5 \\
      -4 & 3 & 1
    \end{bmatrix}
  \end{displaymath}
  \begin{align*}
    R_{3} &\gets R_{3} - 3R_{2} \\
    R_{3} &\gets R_{3} + 3R_{1} \\
    R_{1} &\leftrightarrow R_{3}
  \end{align*}
\item
  \begin{displaymath}
    \begin{bmatrix}
      9 & 5 & -7 & -5 & -9 \\
      5 & -7 & 1 & -2 & -9 \\
      5 & 1 & -10 & 6 & -5 \\
      5 & 7 & -5 & 2 & 1
    \end{bmatrix}
  \end{displaymath}
  \begin{align*}
    R_{4} &\gets -R_{4} \\
    R_{2} &\gets R_{2} - 2R_{3} \\
    R_{2} &\gets R_{2} - 5R_{4} \\
    R_{3} &\gets R_{3} + 3R_{4} \\
    R_{3} &\leftrightarrow R_{2}
  \end{align*}
\end{enumerate}
Determine a general form solution from each of the following
matrices. That is, determine a general form solution for a linear
system whose augmented matrix is row equivalent to the given matrix.
\begin{enumerate}[resume]
\item
  \begin{displaymath}
    \begin{bmatrix}
      1 & -3 & 0 & 0 & -5 \\
      0 & 0 & 1 & 0 & 8 \\
      0 & 0 & 0 & 1 & 5
    \end{bmatrix}
  \end{displaymath}
\item
  \begin{displaymath}
    \begin{bmatrix}
      1 & 0 & -3 & 0 & 1 \\
      0 & 1 & 1 & 1 & 1 \\
      0 & 0 & 0 & 1 & 5
    \end{bmatrix}
  \end{displaymath}
\item
  \begin{displaymath}
    \begin{bmatrix}
      1 & 0 & -6 & 0 & 3 \\
      0 & 1 & -3 & 2 & -9 \\
      0 & 0 & 0 & 1 & -6
    \end{bmatrix}
  \end{displaymath}
\item
  \begin{displaymath}
    \begin{bmatrix}
      1 & 1 & 0 & 0 & 0 & -4 & 5 & -3 \\
      0 & 0 & 1 & 0 & 0 & 1 & 3 & -4 \\
      0 & 0 & 0 & 0 & 1 & 5 & -3 & 2 \\
      0 & 0 & 0 & 0 & 0 & 0 & 0 & 0 \\
      0 & 0 & 0 & 0 & 0 & 0 & 0 & 0
    \end{bmatrix}
  \end{displaymath}
\end{enumerate}
Determine three particular solutions of the linear system underlying
each of the following matrices.  That is, determine three particular
solutions of a linear system whose augmented matrix is row equivalent
to the given matrix.
\begin{enumerate}[resume]
\item
  \begin{displaymath}
    \begin{bmatrix}
      1 & 1 & 0 & -3 & 0 & 0 & -3 & 1 \\
      0 & 0 & 1 & -1 & 0 & 3 & 2 & 3 \\
      0 & 0 & 0 & 0 & 1 & 3 & -4 & -3 \\
      0 & 0 & 0 & 0 & 0 & 0 & 0 & 0 \\
      0 & 0 & 0 & 0 & 0 & 0 & 0 & 0
    \end{bmatrix}
  \end{displaymath}
\end{enumerate}
Determine the row-reduced echelon form of each of the following
matrices.  You must include the intermediate matrices and row
operations used.
\begin{enumerate}[resume]
\item
  \begin{displaymath}
    \begin{bmatrix}
      5 & 2 & 9 \\
      7 & 3 & 12 \\
      2 & 1 & 3
    \end{bmatrix}
  \end{displaymath}
\item
  \begin{displaymath}
    \begin{bmatrix}
      0 & 1 & -6 & -2 \\
      1 & 1 & -3 & -6 \\
      -2 & -4 & 18 & 16
    \end{bmatrix}
  \end{displaymath}
\item
  \begin{displaymath}
    \begin{bmatrix}
      2 & 3 & -13 & -8 \\
      1 & 1 & -4 & -1 \\
      1 & 0 & 1 & 5
    \end{bmatrix}
  \end{displaymath}
\item
  \begin{displaymath}
    \begin{bmatrix}
      1 & -1 & -2 & 1 \\
      -1 & 2 & 4 & 0 \\
      2 & -3 & -6 & 2 \\
      -2 & 1 & 2 & -1
    \end{bmatrix}
  \end{displaymath}
\end{enumerate}
For each of the following linear systems, determine
\begin{itemize}
\item its augmented matrix;
\item the row-reduced echelon form of its augmented matrix;
\item whether it has \textbf{no solutions}, a \textbf{unique
  solution}, or \textbf{infinitely many solutions}.
\end{itemize}
\begin{enumerate}[resume]
\item \faCalculator
  \begin{align*}
    x_1 - 2x_2 + 5x_3 &= 6 \\
    -2x_1 + 6x_2 - 11x_3 + 7x_4 &= -8 \\
    5x_1 - 10 x_2 + 25x_3 + 3x_4 &= 30
  \end{align*}
\item \faCalculator
  \begin{align*}
    x_1 - 3x_2 + 4x_3 &= 0 \\
    -x_1 + 6x_2 + x_3 &= 4 \\
    23x_2 + 5x_3 &= 9
  \end{align*}
\item \faCalculator
  \begin{align*}
    x_1 + 5x_2 + 3x_3 &= -4 \\
    x_1 + 6x_2 + 7x_3 &= -13 \\
    -2x_1 - 12 x_2 - 14 x_3 &= 25
  \end{align*}
\end{enumerate}
Determine a sequence of elementary row operations from the left matrix
to the right matrix.  You must including the intermediate matrices and
row operations used.
\begin{enumerate}[resume]
\item
  \begin{displaymath}
    \begin{bmatrix}
      -4 & -7 & 4 \\
      3 & 8 & 2 \\
      -10 & 1 & -9
    \end{bmatrix}
    \sim
    \begin{bmatrix}
      3 & 8 & 2 \\
      -14 & -6 & -5 \\
      -10 & 1 & -9
    \end{bmatrix}
  \end{displaymath}
\item
  \begin{displaymath}
    \begin{bmatrix}
      1 & 2 & -1 & 5 \\
      0 & 1 & 0 & 1 \\
      0 & 0 & 0 & 5 \\
      2 & 2 & 1 & 1
    \end{bmatrix}
    \sim
    \begin{bmatrix}
      3 & 4 & 0 & 6 \\
      4 & 4 & 2 & 2 \\
      0 & -1 & 0 & 4 \\
      0 & 1 & 0 & 1
    \end{bmatrix}
  \end{displaymath}
\end{enumerate}
Determine a general form solution for each of the following linear
systems by first determining the row-reduced echelon form of its
augmented matrix.  If this system has no solutions then write
\enquote{no solution}.
\begin{enumerate}[resume]
\item
  \begin{align*}
    -x - 2y - z &= -4 \\
    x + 2y &= 3 \\
    -2x -4y + z &= -7
  \end{align*}
\item
  \begin{align*}
    x_{1} - 6x_{2} + x_{4} &= 2 \\
    2x_{1} - 12x_{2} + x_{3} - 4x_{4} &= 7 \\
    x_{1} - 6x_{2} - 2x_{3} + 13x_{4} &= -4
  \end{align*}
\item \faCalculator
  \begin{align*}
    88x_1 + 61x_2 + 8x_3 + 61 x_4 &= -68 \\
    37x_1 + 14x_2 + 72x_3 + 41 x_4 &= -460 \\
    57x_1 + 59x_2 + 69x_3 + 75x_4 &= -333 \\
    92 x_1 + 60x_2 + 28x_3 + 72x_4 &= -192
  \end{align*}
\item \faCalculator
  \begin{align*}
    x_2 + x_3 + x_4 + 3x_5 - x_6 + 2x_7 &= 34 \\
    x_3 - x_4 + 4x_5 &= -14 \\
    2x_2 - 4x_3 + 8x_4 - 18x_5 - x_6 &= 110 \\
    3x_3 - 3x_4 + 12 x_5 + 2x_7 &= -18 \\
    x_2 + 4x_3 - 2x_4 + 15x_5 - x_6 + 7x_7 &= 52
  \end{align*}
\item \faCalculator
  \begin{align*}
    8x_1 + 3x_2 - 3x_3 &= -4 \\
    -3x_1 + 8x_2 + 5x_3 &= -4 \\
    6x_1 + 7x_2 - 3x_3 &= -3 \\
    8x_1 + x_2 - x_3 &= -4
  \end{align*}
\end{enumerate}

\section*{True/False}

\trueFalseHeader

\begin{enumerate}[resume]
\item
  Elementary row operations cannot change the solution set of a linear
  system.
\item
  There is a linear system with exactly three solutions.
\item
  If $A$ is the augmented matrix of an inconsistent linear system, and
  $B$ is a matrix such that $A \sim B$ (that is, $A$ and $B$ are row
  equivalent), then $B$ is the augmented matrix of an inconsistent
  linear system.
\item
  If $A \sim B$ and $A \sim C$ and $B$ and $C$ are in reduced echelon
  form, then $B = C$.
\item
  There is a unique sequence of row operations that reduces a given
  matrix to reduced echelon form.
\item
  If a general form solution of a linear system has a free variable,
  then the system must have infinitely many solutions.
\item
  A matrix may have different pivot positions depending on the
  sequence of row operations used to attain a matrix in echelon form.
\item
  A linear system over 3 variables and 2 equations must be consistent.
\item
  If the coefficient matrix of a linear system as more rows than
  columns, then the system must have infinitely many solutions.
\item
  A \textit{consistent} linear system whose augmented matrix is a $207
  \times 209$ matrix must have infinitely many solutions.
\item
  A \textit{consistent} linear system whose coefficient matrix is
  square must have infinitely many solutions.
\item
  If the rightmost column of the augmented matrix of a linear system
  is a pivot column, then the system is inconsistent.
\end{enumerate}

\section*{More Difficult Problems}
\begin{enumerate}[resume]
\item
  Find a solution to the linear system below with the following
  properties:
  \begin{enumerate}
  \item
    the solution consists entirely of integer values;
  \item
    the values in the solution are relatively prime, i.e., it's not
    possible to divide the solution by a number to get another integer
    solution. So $(2, 4, 6)$ does not satisfy this property but $(1,
    2, 3)$ does.
  \end{enumerate}
  In addition to the solution, you must write down the RREF of the
  augmented matrix of this system, but otherwise, you don't have to
  show your work.\footnote{This is the process we would use to find a
  valid solution to the problem of balancing chemical equations.}
  \begin{align*}
    4x_2 + x_3 &= 16 \\
    9x_1 - 20x_2 - 8x_3 &= -71 \\
    3x_1 - 8x_2 - 3x_3 &= -29
  \end{align*}
\item
  Determine all the $2 \times 2$ matrices in echelon form whose
  entries are either $0$ or $1$.  Mark which ones are in reduced
  echelon form.
\item
  Determine every $3 \times 3$ matrix in reduced echelon form with at
  least two pivot positions whose entries are either $0$ or $1$.
\item
  Determine a linear system over three variables such that $(a, b, c)$
  is a solution exactly when the cubic function $f(x) = ax^3 + bx^2 +
  c$ intersects the points $(-1, 4)$, $(1, 5)$ and $(2, 10)$.  You do
  not need to solve the linear system.
\item
  Determine the slope-intercept form of the line equation which
  defines the intersection of the plane
  \begin{align*}
    2x + 3y + 3z = 6
  \end{align*}
  with the $xy$-plane.
\item
  For what values of the coefficient $h$ is the following system
  inconsistent?
  \begin{align*}
    x + 4y &= -1 \\
    3x - hy &= 7
  \end{align*}
  Is there a value of $h$ for which the above system has infinitely
  many solutions? Justify your answer.
\item
  Consider the following linear system with two unknown coefficients
  $h$ and $k$.
  \begin{align*}
    hx + 2y &= 1 \\
    3x + 9y &= k
  \end{align*}
  \begin{enumerate}
  \item
    Determine values of $h$ and $k$ so that the above linear system
    has no solutions.
  \item
    Determine values of $h$ and $k$ so that the above linear system
    has exactly one solution.
  \item
    Determine values of $h$ and $k$ so that the above linear system
    has infinitely many solutions.
  \end{enumerate}
\item
  Consider the following general form solution
  \begin{align*}
    x_1 &= 3x_2 - 2x_3 \\
    x_2 &\quad \text{is free} \\
    x_3 &\quad \text{is free} \\
    x_4 &= 2 + x_3
  \end{align*}
  \begin{enumerate}
  \item
    Determine another general form solution which describes the same
    solution set, but for which $x_3$ is basic and $x_4$ is free.
  \item
    Determine an RREF which yields the general form solution from the
    previous part.
  \end{enumerate}
\item
  Consider the following general form solution.
  \begin{align*}
    x_{1} &= -6 + 6x_{3} + 2x_{5} \\
    x_{2} &= 4 + 4x_{3} + 6x_{5} \\
    x_{3} &\text{ is free} \\
    x_{4} &= -4 + 5x_{5} \\
    x_{5} &\text{ is free}
  \end{align*}
  Determine a general form solution that describes the same solution
  set but in which $x_1$ is free.
\item
  \faCalculator \ Suppose you're investigating a claim that, out of
  five competing car companies who purchase engine parts from the same
  suppliers, one purportedly \textit{overestimated} their total
  spending by \textdollar 1,300,000.  Companies are required to report
  their total spending, and you've been able to determine how many
  units (× 10,000) of each part that each company has purchased.  You
  haven't been unable to determine how much each item costs per unit
  (it's an industry secret), but you can assume that each company pays
  the same amount per unit.  Given the following data, which company
  is falsifying their records?  Justify our answer.  The total amount
  spent by each company in the table below is multiplied by
  \textdollar 100,000.
  \begin{center}
    \begin{tabular}{|c|r|r|r|r|r|r|}
      \hline
      Co. & Part 1 & Part 2 & Part 3 & Part 4 & Part 5 & Total Spent\\
      \hline
      A & 15 & 3 & 24 & 46 & 182 & 1013 \\
      B & 5 & 3 & 14 & 25 &100 & 552 \\
      C & 15 & 3 & 24 & 46 &188 & 1038 \\
      D & 5 & 0 &  5 & 10 & 40 & 225 \\
      E & 15 & 3 & 26 & 47 &190 & 1056 \\
      \hline
    \end{tabular}
  \end{center}
\item
  Suppose you're given a system of linear equations with three
  variables and two equations, and that $(4, 1, 0)$ and $(2, 4, 1)$
  are solutions to this system.  You are also given that the two
  equations define distinct planes in $\mathbb R^3$.  Determine the
  RREF of the augmented matrix for this system.
\item
  Consider an arbitrary system of linear equations with $n$ unknowns
  and $m$ equations.  Further suppose that
  \begin{itemize}
  \item
    it has a unique solution;
  \item
    it has at least as many equations as unknowns ($m \geq n$).
  \end{itemize}
  Write down an expression in terms of $m$ and $n$ for the number of
  all-zero rows which appear in the rwo-reduced echelon form of its
  augmented matrix.  Justify your answer.
\end{enumerate}

\section*{Challenge Problems}

\begin{enumerate}[resume]
\item
  Consider the following pair of linear systems.
  \begin{align*}
    ax + by &= c \\
    dx + ey &= f
  \end{align*}
  \begin{align*}
    ax + by - cz &= 0 \\
    dx + ey - fz &= 0
  \end{align*}
  \begin{enumerate}
  \item
    Demonstrate that if the first system has a solution, then so does
    the second one.
  \item
    Give \textbf{nonzero} values to $a$ through $f$ such that the
    second system has a solution, but the first does not.  Present
    your solution as an augmented matrix for the first system, i.e.,
    of the form
    \begin{displaymath}
      \begin{bmatrix}
        a & b & c \\
        d & e & f
      \end{bmatrix}
    \end{displaymath}
  \end{enumerate}
\item
  Consider the following pair of linear systems.
  \begin{align*}
    x_1 - 2x_2 &= 3 \\
    4x_1 + x_2 &= 21
  \end{align*}
  \begin{align*}
    10x_3 + 2x_4 &= x_1 \\
    -8x_3 + 9x_4 &= x_2
  \end{align*}
  \begin{enumerate}
  \item
    Solve the first system of linear equations (in $x_1$ and $x_2$)
    and write down the augmented matrix of the second system with the
    solutions of $x_1$ and $x_2$ substituted in.
  \item
    Determine the augmented matrix of a \textit{single} system of
    linear equations \textit{with all four equations} in the variables
    $x_1$, $x_2$, $x_3$, $x_4$.  Describe the relationship between
    this matrix and the one in the previous part.
\end{enumerate}
\item
  Determine what must hold of $a$, $b$, $c$, $d$ so that the following
  linear system has exactly one solution.
  \begin{align*}
    ax + by &= 0 \\
    cx + dy &= 0
  \end{align*}
\item
  Determine an inconsistent linear system in 3 variables such that
  every pair of equations is consistent.  That is, give values for $a$
  through $l$ in the system
  \begin{align*}
    ax + by + cz &= d \\
    ex + fy + gz &= h \\
    ix + jy + kz &= l
  \end{align*}
  such that the system is inconsistent, but each pair of equations
  forms a consistent system.  Present your solution to each part as
  the an augmented matrix, i.e., of the form
  \begin{displaymath}
    \begin{bmatrix}
      a & b & c & d \\
      e & f & g & h \\
      i & j & k & l
    \end{bmatrix}
  \end{displaymath}
  \begin{enumerate}
  \item
    Achieve this with no more than 5 nonzero values for $a$ through
    $l$.
  \item
    Achieve this with all nonzero values for $a$ through $l$.
  \end{enumerate}
\item
  Consider an arbitrary linear system in three variables:
  \begin{align*}
    ax + by + cz &= d \\
    ex + fy + gz &= h \\
    ix + jy + kz &= l
  \end{align*}
  Show that if $(s_x, s_y, s_z)$ and $(t_x, t_y, t_z)$ are solutions
  to the above system, then
  \begin{displaymath}
    \left(
    \frac{s_x + t_x}{2},
    \frac{s_y + t_y}{2},
    \frac{s_z + t_z}{2}
    \right)
  \end{displaymath}
  is also a solution.  Describe why $(s_x, s_y, s_z) \not = (t_x, t_y,
  t_z)$ implies the system has infinitely many solutions.
\item
  One way to describe a line in 3 dimensions is to use a parameter
  $t$.  That is, after fixing values $a_x$, $b_x$, $a_y$, $b_y$,
  $a_z$, and $b_z$, a line can be described as all points of the form
  \begin{displaymath}
    (a_xt + b_x, a_yt + b_y, a_zt + b_z)
  \end{displaymath}
  for any real value of $t$.  Give a pair of 3 dimensional linear
  equations (each of which represents a plane in $\R^3$) whose
  intersection is exactly the line whose points are defined by the
  above parametric form.  \textit{Hint:} The line above can be thought
  of as a system in the variables $x$, $y$, $z$, and $t$, e.g., one of
  its equations is $x - a_xt = b_x$. Write $t$ in terms of $x$ and
  substitute this value for $t$ into the other equations.
\item
  There are several ways to define row equivalence.  For example,
  suppose we restrict the replacement rule to only allow operations of
  the form
  \begin{displaymath}
    R_i \gets R_i + R_j
  \end{displaymath}
  That is, we can only add one row to another, without doing any
  scaling of that row.  We call this an \textit{addition operation}.

  Demonstrate that two matrices $A$ and $B$ are row equivalent if
  there is a sequence of addition and scaling operations which
  transform $A$ to $B$.  In particular, addition and scaling
  operations can simulate replacement and exchange operations.
\end{enumerate}

% ======================================================================
% ======================================================================
% ======================================================================

\chapter{Vector Equations and Spans}

\section*{Basic Exercises}

Compute the following linear combinations of vectors.
\begin{enumerate}
\item
  \begin{displaymath}
    7 \vFour 2 {-3} {-8} 9
    + 3 \vFour 2 {-3} {-4} {-3}
    - 4 \vFour 5 {-1} 5 {-7}
    - 2 \vFour {-2} {-9} {-2} {-10}
  \end{displaymath}
\end{enumerate}
For each of the following linear systems, determine an equivalent
vector equation, i.e., one that has the same solution set as the given
linear system.
\begin{enumerate}[resume]
\item
  \begin{align*}
    8x_{1} + 6x_{2} - 9x_{3} &= -5 \\
    4x_{1} + 3x_{2} + 9x_{3} + 2x_{4} &= -1 \\
    4x_{1} + 5x_{2} &= 9 \\
    3x_{2} + 8x_{3} - 3x_{4} &= 2
  \end{align*}
\end{enumerate}
Determine a vector that is in $\vspan\{\vv_1, \vv_2\}$ but not in
$\vspan\{\vv_1\}$ or $\vspan\{\vv_2\}$, if possible.  Justify your
answer.  In particular, if it is not possible, then explain why.
\begin{enumerate}[resume]
\item
  \begin{displaymath}
    \vv_1 = \vThree {-7} {-1} 0
    \quad
    \vv_2 = \vThree {-8} 9 {-7}
  \end{displaymath}
\end{enumerate}
Determine a general form solution for each vector equation.  If the
given equation has no solutions then write \enquote{no solution}.
\begin{enumerate}[resume]
\item
  \begin{displaymath}
    x_1 \vThree 1 {-2} 0
    + x_2 \vThree {-2} 5 2
    + x_3 \vThree {-8} {21} {10}
    = \vThree {16} {-38} {-12}
  \end{displaymath}
\end{enumerate}
For each of the following collections of vectors, determine if the
vector $\vv_1$ is in the span of the remaining vectors. If it
is, determine the corresponding dependence relation, i.e., write
$\vv_1$ as a linear combination of the remaining vectors.
\begin{enumerate}[resume]
\item
  \begin{displaymath}
    \vv_1 = \begin{bmatrix} 0 \\ -3 \\ -20 \end{bmatrix}
    \qquad
    \vv_2 = \begin{bmatrix} 1 \\ 1 \\ 3 \end{bmatrix}
    \qquad
    \vv_3 = \begin{bmatrix} 1 \\ 2 \\ 8 \end{bmatrix}
  \end{displaymath}
\item
  \begin{displaymath}
    \vv_1 = \begin{bmatrix} 1 \\ 4 \\ -7 \end{bmatrix}
    \quad
    \vv_2 = \begin{bmatrix} 0 \\ 1 \\ -3 \end{bmatrix}
    \quad
    \vv_3 = \begin{bmatrix} 5 \\ 7 \\ -1 \end{bmatrix}
    \quad
    \vv_4 = \begin{bmatrix} 1 \\ 2 \\ -2 \end{bmatrix}
  \end{displaymath}
\item \faCalculator
  \begin{align*}
    \mathbf v_1
    = \begin{bmatrix} 42 \\ -23 \\ 98 \\ 11 \\ -87 \end{bmatrix}
    \qquad
    \mathbf v_2
    = \begin{bmatrix} 10 \\ -33 \\ -5 \\ -30 \\ -3 \end{bmatrix}
    \qquad
    \mathbf v_3
    = \begin{bmatrix} -47 \\ -1 \\ -2 \\ -25 \\ 24 \end{bmatrix}
  \end{align*}
  \begin{align*}
    \mathbf v_4
    = \begin{bmatrix} 31 \\ -34\\ 11 \\ 39 \\ 25 \end{bmatrix}
    \qquad
    \mathbf v_5
    = \begin{bmatrix}-14 \\ 22 \\ 12 \\ 42 \\ 3\end{bmatrix}
  \end{align*}
\item \faCalculator
  \begin{align*}
    \mathbf v_1
    = \begin{bmatrix} 87 \\ -19 \\ -24 \\ -61 \\ -79 \end{bmatrix}
    \qquad
    \mathbf v_2
    = \begin{bmatrix} 10 \\ -33 \\ -5 \\ -30 \\ -3 \end{bmatrix}
    \qquad
    \mathbf v_3
    = \begin{bmatrix} -47 \\ -1 \\ -2 \\ -25 \\ 24 \end{bmatrix}
  \end{align*}
  \begin{align*}
    \mathbf v_4
    = \begin{bmatrix} 31 \\ -34\\ 11 \\ 39 \\ 25 \end{bmatrix}
    \qquad
    \mathbf v_5
    = \begin{bmatrix}-14 \\ 22 \\ 12 \\ 42 \\ 3\end{bmatrix}
  \end{align*}
\end{enumerate}
For each of the following pairs of vectors $\mathbf v_1$ and $\mathbf
v_2$, determine a linear equation whose point set is the
$\mathrm{span}\{\mathbf v_1, \mathbf v_2\}$.  The linear equation you
determine should have relatively prime integer coefficients (i.e., it
should not be possible to divide the equation by an integer value and
get a new equation with integer coefficients).
\begin{enumerate}[resume]
\item
  \begin{align*}
    \mathbf{v}_{1} = \begin{bmatrix} -2 \\ 1 \\ 2 \end{bmatrix}
    \quad
    \mathbf{v}_{2} = \begin{bmatrix} -1 \\ 0 \\ 2 \end{bmatrix}
  \end{align*}
\item
  \begin{align*}
    \mathbf v_1 = \begin{bmatrix} 1 \\ 0 \\ -2 \end{bmatrix}
    \qquad
    \mathbf v_2 = \begin{bmatrix} 1 \\ 1 \\ 1 \end{bmatrix}
  \end{align*}
\end{enumerate}

\section*{True/False}
\trueFalseHeader
\begin{enumerate}[resume]
\item
  For any two vectors $\mathbf v_1$ and $\mathbf v_2$ in $\mathbb
  R^n$, there is a vector $\mathbf u$ such that $\mathbf v_1 + \mathbf
  v_2 + \mathbf u = \mathbf 0$.
\item
  For any vectors $\mathbf v_1$, $\mathbf v_2$, and $\mathbf v_3$ in
  $\mathbb R^n$, if $\mathbf v_1 \in \mathsf{span}\{\mathbf v_2,
  \mathbf v_3\}$, then $\mathbf v_2 \in \mathsf{span}\{\mathbf v_1,
  \mathbf v_3\}$.
\item
  The span of any two distinct nonzero vectors in $\mathbb R^3$ is a
  plane.
\item
  For any vector $\mathbf v_1$, $\mathbf v_2$, and $\mathbf v_3$ in
  $\mathbb R^n$, $\mathsf{span}\{\mathbf v_1, \mathbf v_2, \mathbf
  v_3\} = \mathsf{span}\{\mathbf v_1 + \mathbf v_3, \mathbf v_2\}$.
\end{enumerate}

\section*{More Difficult Problems}

\begin{enumerate}[resume]
\item
  Consider the vectors
  \begin{displaymath}
    \mathbf v_1 = \begin{bmatrix} 1 \\ 1 \\ 1 \\ 1 \end{bmatrix}
    \qquad
    \mathbf v_2 = \begin{bmatrix} 1 \\ 1 \\ 2 \\ 2 \end{bmatrix}
  \end{displaymath}
  Write vectors $\mathbf v_3$ and $\mathbf v_4$ such that $\mathbf v_3$ is not in $\mathsf{span}\{\mathbf v_1, \mathbf v_2\}$ and $\mathbf v_4$ is not in $\mathsf{span}\{\mathbf v_1, \mathbf v_2, \mathbf v_3\}$.
\end{enumerate}

\section*{Challenge Problems}

\begin{enumerate}[resume]
\item
  Find a linear equation in three variables whose point set is exactly
  \begin{align*}
    \left\{
    \mathbf v +
    \begin{bmatrix} 2 \\ -3 \\ 2 \end{bmatrix}
    \ \text{where} \
    \mathbf v \in
    \mathsf{span}
    \left\{
    \begin{bmatrix} 1 \\ 1 \\ -5 \end{bmatrix},
    \begin{bmatrix} 0 \\ 1 \\ -1 \end{bmatrix}
    \right\}
    \right\}
  \end{align*}
  In other words, every point in the point set of the equation can be
  expressed as the sum of $[2 \ \ (-3) \ \ 2]^T$ and a vector in the
  span of $[1 \ \ 1 \ \ (-5)]^T$ and $[0 \ \ 1 \ \ (-1)]^T$.
\item
  Consider the following linear equation.
  \begin{displaymath}
    x + y + z = 5
  \end{displaymath}
  The plane represented by this equation does not include the origin;
  such a plane is called \textit{affine}.  Determine vectors $\mathbf
  b$, $\mathbf v_1$, and $\mathbf v_2$ such that every point in the
  plane represented by the above equation (that is, every point $(a,
  b, c)$ such that $a + b + c = 5$) can be written as $\mathbf b +
  \mathbf v$ where $\mathbf v$ is in $\mathsf{span} \{\mathbf v_1,
  \mathbf v_2\}$, i.e.,
  \begin{displaymath}
    \{(x, y, z) : x + y + z = 5\} =
    \{\vb + \vv : \vv \in \vspan\{\vv_1, \vv_2\}\}.
  \end{displaymath}
\end{enumerate}

% ======================================================================
% ======================================================================
% ======================================================================

\chapter{Matrix Equations}

Compute the matrix-vector multiplication $A\mathbf v$ where $A$ and
$\mathbf v$ are given below.  If it is not possible to multiply $A$
with $\mathbf v$, then explain why.
\begin{enumerate}
\item
  \begin{align*}
    A =
    \begin{bmatrix}
      -10 & 6 & 2 & 8 \\ 1 & 3 & 4 & 5 \\ 0 & -2 & 0 & -9
    \end{bmatrix}
    \quad \mathbf v = \begin{bmatrix} 4 \\ -5 \\ 3 \\ 1 \end{bmatrix}
\end{align*}
\item
  \begin{align*}
    A =
    \begin{bmatrix}
      6 & 1 & -8 & -3 \\ 5 & 0 & -9 & -4
    \end{bmatrix}
    \quad \mathbf v = \begin{bmatrix} -6 \\ 2 \end{bmatrix}
\end{align*}
\end{enumerate}
Determine a general form solution for the matrix equation $A\mathbf x
= \mathbf b$ where $A$ and $\mathbf b$ are given below.  If this
equation has no solutions then write \textit{NO SOLUTION}.
\begin{enumerate}[resume]
\item
  \begin{align*}
    A =
    \begin{bmatrix}
      1 & -3 & 1 & -9 \\ 1 & -2 & 0 & -5 \\ -3 & 8 & -1 & 19 \\ -2 & 4
      & 0 & 10
    \end{bmatrix}
    \quad \mathbf b = \begin{bmatrix} 3 \\ 7 \\ -18
      \\ -14 \end{bmatrix}
\end{align*}
\item \faCalculator
  \begin{align*}
    A =
    \begin{bmatrix}
      1 & 1 & 3 & 1 & -7 & 1 \\ -5 & -4 & -14 & -3 & 24 & -8\\ 0 & -2
      & -2 & -3 & 21 & 3 \\ -6 & -3 & -15 & -2 & 11 & -8
    \end{bmatrix}
    \quad \mathbf b =
    \begin{bmatrix}
      7 \\ -16 \\ -20 \\ -27
    \end{bmatrix}
  \end{align*}
\end{enumerate}
For each of the following matrices, Determine if its columns have full
span, i.e., given a matrix in $\mathbb R^{m \times n}$, determine if
the columns span $\mathbb R^m$.
\begin{enumerate}[resume]
\item
  \begin{align*}
    \begin{bmatrix}
      1 & 2 \\ 2 & 5
    \end{bmatrix}
  \end{align*}
\item
  \begin{align*}
    \begin{bmatrix}
      1 & -5 & 4\\ -1 & 6 & -3\\ -2 & 13 & -7\\
    \end{bmatrix}
  \end{align*}
\item
  \begin{align*}
    \begin{bmatrix}
      1 & 2 & 0 & 4 \\ -1 & -1 & 2 & 9 \\ -1 & -2 & 1 & 1 \\ 0 & 2 & 6
      & 36
    \end{bmatrix}
  \end{align*}
\item
  \begin{align*}
    \begin{bmatrix}
      8 & 3 & 8 \\ 8 & -4 & 4 \\ -2 & 1 & -5 \\ -5 & -7 & -10 \\ -8 &
      9 & 1
    \end{bmatrix}
  \end{align*}
\end{enumerate}

\section*{True/False}

\trueFalseHeader

\begin{enumerate}[resume]
\item
  For $A \in \mathbb R^{m \times n}$ where $m > n$, it is not possible
  for the columns of $A$ to span $\mathbb R^m$.
\item
  For $A \in \mathbb R^{m \times n}$, if $A \mathbf x = \mathbf b$ is
  inconsistent for some vector $\mathbf b$, then it is not possible
  for the columns of $A$ to span $\mathbb R^m$.
\item
  For $A \in \mathbb R^{m \times n}$, if $A \mathbf x = \mathbf 0$ has
  infinitely many solutions, then the columns of $A$ span $\mathbb
  R^m$.
\item
  For $A \in \mathbb R^{m \times n}$, if $A \mathbf x = \mathbf 0$ has
  infinitely many solutions, then $\{ \mathbf v : A\mathbf v = \mathbf
  0\} = \mathbb R^n$.
\item
  For matrices $A$ and $B$ in $\mathbb R^{m \times n}$, let $[A \ B]
  \in \mathbb R ^ {m \times 2n}$ be the matrix obtained by
  horizontally stacking $A$ and $B$.  If $A\mathbf x = \mathbf b$ is
  consistent and $B\mathbf x = \mathbf b$ is consistent, then so is
  $[A \ B] \mathbf x = \mathbf b$.
\item
  For matrices $A$ and $B$ in $\mathbb R^{m \times n}$, if $[A
    \ B]\mathbf x = \mathbf b$ is consistent then $A\mathbf x =
  \mathbf b$ is consistent or $B \mathbf x = \mathbf b$ is consistent.
\item
  For $A \in \mathbb R^{m \times n}$, if $A \mathbf x = \mathbf b$ is
  inconsistent for some vector $\mathbf b$, then $A$ does not have a
  pivot position in every column.
\item
  If $\mathbf x \mapsto A \mathbf x$ is one-to-one and onto, then $A$
  is a square matrix.
\item
  If $\mathbf x \mapsto A \mathbf x$ is neither one-to-one or onto,
  and the reduced echelon form of $A$ only has $0$s and $1$s, then one
  of the columns of $A$ is $\mathbf 0$.
\item
  If the columns of $A$ are linearly independent, then the reduced
  echelon form of $A$ has only $0$s and $1$s.
\item
  If $A$ is an $m \times n$ matrix and $m < n$, then the range of
  $\mathbf x \mapsto A\mathbf x$ is $\mathbb R^m$.
\item
  If the columns of $A \in \mathbb R^{m \times n}$ span all of
  $\mathbb R^m$, then the reduced echelon form of $A$ has only $0$s
  and $1$s.
\end{enumerate}

\section*{More Difficult Problems}

\begin{enumerate}[resume]
\item
  \begin{displaymath}
    A
    \begin{bmatrix}
      1 \\ 1 \\ 1 \\ 1
    \end{bmatrix}
    =
    \begin{bmatrix}
      1 \\ 2 \\ 3 \\ 4
    \end{bmatrix}
  \end{displaymath}
  For each of the following shapes, determine a concrete matrix $A$ so
  that the above equation holds, where $\blacksquare$ represents a
  nonzero entry.  { \newcommand{\bs}{\blacksquare}
    \begin{enumerate}
    \item
      \begin{displaymath}
        \begin{bmatrix}
          \bs&0&0&0 \\ 0&\bs&0&0 \\ 0&0&\bs&0 \\ 0&0&0&\bs
        \end{bmatrix}
      \end{displaymath}
    \item
      \begin{displaymath}
        \begin{bmatrix}
          \bs&0&0&0 \\ \bs&\bs&0&0 \\ \bs&\bs&\bs&0 \\ \bs&\bs&\bs&\bs
        \end{bmatrix}
      \end{displaymath}
    \item
      \begin{displaymath}
        \begin{bmatrix}
          \bs&\bs&\bs&\bs \\ \bs&\bs&\bs&0 \\ \bs&\bs&0&0 \\ \bs&0&0&0
          \\
        \end{bmatrix}
      \end{displaymath}
    \end{enumerate}
  }
\item \faCalculator \ Any three distinct points in the plane define a
  \textit{unique} quadratic equation.  Given three points $(x_1,
  y_1)$, $(x_2, y_2)$ and $(x_3, y_3)$, the equation $y = ax^2 + bx +
  c$ that passes through these three points is given by the solution
  to the following matrix equation.
  \begin{align*}
    \begin{bmatrix}
      x_1^2 & x_1 & 1 \\ x_2^2 & x_2 & 1 \\ x_3^2 & x_3 & 1
    \end{bmatrix}
    \begin{bmatrix} a \\ b \\ c \end{bmatrix}
    =
    \begin{bmatrix} y_1 \\ y_2 \\ y_3 \end{bmatrix}
  \end{align*}
  Use it to determine the unique quadratic equation which passes
  through the points $(2, 13)$, $(3, 25)$ and $(-2, 5)$.
\item
  Determine the RREF of the following matrix in terms of $x$, $y$,
  assuming $x \not = 1$.
  \begin{align*}
    \begin{bmatrix}
      x^2 & x & 1 & y \\ 1 & 1 & 1 & 1 \\ \left(\frac{x +
        1}{2}\right)^2 & \frac{x + 1}{2} & 1 & \frac{y + 1}{2}
    \end{bmatrix}
  \end{align*}
  \textit{Hint:} Don't try to row reduce it. Think in terms of
  polynomial interpolation.
\item
  Let $A$ be a matrix with $n$ rows and 6 columns.  Each row of $A$
  contains the \textbf{unweighted} percentage scores (out of 100) of
  one student on 4 homework assignments (columns 1 through 4) a
  midterm exam (column 5) and a final exam (column 6).
  \begin{displaymath}
    \begin{matrix}
      H_1 & H_2 & H_3 & H_4 & M & F
    \end{matrix}
  \end{displaymath}
  \begin{displaymath}
    \begin{bmatrix}
      p_{11} & p_{12} & p_{13} & p_{14} & p_{15} & p_{16} \\ p_{21} &
      p_{22} & p_{23} & p_{24} & p_{25} & p_{26} \\ \vdots & \vdots &
      \vdots & \vdots & \vdots & \vdots \\ p_{n1} & p_{n2} & p_{n3} &
      p_{n4} & p_{n5} & p_{n6} \\
    \end{bmatrix}
  \end{displaymath}
  All homework assignments are worth the same amount.  Let $T$ denote
  the linear transformation implemented by this matrix.
  \begin{enumerate}
  \item
    Suppose that homework assignments account for \textbf{50} percent
    of the final grade, the midterm exam accounts for \textbf{20}
    percent and the final exam accounts for \textbf{30} percent.  Find
    a vector $\mathbf v$ such that $T(\mathbf v)$ is the vector whose
    $i^\text{th}$ entry is the final percentage grade of the
    $i^\text{th}$ student.  For example, if the $i^\text{th}$ student
    recieved $90$ percent on every homework assignment, $85$ percent
    on the midterm, and $92$ percent on the final, then the
    $i^\text{th}$ entry of the output vector should be $90 * 0.5 + 85
    * 0.2 + 92 * 0.3$.
  \item
    Find a vector $\mathbf v$ such that $T(\mathbf v)$ is the vector
    whose $i^\text{th}$ entry is the unweighted homework grade for
    student $i$. For the same example as above, the $i^\text{th}$
    entry would be $90$.
  \end{enumerate}
\end{enumerate}

% ======================================================================
% ======================================================================
% ======================================================================

\chapter{Linear Independence}

\section*{Basic Exercises}

Determine if the following collections of vectors are linearly
dependent.  If they are write a dependence relation, i.e., determine
linear combination of the given vectors which sums to $\mathbf 0$.
\begin{enumerate}
\item
  \begin{align*}
    \mathbf{v}_{1} = \begin{bmatrix} 1 \\ -1 \end{bmatrix} \quad
    \mathbf{v}_{2} = \begin{bmatrix} -1 \\ 2 \end{bmatrix}
  \end{align*}
\item
  \begin{align*}
    \mathbf v_1 =
    \begin{bmatrix}
      1 \\ -1 \\ 2
    \end{bmatrix}
    \qquad \mathbf v_2 =
    \begin{bmatrix}
      -1 \\ 4 \\ -3
    \end{bmatrix}
    \qquad \mathbf v_3 =
    \begin{bmatrix}
      -3 \\ 9 \\ -8
    \end{bmatrix}
  \end{align*}
\item
  \begin{align*}
    \mathbf{v}_{1} = \begin{bmatrix} 1 \\ -2 \\ 1 \end{bmatrix} \quad
    \mathbf{v}_{2} = \begin{bmatrix} -3 \\ 7 \\ -2 \end{bmatrix} \quad
    \mathbf{v}_{3} = \begin{bmatrix} -6 \\ 15 \\ -3 \end{bmatrix}
    \quad \mathbf{v}_{4} = \begin{bmatrix} -1 \\ 3 \\ 1 \end{bmatrix}
  \end{align*}
\end{enumerate}

\section*{True/False}
\trueFalseHeader
\begin{enumerate}[resume]
\item
  For $A \in \mathbb R^{m \times n}$, if the columns of $A$ are
  linearly dependent, then they do not span $\mathbb R^m$.
\end{enumerate}

\section*{More Difficult Problems}

\begin{enumerate}[resume]
\item
  Consider four vectors $\mathbf v_1$, $\mathbf v_2$, $\mathbf v_3$
  and $\mathbf v_4$ in $\mathbb R^4$ with the property that
  \begin{align*}
    \begin{bmatrix}
      \mathbf v_1 & \mathbf v_2 & \mathbf v_3 & \mathbf v_4
    \end{bmatrix}
    \sim
    \begin{bmatrix}
      1 & 1 & 3 & -2 \\ 0 & 4 & -4 & 12 \\ 0 & 0 & 3 & 3 \\ 0 & 0 & 0
      & 0
    \end{bmatrix}
  \end{align*}
  \begin{enumerate}
  \item
    Determine if $\mathbf v_4 \in \mathrm{span}\{\mathbf v_1, \mathbf
    v_2, \mathbf v_3\}$.  If so, express $\mathbf v_4$ as a linear
    combination of the vectors $\mathbf v_1$, $\mathbf v_2$ and
    $\mathbf v_3$.
  \item Deterime if vectors $\mathbf v_2$, $\mathbf v_3$, and $\mathbf
    v_4$ linearly independent.  If so, justify your answer.  If not,
    determine a dependence relation for these vectors.
  \end{enumerate}
\item
  Determine all values of $h$ for which the following set of vectors
  is linearly dependent.
  \begin{align*}
    \mathbf v_1 =
    \begin{bmatrix} 1 \\ 9 \\ -3 \end{bmatrix}
    \qquad \mathbf v_2 =
    \begin{bmatrix} 1 \\ 4 \\ 3 \end{bmatrix}
    \qquad \mathbf v_3 =
    \begin{bmatrix} -2 \\ h \\ -6 \end{bmatrix}
    \qquad
  \end{align*}
\item
  Determine three \textit{nonzero} vectors $\mathbf v_1$, $\mathbf
  v_2$, and $\mathbf v_3$ in $\mathbb R^3$ such that
  \begin{itemize}
  \item
    $\{\mathbf v_1, \mathbf v_2, \mathbf v_3\}$ is linearly dependent
    and
  \item
    $\mathbf v_1$ cannot be written as a linear combination of
    $\mathbf v_2$ and $\mathbf v_3$.
  \end{itemize}
\item \faCalculator \ Consider the following matrix, presented as a
  SymPy array.
  \begin{lstlisting}

  import sympy a = sympy.Matrix( [[ 13, -19, 19, 16, 5, 1, 10, 5, 15],
    [-11, -7, -10, 7, 2, -8, -10, -19, 6], [ -5, 10, -7, 2, -8, 2,
      -15, -16, -11], [ 17, -13, 9, 13, 19, 8, -3, -9, 0], [ 9, -18,
      5, 1, 4, 14, 9, 8, -4], [ 8, 14, 17, 5, -6, 7, -13, 2, 12], [
      18, 12, -7, 2, -10, 15, -12, 1, -12], [ 19, -12, 1, -16, 2, -6,
      -4, 17, 15], [-19, -6, -16, -20, -20, -3, 7, 3, 14], [ 6, 8,
      -15, 5, -5, 8, -14, 5, -19]])
  \end{lstlisting}
  \begin{enumerate}
  \item Determine if the columns of this matrix span all of $\mathbb
    R^{10}$. Justify your answer.
  \item Determine if the columns of this matrix linearly
    independent. Justify your answer.
  \end{enumerate}
\item
  Consider the following collection of vectors.
  \begin{align*}
    &\mathbf v_1 =
    \begin{bmatrix} -10 \\ -6 \\ 2 \\ 0 \\ 8 \\ -3 \\ -4 \end{bmatrix}
    \qquad \mathbf v_2 =
    \begin{bmatrix} -3 \\ 3 \\ -8 \\ 5 \\ -2 \\ 7 \\ -3 \end{bmatrix}
    \qquad \mathbf v_3 =
    \begin{bmatrix} -20 \\ -12 \\ 4 \\ 0 \\ 16 \\ -6 \\ -8 \end{bmatrix}
    \qquad \mathbf v_4 =
    \begin{bmatrix} 6 \\ 1 \\ 3 \\ -1 \\ 10 \\ 7 \\ 4 \end{bmatrix}
  \end{align*}
  \begin{align*}
    &\mathbf v_5 =
    \begin{bmatrix} -2 \\ -10 \\ 10 \\ -5 \\ -3 \\ -10 \\ -5 \end{bmatrix}
    \qquad \mathbf v_6 =
    \begin{bmatrix} -2 \\ 3 \\ -9 \\ -2 \\ -9 \\ -2 \\ 7 \end{bmatrix}
    \qquad \mathbf v_7 =
    \begin{bmatrix} -13 \\ 25 \\ -22 \\ 13 \\ 24 \\ 41 \\ 15 \end{bmatrix}
  \end{align*}
  \begin{enumerate}
  \item
    Determine the first vector (i.e., the vector with the smallest
    index) which can be written as a linear combination as of the
    vectors which precede it.  Express this vectors as a linear
    combination of the vectors which precede it.  You should write
    your final solution using the vector names of the form $\mathbf
    v_i$.
  \item
    Determine a dependence relation for the entire set of vectors with
    the following properties:
    \begin{itemize}
    \item
      the coefficients of the dependence relation are relatively prime
      integers;
    \item
      the number of nonzero coefficients is maximum, i.e., there is no
      other dependence relation with a greater number of nonzero
      coefficients.
    \end{itemize}
  \end{enumerate}
\end{enumerate}

\section*{Challenge Problems}

\begin{enumerate}[resume]
\item
  Determine all values of $h$ for which the following set of vectors
  is linearly dependent.
  \begin{align*}
    \mathbf v_1 =
    \begin{bmatrix} 1 \\ -1 \\ h \end{bmatrix}
    \qquad \mathbf v_2 =
    \begin{bmatrix} 4 \\ 0 \\ 2 \end{bmatrix}
    \qquad \mathbf v_3 =
    \begin{bmatrix} 13 \\ h \\ -1 \end{bmatrix}
    \qquad
  \end{align*}
\end{enumerate}

% ======================================================================
% ======================================================================
% ======================================================================

\chapter{Linear Transformations}
\section*{Basic Exercises}

For each of the following linear transformations, its domain, codomain, and the matrix that implements it (i.e., determine the matrix $A$ such that the given transformation is $\mathbf x \mapsto A \mathbf x$).
\begin{enumerate}
\item
  \begin{displaymath}
    \begin{bmatrix}
      x_1 \\ x_2 \\ x_3
    \end{bmatrix}
    \mapsto
    \begin{bmatrix}
      x_1 + x_2 \\
      -2x_1 - x_3 \\
      x_1 + x_3
    \end{bmatrix}
  \end{displaymath}
\item
  \begin{align*}
    \begin{bmatrix} x_{1} \\ x_{2} \\ x_{3} \\ x_{4} \\ x_{5} \end{bmatrix}
    \mapsto
    \begin{bmatrix}
      -9x_{1} + 4x_{2} + 5x_{3} - 3x_{4} + 2x_{5} \\
      3x_{1} + 8x_{2} + 5x_{3} - 5x_{4} + 7x_{5} \\
      -x_{1} + 5x_{2} - x_{3} + 4x_{4} + 8x_{5}
    \end{bmatrix}
  \end{align*}
\item
  \begin{align*}
    \begin{bmatrix}
      x_1 \\ x_2 \\ x_3 \\ x_4
    \end{bmatrix}
    \mapsto
    \begin{bmatrix}
      -2x_2 + x_3 + x_4 \\ x_1 + x_3 \\ -3x_3 - 3x_4 \\ 9x_4
    \end{bmatrix}
  \end{align*}
\end{enumerate}
Let $T$ be a linear transformation with the following input-output behavior.
\begin{align*}
  T(\mathbf v_{1}) = \begin{bmatrix} -6 \\ 3 \\ -2 \\ -10 \end{bmatrix}
  \quad
  T(\mathbf v_{2}) = \begin{bmatrix} -5 \\ 1 \\ -2 \\ 9 \end{bmatrix}
  \quad
  T(\mathbf v_{3}) = \begin{bmatrix} 8 \\ -7 \\ 6 \\ 6 \end{bmatrix}
\end{align*}
Compute the following.
\begin{enumerate}[resume]
\item
  $T(\mathbf v_1 + \mathbf v_2)$
\item
  $T(-3\mathbf v_{1} - \mathbf v_{2} - 2\mathbf v_{3})$.
\end{enumerate}
Let $T$ be a linear transformation with the following input-output behavior.
\begin{align*}
  T
  \left(
  \begin{bmatrix}
    -3 \\ 9 \\ 5 \\ 4
  \end{bmatrix}
  \right)
  =
  \begin{bmatrix}
    -4 \\ 3 \\ 4
  \end{bmatrix}
  \qquad
  T
  \left(
  \begin{bmatrix}
    -9 \\ -5 \\ 0 \\ -7
  \end{bmatrix}
  \right)
  =
  \begin{bmatrix}
    0 \\ 1 \\ 2
  \end{bmatrix}
  \qquad
  T
  \left(
  \begin{bmatrix}
    9 \\ -1 \\ -2 \\ 4
  \end{bmatrix}
  \right)
  =
  \begin{bmatrix}
    3 \\ -3 \\ 7
  \end{bmatrix}
\end{align*}
Compute the following
\begin{enumerate}[resume]
\item \faCalculator \
  \begin{align*}
    T
    \left(
    \begin{bmatrix}
      30 \\ 0 \\ -7 \\ 22
    \end{bmatrix}
    \right)
  \end{align*}
\item \faCalculator \
  \begin{align*}
    T
    \left(
    \begin{bmatrix}
      1 \\ 0 \\ 0 \\ 0
    \end{bmatrix}
    \right)
  \end{align*}
\end{enumerate}
Determine the matrix that implements the linear transformation $T$, given its input-output behavior.
\begin{enumerate}[resume]
\item
  \begin{align*}
    T\left(\begin{bmatrix} 1 \\ -2 \end{bmatrix}\right)
    = \begin{bmatrix} -3 \\ -1 \end{bmatrix}
    \quad
    T\left(\begin{bmatrix} 2 \\ -3 \end{bmatrix}\right)
    = \begin{bmatrix} -4 \\ -1 \end{bmatrix}
  \end{align*}
\item
  \begin{align*}
    T\left(\begin{bmatrix} 1 \\ -2 \\ -3 \end{bmatrix}\right)
    = \begin{bmatrix} 1 \\ 1 \\ 1 \end{bmatrix}
    \quad
    T\left(\begin{bmatrix} 2 \\ -3 \\ -4 \end{bmatrix}\right)
    = \begin{bmatrix} 1 \\ 2 \\ 3 \end{bmatrix}
    \quad
    T\left(\begin{bmatrix} 2 \\ -4 \\ -5 \end{bmatrix}\right)
    = \begin{bmatrix} 2 \\ 1 \\ 3 \end{bmatrix}
  \end{align*}
\item
  \begin{align*}
    T\left(\begin{bmatrix} 2 \\ 1 \\ 0 \end{bmatrix} \right)
    = \begin{bmatrix} 3 \\ -3 \\ 4 \end{bmatrix}
    \qquad
    T\left(\begin{bmatrix} 1 \\ 1 \\ 0 \end{bmatrix} \right)
    = \begin{bmatrix} 8 \\ 0 \\ 2 \end{bmatrix}
    \qquad
    T\left(\begin{bmatrix} 2 \\ 1 \\ 2 \end{bmatrix} \right)
    = \begin{bmatrix} -1 \\ -1 \\ -1 \end{bmatrix}
  \end{align*}
\item
  \begin{align*}
    T \left(\begin{bmatrix} -7 \\ -3 \end{bmatrix} \right)
    = \begin{bmatrix} -4 \\ 3 \\ 4 \\ 0 \\ 1 \end{bmatrix}
    \qquad
    T \left(\begin{bmatrix} 9 \\ 4 \end{bmatrix} \right)
    = \begin{bmatrix} 13 \\ 0 \\ 0 \\ 8 \\ 3 \end{bmatrix}
  \end{align*}
\end{enumerate}
For each matrix, determine if its transformation is (a) one-to-one but not onto, (b) onto but not one-to-one, (c) both, or (d) neither. Justify your answer.
\begin{enumerate}[resume]
\item
  \begin{align*}
    \begin{bmatrix}
      1 & 2 & 1 & 3 \\
      1 & 3 & 1 & 3 \\
      1 & 5 & 1 & 4
    \end{bmatrix}
  \end{align*}
\item
  \begin{align*}
    \begin{bmatrix}
      1 & 1 & 2 \\
      2 & 3 & 5 \\
      -1 & -3 & -3
    \end{bmatrix}
  \end{align*}
\item
  \begin{align*}
    \begin{bmatrix}
      5 & 0 & 5 \\
      0 & 0 & 0 \\
      4 & 0 & 7
    \end{bmatrix}
  \end{align*}
\item
  \begin{align*}
    \begin{bmatrix}
      1 & 2 \\
      0 & 1 \\
      -2 & -2 \\
      1 & 3 \\
      1 & 4
    \end{bmatrix}
  \end{align*}
\item
  \begin{align*}
    \begin{bmatrix}
      1 \\ 1 \\ 1 \\ 1 \\ 1 \\ 1
    \end{bmatrix}
  \end{align*}
\item
  \begin{align*}
    \begin{bmatrix}
      2 & 0 & 0 \\
      0 & 1 & 3 \\
      0 & 0 & 3 \\
      0 & 0 & 3
    \end{bmatrix}
  \end{align*}
\item
  \begin{align*}
    \begin{bmatrix}
      1 & 2 & -1 & 6 & 9 \\
      2 & 5 & -5 & 13 & 22 \\
      -3 & -4 & -3 & -16 & -19
    \end{bmatrix}
  \end{align*}
\item
  \begin{align*}
    \begin{bmatrix}
      2 & 1 & 3 & 0 \\
      1 & 4 & 3 & 0 \\
      0 & 0 & 1 & 0
    \end{bmatrix}
  \end{align*}
\item
  \begin{align*}
    \begin{bmatrix}
      1 & -2 & 3 & -4 \\
      0 & 5 & -6 &  7 \\
      0 & 0 & -8 & 9 \\
      0 & 0 & 0 & -10
    \end{bmatrix}
  \end{align*}
\item
  \begin{align*}
    \begin{bmatrix}
      12 & 45 & -3 & 20 & 1 \\
      0 & 24 & 121 & 0 & -47 \\
      0 & 0 & 252 & 44 & 46 \\
      0 & 0 & 0 & 21 & -44 \\
      0 & 0 & 0 & 0 & 11
    \end{bmatrix}
  \end{align*}
\item
  \begin{align*}
    \begin{bmatrix}
      1 & 2 & 3 & 4 \\
      -13 & 6 & -39 & 4 \\
      2 & 9 & 6 & 7 \\
      5 & 4  & 15 & 16 \\
      0 & 1 & 0 & 65 \\
      1 & 1 & 1 & 1 \\
      -12 & 3 & -36 & 33 \\
      0 & 0 & 0 & 0
    \end{bmatrix}
  \end{align*}
\item
\end{enumerate}
Draw the image of the unit square under the following linear transformations.
\begin{enumerate}[resume]
\item
  \begin{align*}
    \begin{bmatrix} x_1 \\ x_2 \end{bmatrix}
    \mapsto
    \begin{bmatrix}
      -2 & -2 \\
      -1 & 0
    \end{bmatrix}
    \begin{bmatrix} x_1 \\ x_2 \end{bmatrix}
  \end{align*}
\end{enumerate}

\section*{True/False}

\trueFalseHeader

\begin{enumerate}[resume]
\item
  If $\mathbf x \mapsto A \mathbf x$ is one-to-one and onto, then $A$ is a square matrix.
\item
  If $\mathbf x \mapsto A \mathbf x$ is neither one-to-one or onto, and the reduced echelon form of $A$ only has $0$s and $1$s, then one of the columns of $A$ is $\mathbf 0$.
\item
  For any matrix $A \in \mathbb R^{m \times n}$ where $m > n$, it is not possible for the transformation $\mathbf x \mapsto A \mathbf x$ to be one-to-one.
\end{enumerate}


\begin{enumerate}[resume]
\item
  If $T : \mathbb R^m \to \mathbb R^n$ is a linear transformation and $T(\mathbf v) = A \mathbf v$ for all vectors $\mathbf v$ in the domain of $T$, then $A$ is an $m \times n$ matrix.
\item
  The matrix that implements a linear transformation is unique.
\end{enumerate}

\section*{More Difficult Problems}

\begin{enumerate}[resume]
\item
  Determine the matrix which implements the linear tranformation $T : \mathbb R^3 \to \mathbb R^3$ that reflects vectors across the $x_1x_2$-plane.
\item
  Determine the matrix which implements the linear transformation $T : \mathbb R^2 \to \mathbb R^4$ that repeats the input vector, e.g.,
  \begin{align*}
    T\left(\begin{bmatrix} 1 \\ 2 \end{bmatrix} \right)
    = \begin{bmatrix} 1 \\ 2 \\ 1 \\ 2 \end{bmatrix}
  \end{align*}
\item
  Determine the matrix which implements the linear transformation $T : \mathbb R^2 \to \mathbb R^2$ that reflects vectors across the line $y = x$.
\item
  Determine the matrix which implements the linear transformation $T : \mathbb R^2 \to \mathbb R^2$ that reflects vectors across the line $y = \tan(\frac{3\pi}{8})x$.
\item
  Determine the matrix which implements the transformation $T : \mathbb R^4 \to \mathbb R^4$ that swaps the first and second entries of its input, e.g.,
  \begin{align*}
    T \left(\begin{bmatrix} 1 \\ 2 \\ 3 \\ 4 \end{bmatrix} \right)
    = \begin{bmatrix} 2 \\ 1 \\ 3 \\ 4 \end{bmatrix}
  \end{align*}
\item
  Consider the following transformation.
  \begin{align*}
    \begin{bmatrix}
      x \\ y \\ z
    \end{bmatrix}
    \mapsto
    \begin{bmatrix}
      (x^3 + y^3 + z^3)^{1 / 3} \\ y + z
    \end{bmatrix}
  \end{align*}
  \begin{enumerate}
  \item
    Demonstrate that the above transformation is homogeneous.
  \item
    Determine if the above transformation is not linear.
    Justify your answer.
  \end{enumerate}
\item
  Show that the transformation $T: \mathbb R^4 \to \mathbb R^4$ given by
  \begin{displaymath}
    \begin{bmatrix}
      v_1 \\ v_2 \\ v_3 \\ v_4
    \end{bmatrix}
    \mapsto
    \begin{bmatrix}
      \min(v_1, 100) \\
      \min(v_2, 100) \\
      \min(v_3, 100) \\
      \min(v_4, 100) \\
    \end{bmatrix}
  \end{displaymath}
  is not linear.
\end{enumerate}

\section*{Challenge Problems}

\begin{enumerate}[resume]
\item
  Determine the matrices which implement the linear transformations that rotates vectors in $\mathbb R^3$ 120 degrees about
  \begin{align*}
    \mathrm{span}\left\{\begin{bmatrix}1 \\ 1 \\ 1\end{bmatrix}\right\}
  \end{align*}
  You should determine \textit{two} matrices, one for clockwise rotation and the other for counterclockwise rotation.
\item
  Determine the matrices which implement the linear transformations that reflects vectors across the plane defined by the linear equation $x + y + z = 0$, and then rotates vectors 60 degrees about
  \begin{align*}
    \mathrm{span}\left\{\begin{bmatrix}1 \\ 1 \\ 1\end{bmatrix}\right\}
  \end{align*}
  You should determine \textit{two} matrices, one for clockwise rotation and the other for counterclockwise rotation.
\item
  Determine the matrix which implements the linear transformation $T : \mathbb R^3 \to \mathbb R^3$ that reflects vectors across the plane defined by the linear equation $x + y + z = 0$.
\item
  Consider the following $\mathbb R^3$ rotation matrices.
  \begin{displaymath}
    A =
    \begin{bmatrix}
      \cos 45^\circ & -\sin 45^\circ & 0 \\
      \sin 45^\circ & \cos 45^\circ & 0 \\
      0 & 0 & 1
    \end{bmatrix}
    \qquad
    B =
    \begin{bmatrix}
      1 & 0 & 0 \\
      0 & \cos 180^\circ & -\sin 180^\circ \\
      0 & \sin 180^\circ & \cos 180^\circ
    \end{bmatrix}
  \end{displaymath}
  The matrix $A$ rotates vectors around the $x_3$-axis by 45 degrees, and $B$ rotates vectors around the $x_1$-axis by 180 degrees.
  \begin{enumerate}
  \item
    Determine $A^{-1}$.
  \item
    Calculate $ABA^{-1}$.
  \item
    Describe what the transformation implemented by $ABA^{-1}$ does geometrically.
  \end{enumerate}
\end{enumerate}


% ======================================================================
% ======================================================================
% ======================================================================

\chapter{Matrix Algebra}
\section*{Basic Exercises}

\begin{align*}
  A = \begin{bmatrix}
    6 & 1 \\
    -1 & -3 \\
    8 & 1
  \end{bmatrix} \quad B = \begin{bmatrix}
    6 & -5 \\
    -6 & -2 \\
    -3 & 4
  \end{bmatrix} \quad C = \begin{bmatrix}
    -4 & 0 \\
    -8 & 5 \\
    -4 & -1
  \end{bmatrix}
\end{align*}
Compute the following expressions.
\begin{enumerate}
\item $A + B$
\item $2A - 4B - 3C$
\item $A^TA + B^TB$
\item $AC^T + CA^T$
\end{enumerate}
Compute the following matrix multiplications, if possible.
If it is not possible, explain why.
\begin{enumerate}[resume]
\item
  \begin{align*}
    \begin{bmatrix}
      -2 & 6 \\
      -2 & -7 \\
      -3 & -2
    \end{bmatrix}
    \begin{bmatrix}
      -4 & 8 & 4 \\
      -10 & -4 & 9
    \end{bmatrix}
  \end{align*}
\item
  \begin{align*}
    \begin{bmatrix} 8 & 6 & -1 & -2 \end{bmatrix}
    \begin{bmatrix} 3 \\ 5 \\ 1 \\ 5 \end{bmatrix}
  \end{align*}
\item
  \begin{align*}
    \begin{bmatrix} -3 \\ 6 \\ 8 \\ 5 \end{bmatrix}
    \begin{bmatrix} -9 & -7 & 0 & -8 \end{bmatrix}
  \end{align*}
\item
  \begin{align*}
    \begin{bmatrix}
      7 & -10 \\
      -6 & 6 \\
      1 & 0 \\
      -4 & 2
    \end{bmatrix}
    \begin{bmatrix}
      -8 & 9 \\
      -10 & 0 \\
      -9 & 1 \\
      -4 & -3
    \end{bmatrix}
  \end{align*}
\item
  \begin{align*}
    \begin{bmatrix}
      2 & 3 & -1 \\
      7 & 4 & 3 \\
      0 & -1 & 2
    \end{bmatrix}
    \begin{bmatrix}
      1 & 4 & 1 \\
      0 & 2 & 1 \\
      0 & 0 & 3
    \end{bmatrix}
  \end{align*}
\item
  \begin{align*}
    \begin{bmatrix}
      1 & 0 & 1 & 0 & 1 \\
      0 & 1 & 0 & 1 & 0 \\
      1 & 0 & 1 & 0 & 1
    \end{bmatrix}
    \begin{bmatrix}
      1 & 2 & 3 \\
      2 & 3 & 4 \\
      3 & 4 & 5 \\
      4 & 5 & 6 \\
      5 & 6 & 7
    \end{bmatrix}
  \end{align*}
\item
  \begin{align*}
    \begin{bmatrix}
      1 & 2 & 3 & 4 & 5 & 6
    \end{bmatrix}
    \begin{bmatrix}
      1 \\ 2 \\ 3 \\ 4 \\ 5 \\ 6
    \end{bmatrix}
  \end{align*}
\item
  \begin{align*}
    \begin{bmatrix}
      1 \\ 2 \\ 3 \\ 4 \\ 5 \\ 6
    \end{bmatrix}
    \begin{bmatrix}
      1 & 2 & 3 & 4 & 5 & 6
    \end{bmatrix}
  \end{align*}
\item
  \begin{displaymath}
    \begin{bmatrix}
      1 & -1 & 4 \\
      3 & 0 & 1 \\
      0 & 1 & 1 \\
      -3 & 1 & 2
    \end{bmatrix}
    \begin{bmatrix}
      2 & 0 & 1 & 1 \\
      0 & 0 & 2 & 3 \\
      1 & 1 & 3 & 5
    \end{bmatrix}
  \end{displaymath}
\item
  \begin{displaymath}
    \begin{bmatrix}
      1 \\ 2 \\ 3 \\ -1
    \end{bmatrix}
    \begin{bmatrix}
      2 & -1 & 2 & 2
    \end{bmatrix}
  \end{displaymath}
\item
  \begin{displaymath}
    \begin{bmatrix}
      1 \\ 1 \\ 1 \\ 1 \\ 1 \\ 1
    \end{bmatrix}^T
    \begin{bmatrix}
      1 & 0 & 1 & 0 & 1 & 0 \\
      0 & 1 & 0 & 1 & 0 & 1 \\
      1 & 0 & 1 & 0 & 1 & 0 \\
      0 & 1 & 0 & 1 & 0 & 1 \\
      1 & 0 & 1 & 0 & 1 & 0 \\
      0 & 1 & 0 & 1 & 0 & 1 \\
    \end{bmatrix}
    \begin{bmatrix}
      1 \\ 1 \\ 1 \\ 1 \\ 1 \\ 1
    \end{bmatrix}
  \end{displaymath}
\item
  \begin{displaymath}
    \begin{bmatrix}
      \mathbf a_1 & \mathbf a_2 & \mathbf a_3 & \mathbf a_4 & \mathbf a_5
    \end{bmatrix}
    \begin{bmatrix}
      0 & 0 & 1 & 0 & 0 \\
      0 & 1 & 0 & 0 & 0 \\
      0 & 0 & 0 & 1 & 0 \\
      1 & 0 & 0 & 0 & 0 \\
      0 & 0 & 0 & 0 & 1
    \end{bmatrix}
  \end{displaymath}
  where $\mathbf a_1$, $\mathbf a_2$, $\mathbf a_3$, $\mathbf a_4$, $\mathbf a_5$ are vectors in $\mathbb R^n$.
\item
  \begin{align*}
    \begin{bmatrix}
      1 & 1 \\
      0 & 1
    \end{bmatrix}^{2023}
  \end{align*}
\item
  \begin{align*}
    \begin{bmatrix}
      \cos2 & -\sin2 & 0 \\
      \sin2 & \cos2 & 0 \\
      0 & 0 & 1
    \end{bmatrix}^{2023}
  \end{align*}
\end{enumerate}
Let $A$ be matrix such that $A^{-1}$ is defined as below.
Use this to determine the solution to the matrix equations of the form $A\mathbf x = \mathbf b_i$, where each $\mathbf b_i$ is defined below.
\begin{enumerate}[resume]
\item
  \begin{align*}
    A^{-1} =
    \begin{bmatrix}
      -10 & -10 & -1 \\
      -2 & -4 & -3 \\
      -8 & -6 & 1
    \end{bmatrix}
    \quad
    \mathbf b_{1} = \begin{bmatrix} -5 \\ 6 \\ -1 \end{bmatrix}
    \quad
    \mathbf b_{2} = \begin{bmatrix} -1 \\ -3 \\ -10 \end{bmatrix}
    \quad
    \mathbf b_{3} = \begin{bmatrix} -6 \\ 4 \\ -4 \end{bmatrix}
  \end{align*}
\end{enumerate}
Determine the inverse of the following matrices.
If the matrix is not invertible, then explain why.
\begin{enumerate}[resume]
\item
  \begin{align*}
    \begin{bmatrix}
      2 & -1 \\
      3 & 3
    \end{bmatrix}
  \end{align*}
\item
  \begin{align*}
    \begin{bmatrix}
      -8 & -2 \\
      -2 & -3
    \end{bmatrix}
  \end{align*}
\item
  \begin{align*}
    \begin{bmatrix}
      1 & 3 \\
      -3 & -2
    \end{bmatrix}
  \end{align*}
\item
  \begin{align*}
    \begin{bmatrix}
      0 & 0 & 1 \\
      0 & 2 & 0 \\
      3 & 0 & 0
    \end{bmatrix}
  \end{align*}
\item
  \begin{align*}
    \begin{bmatrix}
      1 & 0 & -2 \\
      -3 & 1 & 4 \\
      2 & -3 & 4
    \end{bmatrix}
  \end{align*}
\item
  \begin{align*}
    \begin{bmatrix}
      1 & -1 & 1 \\
      -3 & 4 & -4 \\
      0 & 0 & 1
    \end{bmatrix}
  \end{align*}
\item
  \begin{align*}
    \begin{bmatrix}
      1 & 1 & 0 & -1 \\
      -2 & -1 & -1 & 0 \\
      1 & 1 & 0 & 0 \\
      -2 & -2 & 0 & 1
    \end{bmatrix}
  \end{align*}
\item
  \begin{align*}
    \begin{bmatrix}
      1 & 0 & 0 \\
      2 & 1 & 0 \\
      3 & 4 & 1 \\
    \end{bmatrix}
  \end{align*}
\item
  \begin{align*}
    \begin{bmatrix}
      \cos \theta & -\sin \theta \\
      \cos \theta + \sin \theta & \cos \theta - \sin \theta
    \end{bmatrix}
  \end{align*}
  (Simplify your solution as much as possible.)
\end{enumerate}
Determine the matrix in $\mathbb R^{4 \times 4}$ that implements the following row operations, in order from top to bottom.
That is, determine a matrix $A$ such that $AB$ is the result of applying the following row operations, from top to bottom, to $B$.
\begin{enumerate}[resume]
\item
  \begin{align*}
    R_{2} &\gets R_{2} - R_{3} \\
    R_{1} &\leftrightarrow R_{4} \\
    R_{1} &\gets R_{1} + 3R_{2} \\
    R_{3} &\gets R_{3} - 5R_{4} \\
    R_{4} &\gets -2R_{4}
  \end{align*}
\end{enumerate}
Determine the inverse of the following transformation, if it exists.
Your solution should be in the form of a transformation, as given below.
\begin{enumerate}[resume]
\item
  \begin{align*}
    \begin{bmatrix}
      x_{1} \\
      x_{2} \\
      x_{3}
    \end{bmatrix} \mapsto \begin{bmatrix}
      x_{1} + x_{3} \\
      -3x_{1} + x_{2} - 4x_{3} \\
      x_{1} + 2x_{2}
    \end{bmatrix}
  \end{align*}
\end{enumerate}
Suppose that $A$ is a matrix in $\mathbb R^{3 \times 3}$ such that the following sequence of row operations (from top to bottom) transforms $A$ into the identity matrix.
Determine the inverse of $A$.
You should do this without determining $A$ first.
\begin{enumerate}[resume]
\item
  \begin{align*}
    R_{1} &\leftrightarrow R_{2} \\
    R_{1} &\gets R_{1} - R_{2} \\
    R_{3} &\gets R_{3} - 3R_{2} \\
    R_{1} &\gets R_{1} + 3R_{2} \\
    R_{3} &\gets R_{3} - 5R_{1} \\
    R_{1} &\leftrightarrow R_{2}
  \end{align*}
\end{enumerate}

\section*{True/False}

\trueFalseHeader

\begin{enumerate}[resume]
\item
  For all matrices $A$ and $B$ in $\mathbb{R}^{m \times n}$ and $C$ in $\mathbb R^{n \times m}$, we have $(A + B + C^T)^T = C + B^T + A^T$.
\item
  For all matrices $A$ and $B$ such that $AB$ is defined, we have $AB \not = BA$.
\item
  If $A\mathbf x = \mathbf b$ has a unique solution for every vector $\mathbf b$ in the span of the columns of $A$, then $A$ is invertible.
\item
  For any matrix $A$ and vector $\mathbf v$, if $\mathbf v^T A$ is defined, then $A$ is single column.
\item
  For any matrices $A$ and $B$, if $A^{-1} = B^{-1}$, then $A = B$.
\item
  For any matrices $A$ and $B$, if there is a unique matrix $X$ such that $AX = B$, then $A$ is invertible.
\item
  If $A$ and $B$ are invertible, then so is $A + B$.
\item
  If $A$ and $B$ are invertible, then so is $BA$.
\item
  If $A$ and $B$ in $\mathbb R^{n \times n}$ are invertible, then so is $\begin{bmatrix}A & B \\ B & A\end{bmatrix}$, the matrix in $\mathbb R^{2n \times 2n}$ gotten by stacking copies of $A$ and $B$.
\item
  For any matrices $A$ and $B$, if there is a unique matrix $X$ such that $AX = B$, then $A$ is invertible.
\item
  If $A$ and $B$ are symmetric and $AB = BA$ then $AB$ is symmetric.
\item
  For any matrix $A \in \mathbb R^{n \times n}$, if the columns of
  $A^3$ span all of $\mathbb R^n$, then the columns of $A$ are
  linearly independent.
\item
  For any matrix $A \in \mathbb R^{n \times n}$, the matrix $A + A^T$ is symmetric.
\item
  If $AB$ and $BA$ are symmetric, then $A$ and $B$ are symmetric.
\item
  If $A \in \mathbb R^{n \times n}$ has zeros along its diagonal, then
  $A$ is not invertible.
\item
  If $A \in \mathbb R^{n \times n}$ has a row of all zeros, then $A$
  is not invertible.
\item
  If $A \in \mathbb R^{2 \times 2}$ and $A^{-1}$ has integer entries
  then the determinant of $A$ is $1$.
\item
  For any square matrices $A$ and $B$, if $AB = I$, then $AB = BA$.
\item
  The \textbf{Hadamard product} of two matrices is defined as
  \begin{align*}
    (A \circ B)_{ij} = A_{ij} B_{ij}
  \end{align*}
  In other words, $A$ and $B$ are multiplied entry-wise.  For any
  invertible matrices $A$ and $B$, if $A \circ B$ is invertible, then
  $(A \circ B)^{-1} = A^{-1} \circ B^{-1}$.
\end{enumerate}

\section*{More Difficult Problems}

\begin{enumerate}[resume]
\item
  Let $A$ be as defined below.  Is it possible to write the
  inverse of $A$ as a power of $A$?  If so, determine the smallest positive integer $n$ such that $A^n = A^{-1}$.
  \begin{align*}
    A = \begin{bmatrix} 1 & 1 \\ -1 & 0 \end{bmatrix}
  \end{align*}
\item Determine the smallest positive integer $n$ such that $A^n = A^{-1}$.
  \begin{align*}
    A =
    \begin{bmatrix}
      \cos\frac{\pi}{9} & -\sin\frac{\pi}{9} & 0 \\
      \sin\frac{\pi}{9} & \cos\frac{\pi}{9} & 0 \\
      0 & 0 & 1
    \end{bmatrix}
  \end{align*}
\item
  Compute the following matrix expression.
  Your answer should be a single matrix with entries given in terms of $n$.
  \begin{align*}
    \begin{bmatrix}
      1 & 0 & 2 \\
      0 & 1 & 0 \\
      0 & 0 & 1
    \end{bmatrix}^{n}
    \begin{bmatrix}
      1 & 0 & 0 \\
      0 & 1 & 0 \\
      0 & -3 & 1
    \end{bmatrix}^{-n}
  \end{align*}
\item
  Suppose that $A$ and $B$ are invertible matrices such that $AB^TXA^{-1}B = I$ for some matrix $X$.
  Determine $X$ in terms of $A$ and $B$.
\item
  Let $A$, $B$, and $C$ such that $A = A^{-1}$ and $C = C^T$ and
  \begin{align*}
    A(C^{-1}(AB)^T)^TC
  \end{align*}
  is well-defined.
  Simplify this expression using the algebraic properties of matrix operations.
\item
  Determine a matrix in $\mathbb R^{2 \times 2}$ with all nonzero entries that is equal to its inverse.
  \textit{Hint.} Use the closed-form equation for the inverse of a $2 \times 2$ matrix.

\end{enumerate}

\section*{Challenge Problems}

\begin{enumerate}[resume]
\item
  Determine \textit{three} matrices in $\mathbb R^{2 \times 2}$ that satisfy the following equation.
  In particular, you must demonstrate that each matrix in your solution satisfies the equation.
  \begin{align*}
    X^2 +
    \begin{bmatrix}
      3 & 0 \\
      0 & 1
    \end{bmatrix}
    X
    +
    \begin{bmatrix}
      2 & 0 \\
      0 & 0
    \end{bmatrix}
    =
    \begin{bmatrix}
      0 & 0 \\
      0 & 0
    \end{bmatrix}
  \end{align*}
\item
  Determine the reduced echelon form of the matrix
  \begin{align*}
    \begin{bmatrix}
      a & b & 1 & 0 \\
      c & d & 0 & 1
    \end{bmatrix}
  \end{align*}
  in terms of $a$, $b$, $c$, $d$. Show your work.
\item
  Determine two invertible matrices $A$ and $B$ such that $AB^{-1} = -BA^{-1}$.
\item
  Let $A$ and $B$ be two invertible matrices in $\mathbb R^{n \times n}$ such that $AB^{-1} = -BA^{-1}$.
  Determine the inverse of the following matrix in $\mathbb R^{2n \times 2n}$.
  \begin{align*}
    \begin{bmatrix}
      A & B \\
      B & A
    \end{bmatrix}
  \end{align*}
\end{enumerate}


% ======================================================================
% ======================================================================
% ======================================================================

\chapter{LU Factorization}
\section{Basic Exercises}

Determine an LU factorization of the following matrices.
\begin{enumerate}
\item
  \begin{align*}
    \begin{bmatrix}
      1 & -2 \\
      -2 & 5
    \end{bmatrix}
  \end{align*}
\item
  \begin{align*}
    \begin{bmatrix}
      2 & 3 \\
      1 & -4
    \end{bmatrix}
  \end{align*}
\item
  \begin{align*}
    \begin{bmatrix}
      -3 & 0 & -6 & -9 \\
      0 & 2 & 4 & -4 \\
      3 & 2 & 10 & 9 \\
      -3 & 2 & -2 & -9
    \end{bmatrix}
  \end{align*}
\item
  \begin{align*}
    \begin{bmatrix}
      1 & -1 & 4 & 0 & -5 \\
      -2 & 3 & -13 & -1 & 8 \\
      2 & -4 & 18 & 3 & -2 \\
      -3 & 5 & -22 & -4 & 3
    \end{bmatrix}
  \end{align*}
\end{enumerate}


\section{True/False}

\section{More Difficult Problems}
\begin{enumerate}
\item
  Suppose that $LU$ is the LU-factorization of a matrix $A$, and let
  $B$ be a matrix such that $BU = I$.  Determine the inverse of $A$ in
  terms of $L$, $U$, and $B$.
\item
  Let $A$ be a $5 \times 2026$ matrix such that $\rank A = 4$, which
  has an LU decomposition where
  \begin{displaymath}
    L =
    \begin{bmatrix}
      1 & 0 & 0 & 0 & 0 \\
      -1 & 1 & 0 & 0 & 0\\
      0 & 4 & 1 & 0 & 0\\
      2 & 0 & 0 & 1 & 0 \\
      0 & 3 & -3 & 0 & 1
    \end{bmatrix}
  \end{displaymath}
  Determine if $\vv$ in $\col A$, where
  \begin{displaymath}
    \vv = \vFive 2 {-5} {-11} 5 {-12}
  \end{displaymath}

\end{enumerate}
\section{Challenge Problems}


% ======================================================================
% ======================================================================
% ======================================================================

\chapter{Markov Chains}
\section*{Basic Exercises}

For each of the following stochastic matrices $A$, answer the
following items:
\begin{itemize}
\item
  Draw the state diagram for the corresponding Markov chain. You
  should label each node with a positive integer for the corresponding
  column of $A$ (e.g, the node corresponding to the first column
  should be labeled \enquote{1}, to the second column labeled
  \enquote{2}, and so on).
\item
  Determine if $A$ is regular, and write down the smallest $k$ such
  that $A^k$ has strictly positive values. If it is not regular,
  justify your answer.
\item
  Determine the general form solution of the equation
  $(A-I)\,\mathbf{x} = \mathbf{0}$. You may use a computer to do this,
  but you should express your answer in fractions, not decimals.
\item
  Determine a steady state vector for $A$. If the steady state vector
  is unique, note this. You must do this by hand and show your work.
\end{itemize}
\begin{enumerate}
\item
  \begin{align*}
    \begin{bmatrix}
      0.4 & 0.8 \\ 0.6 & 0.2
    \end{bmatrix}
  \end{align*}
\item
  \begin{align*}
    \begin{bmatrix}
      0.6 & 0 & 0.1 \\
      0.4 & 0.3 & 0 \\
      0 & 0.7 & 0.9
    \end{bmatrix}
  \end{align*}
\item
  \begin{displaymath}
    \begin{bmatrix}
      1 & 1/2 & 1/3 \\
      0 & 1/2 & 1/3 \\
      0 & 0 & 1/3
    \end{bmatrix}
  \end{displaymath}
\item
  \begin{displaymath}
    \begin{bmatrix}
      0.9 & 0.9 \\
      0.1 & 0.1
    \end{bmatrix}
  \end{displaymath}
\item
  \begin{align*}
    \begin{bmatrix}
      0.2 & 0 & 0 \\
      0.4 & 1 & 0.3 \\
      0.4 & 0 & 0.7
    \end{bmatrix}
  \end{align*}
\item
  \begin{align*}
    \begin{bmatrix}
      1 & 0 & 0 \\
      0 & 0 & 1 \\
      0 & 1 & 0
    \end{bmatrix}
  \end{align*}
\item
  \begin{displaymath}
    \begin{bmatrix}
      0.2 & 0 & 0.3 \\
      0.8 & 0.5 & 0 \\
      0 & 0.5 & 0.7
    \end{bmatrix}
  \end{displaymath}
\item
  \begin{align*}
    \begin{bmatrix}
      0 & 0 & 0.3 \\
      1 & 0 & 0.7 \\
      0 & 1 & 0
    \end{bmatrix}
  \end{align*}
\item
  \begin{align*}
    \begin{bmatrix}
      0.9 & 0.6 \\
      0.1 & 0.4
    \end{bmatrix}
  \end{align*}
\item
  \begin{align*}
    \begin{bmatrix}
      1 & 0.3 & 0.4 \\
      0 & 0.7 & 0.1 \\
      0 & 0 & 0.5
    \end{bmatrix}
  \end{align*}
\item
  \begin{align*}
    \begin{bmatrix}
      0 & 0.5 & 0.5 \\
      0.5 & 0 & 0.5 \\
      0.5 & 0.5 & 0
    \end{bmatrix}
  \end{align*}
\item
  \begin{align*}
    \begin{bmatrix}
      0 & 0 & 1 \\
      1 & 0 & 0 \\
      0 & 1 & 0
    \end{bmatrix}
  \end{align*}
\item
  \begin{align*}
    \begin{bmatrix}
      0 & 0 & 0 & 0.5 \\
      1 & 0 & 0 & 0.3 \\
      0 & 1 & 0 & 0.2 \\
      0 & 0 & 1 & 0
    \end{bmatrix}
  \end{align*}
\item
  \begin{align*}
    \begin{bmatrix}
      0.4 & 0.8 & 0.5 & 0 \\
      0.6 & 0.2 & 0 & 0 \\
      0 & 0 & 0 & 0 \\
      0 & 0 & 0.5 & 1
    \end{bmatrix}
  \end{align*}
\end{enumerate}


\section*{True/False}

\trueFalseHeader

\begin{enumerate}[resume]
\item
  If $A$ is a stochastic matrix, then the all ones vector $\mathbf 1$
  is an eigenvector for $A^T$.
\item
  If a stochastic matrix has a unique steady state vector, then the
  corresponding Markov chain converges to it regardless of starting
  state.
\item
  A stochastic regular matrix $A \in \mathbb R^{n \times n}$ has a
  smallest integer $k$ such that $A^k$ has strictly positive
  entries. For such a matrix, $k \leq n$.
\item
  Every stochastic matrix has at least one steady state vector.
\item
  If $A$ and $B$ are stochastic matrices, then so is $AB$.
\item
  Every $A$ stochastic matrix is invertible.
\end{enumerate}

\section*{More Difficult Problems}

\begin{enumerate}[resume]
\item
  Suppose you've been watching the stock price of your favorite
  company and you've discovered the following trends based on the data
  you've taken:

  \begin{itemize}
  \item
    If the price is \textbf{STEADY} there's a 70\% chance it will
    remain steady the next day, but a 20\% chance it will go
    \textbf{HIGH} and a 10\% chance it will dip \textbf{LOW}.
  \item
    If the price is \textbf{HIGH} there's a 55\% chance it will remain
    high the next day, but a 40\% chance it will go back to being
    \textbf{STEADY}, and just a 5\% chance it will dip all the way
    down to \textbf{LOW}.
  \item
    If the price is \textbf{LOW}, there is a 60\% change it will
    remain low the next day, but a 30\% chance it will reset to
    \textbf{STEADY}, and a 10\% chance it will suddenly jump to
    \textbf{HIGH}.
  \end{itemize}

  Suppose that the price is currently \textbf{STEADY}.  In the long
  term, is more likely to remain \textbf{STEADY}, go up \textbf{HIGH},
  or dip down \textbf{LOW}?  Justify your answer.

\item
  Suppose that we're trying to predict the preformance of the $A_5$
  Soccer Club based on statistics that we've gathered on their recent
  games. We've found that:
  \begin{itemize}
  \item
    After a win, they have a 70\% chance of winning, a 15\% chance of
    drawing, and a 15\% chance losing their next game.
  \item
    After a draw, they have a 20\% chance of winning, a 50\% chance of
    drawing, and a 30\% chance of losing their next game.
  \item
    After a loss, they have a 20\% chance of winning, a 70\% chance of
    drawing, and a 10\% chance of losing their next game.
  \end{itemize}
  \begin{enumerate}
  \item
    If they win their first game, how should we expected them to
    perform overall?  That is, what is the expected percentage of
    wins, draws, and losses in the long term?
  \item
    If they lose their first game, what will their overall record tend
    towards?
  \end{enumerate}
\end{enumerate}

\section*{Challenge Problems}

\begin{enumerate}[resume]
\item
  Demonstrate that any invertible stochastic matrix $A$ with a
  stochastic inverse must be a permutation matrix.  That is, every
  column and every row of $A$ has a single 1, with the rest of the
  entries being 0s.  \textit{Hint:} Consider what can be inferred
  about the entries of a stochastic $A^{-1}$ from its action on $A
  \mathbf{e}_i$ for any elementary basis vector $\mathbf{e}_i$.
\end{enumerate}


% ======================================================================
% ======================================================================
% ======================================================================

\chapter{Vector Spaces}
\section{Basic Exercises}


For each of the following exercises, justify your answer.
\begin{enumerate}
\item
  Given $A \in \R^{3 \times 6}$, determine value of $n$ such that
  $\nul A$ is a subspace of $\R^n$.
\item
  Given $A \in \R^{10 \times 13}$, determine the minimum dimension of
  $\nul A$.
\item
  Given $A \in \R^{7 \times 5}$ and $\rank A = 4$, determine
  $\dim(\nul A)$.
\item
  Determine if $v$ is in $\nul A$ where
  \begin{align*}
    A =
    \begin{bmatrix}
      -1 & 0 & 1 \\
      3 & 6 & 0\\
      5 & 7 & 2
    \end{bmatrix}
    \qquad
    \vv = \vThree 2 {-1} 2
  \end{align*}
\item Determine if $\vv$ is in $\col A$, where $\vv$ and $A$ are
  defined as in the previous exercise.
\item Without performing any row operations, determine $\rank A$ where
  \begin{align*}
    A =
    \begin{bmatrix}
      2 & 1 & -8 & 3 \\
      -1 & 3 & 4 & 2 \\
      3 & 2 & -12 & 5 \\
      1 & -2 & -4 & -1
    \end{bmatrix}
  \end{align*}
\end{enumerate}
For each of the following matrices, determine a basis for its column
space and a basis for its null space.
\begin{enumerate}[resume]
\item
  \begin{align*}
    \begin{bmatrix}
      1 & -5 \\
      -2 & 10
    \end{bmatrix}
  \end{align*}
\item
  \begin{align*}
    \begin{bmatrix}
      0 & 0 \\
      0 & 0
    \end{bmatrix}
  \end{align*}
\item
  \begin{align*}
    \begin{bmatrix}
      0 & -2 & 2 \\
      -2 & 3 & -9 \\
      -1 & -2 & -1
    \end{bmatrix}
  \end{align*}
\item
  \begin{align*}
    \begin{bmatrix}
      1 & -4 & 3 & -3 \\
      -2 & 8 & -6 & 7
    \end{bmatrix}
  \end{align*}
\item
  \begin{align*}
    \begin{bmatrix}
      1 & -4 & -3 \\
      -3 & 12 & 10 \\
      -2 & 8 & 8 \\
      -1 & 4 & 2
    \end{bmatrix}
  \end{align*}
\item
  The matrix $[ \mathbf a_1 \ \ \ \mathbf a_2 \ \ \ \mathbf a_3
    \ \ \ \mathbf a_4 \ \ \ \mathbf a_5 \ \ \ \mathbf a_6
    \ \ \ \mathbf a_7 \ \ \ \mathbf a_8 ]$ which is row-equivalent to
  the following matrix.
  \begin{align*}
    \begin{bmatrix}
      1 & 0 & 3 & 0 & 0 & 5 & 0 & -7 \\
      0 & 1 & 14 & 0 & 0 & 2 & 0 & 1 \\
      0 & 0  & 0  & 1 & 0 & 3 & 0 & 7 \\
      0 & 0  & 0  & 0 & 1 & 8 & 0 & 9 \\
      0 & 0  & 0  & 0 & 0 & 0 & 1 & 1 \\
      0 & 0  & 0  & 0 & 0 & 0 & 0 & 0 \\
    \end{bmatrix}
  \end{align*}
\item \faCalculator
  \begin{lstlisting}

  Matrix(
    [[1,   2,   4,   3,  -4,  1],
     [-3, -4, -10,  -8,  13, -3],
     [ 5,  6,  16,  10, -13,  9],
     [-7, -8, -22, -12,  13, -9],
     [13, 18,  44,  32, -47, 11]]
  )
  \end{lstlisting}
\end{enumerate}
For each of the following subspaces, determine a basis.
\begin{enumerate}[resume]
\item
  \begin{align*}
    \mathrm{span}\!
    \left
    \{\begin{bmatrix} 1 \\ 2 \\ 2 \end{bmatrix},
    \begin{bmatrix} 2 \\ 5 \\ 1 \end{bmatrix},
    \begin{bmatrix} -3 \\ -5 \\ -8 \end{bmatrix}
    \right\}
  \end{align*}
\item
  \begin{align*}
    \mathrm{span}\! \left\{
    \begin{bmatrix} 1 \\ 1 \\ 0 \end{bmatrix},
    \begin{bmatrix} -1 \\ 0 \\ -3 \end{bmatrix},
    \begin{bmatrix} 2 \\ -4 \\ 18 \end{bmatrix},
    \begin{bmatrix} -2 \\ -4 \\ 6 \end{bmatrix}
    \right\}
\end{align*}
\end{enumerate}
Determine the coordinate vector $[\mathbf u]_{\mathcal B}$ where
$\vu$ and $\mathcal B$ are defined below.
\begin{enumerate}[resume]
\item
  \begin{displaymath}
    \vu = \vTwo 8 {-12}
    \qquad
    \mathcal B = \left\{ \vTwo 1 {-1}, \vTwo {-3} 4 \right\}
  \end{displaymath}
\item
  \begin{displaymath}
    \vu = \vThree 4 1 3
    \qquad
    \mathcal B = \left\{
    \vThree 1 {-1} 0
    \vThree {-1} 2 1
    \vThree {-2} 2 1
    \right\}
  \end{displaymath}
\item
  \begin{displaymath}
    \vu = \vTwo {-17} {15}
    \qquad
    \mathcal B =
    \left\{
    \vTwo 1 {-5}
    \vTwo 3 {-1}
    \right\}
  \end{displaymath}
\item \faCalculator
  \begin{displaymath}
    \vu = \vFour 5 {29} {-80} {42}
    \qquad
    \mathcal B =
    \left\{
    \vFour {-11} {-19} {17} 9,
    \vFour 9 {17} {-20} {-2},
    \vFour {-3} 1 {-18} {18}
    \right\}
  \end{displaymath}
\end{enumerate}
Determine the change-of-basis matrix for the following bases.
\begin{enumerate}[resume]
\item
  \begin{displaymath}
    \left\{
    \vTwo 1 2,
    \vTwo {-3} {-5}
    \right\}
  \end{displaymath}
\item
\begin{displaymath}
\left\{
\vThree 1 {-3} {-2},
\vThree {-3} {10} {5},
\vThree {-2} 8 3
\right\}
\end{displaymath}
\end{enumerate}
For each of the following matrices, determine a vector that is
\textit{not} in its column space.
\begin{enumerate}[resume]
\item
  \begin{displaymath}
    \begin{bmatrix}
      1 & 0 & 0 & 0 \\
      0 & 1 & 2 & 0 \\
      0 & 0 & 0 & 1 \\
      0 & 0 & 0 & 0
    \end{bmatrix}
  \end{displaymath}
\item
  \begin{displaymath}
    \begin{bmatrix}
      1 & 1 & 5 \\
      -1 & 0 & -1 \\
      1 & 2 & 9
    \end{bmatrix}
  \end{displaymath}
\end{enumerate}

\section{True/False}

\trueFalseHeader

\begin{enumerate}[resume]
\item
  There is a unique basis for any subspace.
\item
  A $7 \times 4$ matrix $A$ may have $\mathrm{dim}(\mathrm{Nul}(A)) = 5$.
\item
  A $3 \times 6$ matrix $A$ may have $\mathrm{dim}(\mathrm{Nul}(A)) = 2$.
\item
  For any matrix $A \in \mathbb R^{n \times n}$, if $A$ is invertible then $\mathrm{rank}(A) = n$.
\item
  For any matrix $A \in \mathbb R^{m \times n}$, $\mathrm{Col}(A)$ is
  the same as the set of vectors $\mathbf{b}$ such that $A \mathbf{x}
  = \mathbf{b}$ has a solution.
\item
  A basis is a spanning set that is as large as possible.
\item
  A linear transformation $T: \mathbb{R}^3 \to \mathbb{R}^3$ that maps
  $\mathbb{R}^3$ to a plane has a trivial kernel (i.e., if
  $T(\mathbf{v}) = \mathbf{0}$, then $\mathbf{v} = \mathbf{0}$).
\item
  If $\cB$ is the standard basis for $R^n$, then for any $\vx \in
  \R^n$ we have that $[\vx]_{\cB} = \vx$
\end{enumerate}

\section{More Difficult Problems}

\begin{enumerate}[resume]
\item
  For a matrix $A \in \mathbb R^{m \times n}$, consider the set of
  vectors $\{\mathbf{x} \in \mathbb{R}^n : A\mathbf{x}=
  \mathbf{e_1}\}$ (\textit{Recall:} $\mathbf{e_1}$ is the first
  standard basis vector).
  \begin{enumerate}
  \item
    Determine if this set if closed under addition. Justify your
    answer.
  \item
    Determine if this set is closed under scaling. Justify your
    answer.
  \item
    Determine if this set is a subspace of $\mathbb R^n$. Justify your
    answer.
  \end{enumerate}
\item
  \begin{align*}
    \mathbf{v_1} = \begin{bmatrix}
      3  \\
      0  \\
      -1
    \end{bmatrix}, \,
    \mathbf{v_2} = \begin{bmatrix}
      -3  \\
      2  \\
      0
    \end{bmatrix}, \,
    \mathbf{v_3} = \begin{bmatrix}
      0  \\
      -2  \\
      -1
    \end{bmatrix}
\end{align*}
List all possible subsets of the above vectors that form a basis of
the subspace $\mathrm{span}\!\left\{\mathbf{v_1}, \, \mathbf{v_2}, \,
\mathbf{v_3} \right\}$.\footnote{The process we teach for determining
a basis gives one choice, but there may be multiple choices that are
suitable.}
\item Consider the following vectors.
\begin{align*}
\mathbf{v_1} = \begin{bmatrix}
      3   \\
      -1
\end{bmatrix}, \,
\mathbf{v_2} = \begin{bmatrix}
      -3  \\
      1
\end{bmatrix}, \,
\mathbf{v_3} = \begin{bmatrix}
      -1  \\
      1
\end{bmatrix}, \,
\mathbf{v_4} = \begin{bmatrix}
      8  \\
      -4
\end{bmatrix}
\end{align*}
List all possible subsets of the above vectors that form a basis of
the subspace $\mathrm{span}\!\left\{\mathbf{v_1}, \, \mathbf{v_2}, \,
\mathbf{v_3}, \, \mathbf{v_4} \right\}$.
\item Consider the 3-dimensional vector space of all quadratic
  polynomials $Q = \{a x^2 + b x + c \, | \, a,b,c \in \mathbb{R}\}$
  and consider the linear derivative map $\frac{d}{dx}: Q \to Q$
  defined by $\frac{d}{dx}(a x^2 + b x + c) = 0 x^2 + 2 a x + b$.
  \begin{enumerate}
  \item
    Using the standard basis given by $\{x^2, x, 1 \}$, determine a $3
    \times 3$ matrix $A$ that implements $\frac{d}{dx}$.
  \item
    Determine a basis for $\mathrm{Col}(A)$.
  \item
    Determine basis for $\mathrm{Nul}(A)$.
  \item
    Determine $\mathrm{rank}(A)$ and $\mathrm{dim}(\mathrm{Nul}(A))$.
  \end{enumerate}
\item
  Let $A$ be an $m \times n$ matrix and let $\vv$ be a vector in $\R^n$.
  Let $\vb$ denote the vector $A\vv$.
  \begin{enumerate}
  \item
    Show that if $\vw \in \nul A$, then $\vv + \vw$ is a solution to the equation $A\vx = \vb$.
  \item
    Show that if $\vb \not = 0$, then the solution set of $A\vx = \vb$ is not a subspace of $\R^n$.
  \item
    A set $H$ is an \textbf{affine} subspace of $\R^n$ if there is a subspace $U$ of $\R^n$ and a vector $\vo$ such that
    \begin{align*}
      H = \{\vu + \vo \ | \ \vu \in U\}
    \end{align*}
    Show that the solution set of $A\vx = \vb$ is an affine subspace of $\R^n$ if it is nonempty.
    In particular, choose a vector $\vo$ and subspace $U$.
  \end{enumerate}
\item
Determine the linear equation whose solution set is the column space
of the following matrix.
\begin{displaymath}
  \begin{bmatrix}
    1 & -4 & -10 \\
    -4 & 17 & 42 \\
    1 & -2 & -6
  \end{bmatrix}
\end{displaymath}
\item
  Let $A$ be a $4 \times 2024$ matrix where $\rank A = 3$. Further
  suppose that the LU-decomposition of $A$ has
  \begin{displaymath}
    L =
    \begin{bmatrix}
      1 & 0 & 0 & 0 \\
      -2 & 1 & 0 & 0 \\
      7 & 5 & 1 & 0 \\
      0 & 0 & 1 & 1
    \end{bmatrix}
  \end{displaymath}
  Determine the linear equation whose solution set is $\col A$.
\item
  Consider the following two matrices.
  \begin{displaymath}
    A =
    \begin{bmatrix}
      6 & -6 & -1 & 17 \\
      9 & -2 & 2 & 36 \\
      -10 & -13 & 8 & -45 \\
      -9 & 4 & -1 & -33 \\
      2 & 4 & 2 & 14
    \end{bmatrix}
    \qquad
    B =
    \begin{bmatrix}
      -8 & 5 & -16 & 1 \\
      2 & 11 & 4 & 16 \\
      3 & -2 & 6 & 3 \\
      2 & -10 & 4 & 0 \\
      8 & 4 & 16 & 9
    \end{bmatrix}
  \end{displaymath}
  Also consider the following set of vectors.
  \begin{displaymath}
    H = \{ \vu + \vv \ | \ \vu \in \col A \text{ and } \vv \in \col B \}
  \end{displaymath}
  That is, $H$ consists of all sums of pairs of vectors, one from
  $\col A$ and one from $\col B$.
  \begin{enumerate}
  \item
    Show that $H$ is a subspace of $\R^5$.
  \item \faCalculator \
    Determine the dimension of $H$.  Justify your answer.  You may use a
    computer but you must determine any matrix you reduce along with its
    reduced echelon form.
  \end{enumerate}
\end{enumerate}

\section{Challenge Problems}

\begin{enumerate}[resume]
\item
  The row space of a matrix $A \in \mathbb R^{m \times n}$ is the span
  of the rows of $A$, denoted $\mathrm{Row}(A)$. Show that
  $\mathrm{dim}(\mathrm{Row}(A)) + \mathrm{dim}(\mathrm{Nul}(A)) = n$.
\item
  In the vector space of all real-valued functions, find a basis for
  the subspaced spanned by $\{\sin t, \, \sin 2t, \, \sin t \cos t
  \}$.
\end{enumerate}


% ======================================================================
% ======================================================================
% ======================================================================

\chapter{Eigenvectors}
\section{Basic Exercises}

Determine if $\mathbf v$ is an eigenvector of $A$.
If it is, find its corresponding eigenvalue.
\begin{enumerate}
\item
  \begin{displaymath}
    A =
    \begin{bmatrix}
      6 & 1 \\
      -3 & 2
    \end{bmatrix}
    \qquad
    \vv = \vTwo {-1} 1
  \end{displaymath}
\item
  \begin{displaymath}
    A =
    \begin{bmatrix}
      -10 & -3 & -5 \\
      5 & -5 & -3 \\
      5 & 7 & -7
    \end{bmatrix}
    \qquad
    \mathbf v = \vThree 3 0 {-5}
\end{displaymath}
\end{enumerate}
Determine if $\lambda$ is an eigenvalue of $A$.
If it is, find a basis for the corresponding eigenspace.
\begin{enumerate}[resume]
\item
  \begin{displaymath}
    A =
    \begin{bmatrix}
      -2 & 2 \\
      -1 & -4
    \end{bmatrix}
    \qquad
    \lambda = -3
  \end{displaymath}
\item
  \begin{displaymath}
    A =
    \begin{bmatrix}
      5 & -6 & 2 \\
      1 & -2 & 2 \\
      -1 & 6 & 2
    \end{bmatrix}
    \qquad
    \lambda = 4
  \end{displaymath}
\item
  \begin{displaymath}
    A =
    \begin{bmatrix}
      4 & -42 \\
      1 & -9
    \end{bmatrix}
    \qquad
    \lambda = -3
  \end{displaymath}
\item
  \begin{displaymath}
    A =
    \begin{bmatrix}
      3 & -8 & 8 \\
      8 & -13 & 8 \\
      2 & -2 & -3
    \end{bmatrix}
    \qquad
    \lambda = -5
  \end{displaymath}
\item
  \begin{displaymath}
    A =
    \begin{bmatrix}
      3 & -8 & 8 \\
      8 & -13 & 8 \\
      2 & -2 & -3
    \end{bmatrix}
    \qquad
    \lambda = 4
  \end{displaymath}
\end{enumerate}
For the following matrices, determine all eigenvalues and bases for the corresponding eigenspaces.
\begin{enumerate}[resume]
\item
  \begin{displaymath}
    \begin{bmatrix}
      -3 & 2 \\
      -10 & 6
    \end{bmatrix}
  \end{displaymath}
\item
  \begin{displaymath}
    \begin{bmatrix}
      1 & 16 & -12 \\
      0 & -3 & -2 \\
      0 & 0 & -4
    \end{bmatrix}
  \end{displaymath}

\item
  \begin{displaymath}
    \begin{bmatrix}
      3 & 0 & 0 \\
      -1 & 3 & 0 \\
      2 & 4 & 2
    \end{bmatrix}
  \end{displaymath}
\end{enumerate}
Calculate the determinant of the following matrices.
\begin{enumerate}[resume]
\item
  \begin{displaymath}
    \begin{bmatrix}
      -1 & 1 & -1 \\
      3 & 3 & 3 \\
      1 & -3 & -2
    \end{bmatrix}
  \end{displaymath}
\item
  \begin{displaymath}
    \begin{bmatrix}
      3 & -3 & 0 \\
      0 & 3 & -1 \\
      2 & 0 & -1
    \end{bmatrix}
  \end{displaymath}
\end{enumerate}
For each of the following matrices, calculate the determinant of its
inverse.
\begin{enumerate}[resume]
\item
  \begin{align*}
    \begin{bmatrix}
      1 & 3 & 4 \\
      -5 & -4 & -3 \\
      2 & 0 & -5
    \end{bmatrix}
  \end{align*}
\end{enumerate}
For each of the following matrices, determine its characteristic
polynomial.  Use the characteristic polynomial to calculate its
determinant.
\begin{enumerate}[resume]
\item
  \begin{displaymath}
    \begin{bmatrix}
      1 & -1\\
      -1 & 3 \\
    \end{bmatrix}
  \end{displaymath}
\item
  \begin{displaymath}
    \begin{bmatrix}
      1 & 0 & 0 & 0 & 0 \\
      -1 & 5 & 0 & 0 & 0 \\
      2 & 6 & 3 & 0 & 0 \\
      10 & -15 & 3 & 4 & 0 \\
      -1 & 5 & 2 & 5 & 5
    \end{bmatrix}
  \end{displaymath}
\item
  \begin{displaymath}
    \begin{bmatrix}
      1 & 0 & 2 & 10 & 5 \\
      0 & 0 & 5 & -3 & 15 \\
      0 & 0 & 16 & 6 & -1 \\
      0 & 0 & 0 & 1 & 5\\
      0 & 0 & 0 & 7 & 4
    \end{bmatrix}
  \end{displaymath}
\item
  \begin{displaymath}
    \begin{bmatrix}
      1 & 0 & 0 \\
      1 & 2 & 5 \\
      0 & 1 & 3
    \end{bmatrix}
  \end{displaymath}
\end{enumerate}

\section{True/False}

\trueFalseHeader

\begin{enumerate}[resume]
\item
  Every eigenspace of $A$ is a null space (potentially of some other
  matrix).
\item
  Every matrix has at least one eigenvector.
\item
  Every matrix $A \in \mathbb R^{n \times n}$ has an eigenbasis, i.e.,
  a set of eigenvectors that form a basis for $\mathbb{R}^n$.
\item
  Only square matrices can have eigenvectors.
\item
  If $0$ is an eigenvalue of $A$, then $A$ is not invertible.
\item
  The determinant of a matrix does not change under elementary row
  operations.
\item
  If $\det (A^2) = 1$, then $\det (A) = 1$.
\item
  A matrix with characteristic polynomial $(x+2)^3$ has a single
  eigenspace with dimension $3$.
\item
  For any square matrices $A$ and $B$, if $A \sim B$ then $\det(A) =
  \det(B)$.
\item
  $\det(5A) = 5\det A$ for any square matrix $A$.
\item
  $\det(A^TA) \geq 0$ for any square matrix $A$.
\item
  $\det(A + B) = \det(A) + \det(B)$ for any square matrices $A$ and
  $B$.
\item
  If $A$ is invertible, then $\det(ABA^{-1}) = \det(B)$.
\end{enumerate}

\section{More Difficult Problems}

\begin{enumerate}[resume]
\item
  Let $A$ be the following matrix.
  \begin{displaymath}
    \begin{bmatrix}
      -17 & 28 & 14 \\
      -7 & 11 & 7 \\
      -7 & 14 & 4
    \end{bmatrix}
  \end{displaymath}
  \begin{enumerate}
  \item
    Determine if the following vectors are eigenvectors of $A$. For
    the ones that are, find their associated eigenvalues.
    \begin{displaymath}
      \vv_1 = \vThree 1 1 1
      \qquad
      \vv_2 = \vThree 2 2 0
      \qquad
      \vv_3 = \vThree 2 1 1
    \end{displaymath}
  \item
    Show that $-3$ is an eigenvalue of $A$ without doing any row
    operations.  \textit{Hint:} Use the invertible matrix theorem.
  \item
    Find a basis for the eigenspace of $A$ corresponding to the
    eigenvalue $-3$.
\end{enumerate}
\item Let $A$ and $B$ be square matrices such that $\det A = 3.5$ and
  $\det B = -2$.  Compute the determinant of the matrix
  $B(AB)^{-1}(AB)^TA$.
\item Consider the following reduction sequence
  \begin{align*}
    R_1 &\gets R_1 + R_2 \\
    \mathsf{swap}&(R_2, R_3) \\
    R_3 &\gets R_3 + 5R_4 \\
    R_2 &\gets -3R_2 \\
    R_5 &\gets R_5 - 10R_3 \\
    R_5 &\gets R_5 / 11 \\
    \mathsf{swap}&(R_5, R_3) \\
    \mathsf{swap}&(R_1, R_2) \\
    R_4 &\gets R_4 + R_1 \\
    R_2 &\gets 5R_2 \\
    R_1 &\gets -R_1
  \end{align*}
  Suppose that $A \in \R^{5 \times 5}$ reduces to $U$ by this sequence
  of reductions, where $U$ is in \textit{reduced} echelon form.  Given
  $\rank A = 5$, determine $\det A$.
\item
  Let $A$ be as in the previous problem. Given $\rank A = 4$, determine $\det A$.
\item
  The characteristic polynomial of the following matrix is
  $-\lambda^3+5\lambda^2-6\lambda$.
  \begin{displaymath}
    \begin{bmatrix}
      0 & 3  & -3 \\
      -2 & 6 & -4 \\
      -2 & 3 & -1
    \end{bmatrix}
  \end{displaymath}
  Find the eigenvalues and bases for the eigenspaces.
\item
  For each of the following statements, find a counterexample by giving
  explicit $2 \times 2$ matrices $A$ and $B$ which falsify the
  statement.  Justify your answer.
  \begin{enumerate}
  \item
    $\det(A + B) = \det(A) + \det(B)$ for any matrices $A$ and $B$ in
    $\R^{n\times n}$.
  \item
    For any square matrix $A$, if $\det(A - \lambda I) = (\lambda - 1)^2$,
    then $\dim(\nul (A - I)) = 2$.
  \item
    For any matrices $A$ and $B$ in $\R^{n \times n}$, if $\det(A -
    \lambda I) = \det(B - \lambda I)$ (i.e., they have the same
    characteristic polynomial) then $A$ is similar to $B$.
  \end{enumerate}
\item
  Let $A$ be a $5 \times 5$ matrix with characteristic polynomial
  $(\lambda - 2)^2 (\lambda + 1)^2 \lambda$ and $\rank(A - 2I) =
  \rank(A + I) = 3$.
  \begin{enumerate}
  \item
    Determine $\rank A$. Justify your answer.
  \item
    Determine if $A$ is diagonalizable. Justify your answer.
  \end{enumerate}
\item
  Suppose that $AP = PD$ for a square matrix $A$, diagonal matrix $D$,
  and arbitrary $m \times n$ matrix $P$. Show that the nonzero columns
  of $P$ are eigenvectors of $A$ and find their corresponding
  eigenvalues in terms of entries of $D$.
\item
  Consider the following pair of vectors.
  \begin{align*}
    \vv_1 = \vFour 1 {-2} {-1} 3
    \qquad
    \vv_2 = \vFour 3 {-6} {-2} 2
  \end{align*}
  \begin{enumerate}
  \item
    Determine the matrix equations whose solution set consists exactly
    of those vectors which are orthogonal to $\vv_1$ and $\vv_2$.
  \item
    Determine a general-form solution for the matrix equation from the
    previous part.
  \item
    Determine the dimension of the subspace of vectors orthogonal to
    both $\vv_1$ and $\vv_2$
  \end{enumerate}
\item
  Show that the angle between two vectors does not depend on their
  length. In other words, show that for any pair of vectors $\vu$ and
  $\vv$ in $\R^n$, the angle between them is equal to the angle
  between $a\vu$ and $b\vv$ for any nonzero real numbers $a$ and $b$.
\item
  Suppose the eigenvalues of a $3 \times 3$ matrix $A$ are $\lambda_1
  = 2$, $\lambda_2 = 1/2$, and $\lambda_3 = 1/4$ with corresponding
  eigenvectors:
  \begin{displaymath}
    \vv_1 = \vThree 1 {-1} 3
    \qquad
    \vv_2 = \vThree {-3} 4 9
    \qquad
    \vv_3 = \vThree = 2 {-2} 4
  \end{displaymath}
  Given a starting state $\vx_0 = [8, -9, 6]^T$, give a closed form
  expression for the state after $k$ iterations: $A^k \vx_0$. Describe
  what happens as $k \to \infty$.
\item
  Recall that counterclockwise rotation by angle $\theta$ in the plane
  can be implemented by the matrix below.
  \begin{align*}
    R_\theta = \begin{bmatrix}
      \cos \theta & -\sin \theta \\
      \sin \theta & \cos \theta
    \end{bmatrix}
  \end{align*}
  Calculate $\det (R_\theta)$.
  \textit{Hint.} The value does not depend on $\theta$.
\item
  Let $R_\theta$ be as in the previous problem.
  \begin{enumerate}
  \item
    Determine the characteristic polynomial of $R_\theta$ as a function
    of $\theta$.
  \item
    Determine the values of $\theta$ for which $R_\theta$ has eigenvalues.
    For those values, also determine bases for the corresponding eigenspaces.
  \end{enumerate}
\item
  Give two $2 \times 2$ matrices where every nonzero vector is an
  eigenvector.
\item \faCalculator \
  Let $A$ be a matrix such that
  \begin{displaymath}
    A \vThree 1 1 {-3} = \vThree {-1} {-1} 3
    \qquad
    A \vThree 0 1 1 = \vThree 0 1 1
    \qquad
    A \vThree 2 {-2} {-9} = \vThree 4 {-4} {-18}
  \end{displaymath}
  Without determining $A$, determine the vector
  \begin{displaymath}
    A^5 \vThree 3 0 {-11}
  \end{displaymath}
  You may use a computer, but you must show your work and justify your
  answer.
\item
  Let $A$ be the matrix from the previous part.
  Consider the basis
  \begin{displaymath}
    \cB =
    \left\{
    \vThree 1 1 {-3},
    \vThree 0 1 1,
    \vThree 2 {-2} {-9}
    \right\}
  \end{displaymath}
  Determine the matrix $C$ such that $C[\vv]_{\cB} = A \vv$ for all $\vv \in \R^3$.
\item
Determine if the following matrix has an eigenbasis. Justify your
answer.
\begin{displaymath}
  \begin{bmatrix}
    1 & 2 & -4 & 1 \\
    0 & 1 & 4 & 1 \\
    0 & 0 & 0 & 3 \\
    0 & 0 & 0 & 2
  \end{bmatrix}
\end{displaymath}
\end{enumerate}

\section{Challenge Problems}

\begin{enumerate}[resume]
\item
  Show that a pair of eigenvectors $\vv_1$ and $\vv_2$ with
  distinct eigenvalues $\lambda_1$ and $\lambda_2$ is linearly
  independent.
\item
  Show that the determinant formula from lecture is correct.
  \textit{Hint:} Consider applying elementary row operations via left
  multiplication by elementary row matrices, and use the fact that
  \begin{displaymath}
    \det (E_k \ldots E_2 E_1 A) = \det E_k \ldots \det E_2 \det E_1 \det A
  \end{displaymath}
\item
  Consider the following matrix.
  \begin{displaymath}
    A =
    \begin{bmatrix}
      1 & 1 \\
      1 & 0
    \end{bmatrix}
  \end{displaymath}
  \begin{enumerate}
  \item
    Verify that the following vectors form an eigenbasis of $A$.  Also
    determine the eigenvalues for each eigenvector.
    \begin{displaymath}
      \vv_1 = \vTwo {\frac{1 + \sqrt{5}}{2}} 1
      \qquad
      \vv_2 = \vTwo {\frac{1 - \sqrt{5}}{2}} 1
    \end{displaymath}
  \item
    Write the vector $\ve_1$ in terms of the eigenbasis you found.  In
    other words, determine $\alpha_1$ and $\alpha_2$ such that
    \begin{displaymath}
      \begin{bmatrix}
        1 \\ 0
      \end{bmatrix}
      =
      \alpha_1\vv_1 + \alpha_2 \vv_2
    \end{displaymath}
    \textit{Hint:} Calcuate $\vv_1 - \vv_2$.
  \item
    Write down a closed-form solution for the linear dynamical system
    determined by $A$ with initial vector $\ve_1$.
  \item \faCalculator \
    If you look at the formula given by the second component of your
    closed-form solution from the previous part, this gives a
    \textit{non-recursive} definition for Fibonacci numbers.  Write
    down this formula and use it to calculate $F_{20}$, the $20$th
    fibonacci number (where $F_0 = 0$ and $F_1 = 1$).
  \end{enumerate}
\item
  A Jordan block $J$ is a square matrix in $\R^{n \times n}$ of the following form.
  \begin{displaymath}
    \begin{bmatrix}
      \alpha & 1 & & \\
      & \alpha & \ddots & \\
      & & \ddots & 1 \\
      & & & \alpha
    \end{bmatrix}
  \end{displaymath}
  where $\alpha \in \R$.
  \begin{enumerate}
  \item
    Determine the characteristic polynomial (in terms of $n$) of $J$.
  \item
    Determine the dimension of the eigenspace of $J$ for the
    eigenvalue $\alpha$.
  \item
    Construct six matrices, each with the characteristic polynomial
    $(\lambda - 3)^3(\lambda + 2)^2$, that achieve all possible
    combinations of eigenspace dimensions for $\lambda = 3$ (1,2, or
    3) and for $\lambda = 2$ (1 or 2).
  \end{enumerate}
\end{enumerate}


% ======================================================================
% ======================================================================
% ======================================================================

\chapter{Diagonalization}
\section{Basic Exercises}

Determine a diagonalization of the following matrix, if it exists.
You can leave the rightmost factor in the form $P^{-1}$, i.e., you
don't have to compute the inverse of $P$.
\begin{enumerate}
\item
  \begin{displaymath}
    \begin{bmatrix}
      9 & 4 \\
      -14 & -6
    \end{bmatrix}
  \end{displaymath}
\item
  \begin{displaymath}
    \begin{bmatrix}
      2 & -6 \\
      2 & -5
    \end{bmatrix}
  \end{displaymath}
\item
  \begin{displaymath}
    \begin{bmatrix}
      -6 & -7\\
      7 & 8
    \end{bmatrix}
  \end{displaymath}
\item
  \begin{displaymath}
    \begin{bmatrix}
      -2 & 0 & 0 \\
      2 & -1 & -1 \\
      6 & 4 & -5
    \end{bmatrix}
  \end{displaymath}
\item
\begin{displaymath}
  \begin{bmatrix}
    -1 & -3 & -6 \\
    -8 & -6 & -18 \\
    4 & 3 & 9
  \end{bmatrix}
\end{displaymath}
\textit{Hint:} The characteristic polynomial of the above matrix is
$\lambda^{3} - 2 \lambda^{2} - 3 \lambda$.
\item
  \begin{displaymath}
    \begin{bmatrix}
      2 & 0 & 0 & 0 \\
      -1 & 3 & 0 & 0 \\
      3 & -3 & 2 & 0 \\
      -3 & 18 & -6 & -1
    \end{bmatrix}
  \end{displaymath}
\item
  \begin{displaymath}
    \begin{bmatrix}
      -1 & -3 & 0 \\
      2 & 4 & 0 \\
      0 & 0 & 1
    \end{bmatrix}
  \end{displaymath}
\item \faCalculator
  \begin{displaymath}
    \begin{bmatrix}
      0 & 2 & 2 \\
      5 & -6 & -5 \\
      -9 & 12 & 11
    \end{bmatrix}
  \end{displaymath}
  In addition, define $P$ so that its entries are integers.
\end{enumerate}

\section{True/False}

\trueFalseHeader

\begin{enumerate}[resume]
\item
  All square matrices are diagonalizable.
\item
  Similar matrices have the same eigenvalues.
\item
  Similar matrices have the same eigenvectors.
\item
  If a matrix does not have $n$ distinct eigenvalues, then it is not
  diagonalizable.
\item
  If a matrix is diagonalizable, then it is invertible.
\item
  A diagonalization of a matrix $A$ if it exists, is unique.
\end{enumerate}

\section{More Difficult Problems}

\begin{enumerate}
\item \faCalculator \
  Consider the following matrix.
  \begin{displaymath}
    A =
    \begin{bmatrix}
      103 & -8 & -47 \\
      24 & 1 & -12 \\
      198 & -16 & -90
    \end{bmatrix}
  \end{displaymath}
  Determine a matrix $B$ such that $B^2 = A$. You can use a computer
  to find eigenvectors or to reduce matrices, but you must otherwise
  show your work and justify your answer.  \textit{Hint:} $A$ is
  diagonalizable.
\end{enumerate}

\section{Challenge Problems}

\begin{enumerate}[resume]
\item
  Consider an arbitrary $2 \times 2$ matrix of the following form.
  \begin{displaymath}
    A =
    \begin{bmatrix}
      a & b \\
      c & d
    \end{bmatrix}
  \end{displaymath}

  \begin{enumerate}
  \item
    Determine an expression for the characteristic polynomial of $A$
    in terms of $a$, $b$, $c$, and $d$.
  \item
    Recall that for a quadratic polynomial $p(x) = ix^2 + jx + k$, the
    discriminant $j^2 - 4ik$ tells us how many roots $p$ has, i.e.,
    $p$ has $0$, $1$ or $2$ roots if the discriminant is less than
    $0$, equal to $0$, or greater than $0$, respectively.  Use this to
    derive an expression $E$ in terms of $a$, $b$, $c$, and $d$ where
    $A$ has $0$, $1$, or $2$ eigenvalues if $E < 0$, $E = 0$, or $E >
    0$, respectively.
  \item Using the expression from the previous part, argue that
    \textit{every $2 \times 2$ matrix with positive entries has two
      distinct eigenvalues.}  Note that this implies every $2 \times
    2$ matrix with positive entries is diagonalizable.  \textit{Hint:}
    Try to write the expression from the previous part so that it is
    of the form `$(\square - \square)^2 + 4\square\square$' and reason
    about why this must be positive.
  \end{enumerate}
\end{enumerate}


% ======================================================================
% ======================================================================
% ======================================================================

\chapter{Analytic Geometry and Orthogonality}
\section*{Basic Exercises}

\faCalculator \ For each pair of vectors $\vv_1$ and $\vv_2$, determine:
\begin{itemize}
\item the lengths of $\vv_1$ and $\vv_2$;
\item the angle between $\vv_1$ and $\vv_2$;
\item the distance between $\vv_1$ and $\vv_2$;
\item the unit length normalizations of $\vv_1$ and $\vv_2$.
\end{itemize}
You must simplify the expression as much as you can before giving the approximate result to a couple decimal places.
\begin{enumerate}
\item
  \begin{displaymath}
    \vv_1 = \vTwo 7 3
    \qquad
    \vv_2 = \vTwo {-1}  5
  \end{displaymath}
\item
  \begin{displaymath}
    \vv_1 = \vThree {-2} 4 {-5}
    \qquad
    \vv_2 = \vThree 2 2 {-2}
  \end{displaymath}
\item
  \begin{displaymath}
    \vv = \vFive 1 {-1} 5 7 4
    \qquad
    \vu = \vFive 3 1 3 {-1} 0
  \end{displaymath}
  Additionally, without doing any further calculations determine
  approximately the angle between $\vv$ and $\vv - \vu$.  Justify your
  answer.
%% \item
%%   \begin{displaymath}
%%     \vv_1 = \vFive 1 {-1} 5 7 4
%%     \qquad
%%     \vv_2 = \vFive 3 0 3 {-5} {-7}
%%   \end{displaymath}
\end{enumerate}
Determine if each of the following set of vectors is an orthogonal set.
\begin{enumerate}[resume]
\item
  \begin{displaymath}
    \left\{
    \vThree 9 {-18} {-18},
    \vThree {-2} {-14} {13}
    \right\}
  \end{displaymath}
\item
  \begin{displaymath}
    \left\{
    \vFour 0 4 4 0,
    \vFour {-2} 2 0 0,
    \vFour 2 1 {-1} {-6}
    \right\}
  \end{displaymath}
\end{enumerate}
Express the vector $\vv$ in terms of the given orthonormal basis
$\cB$.
\begin{enumerate}[resume]
\item
  \begin{displaymath}
    \vv = \vTwo{-8}{9}
    \qquad
    \cB =
    \left\{
    \vTwo{-\frac{\sqrt{2}}{2}}{\frac{\sqrt{2}}{2}},
    \vTwo{\frac{\sqrt{2}}{2}}{\frac{\sqrt{2}}{2}}
    \right\}
  \end{displaymath}
\item
  \begin{displaymath}
    \vv = \vThree 3 9 {-3}
    \qquad
    \cB =
    \left\{
    \vThree {\frac{\sqrt{2}}{2}} {\frac{\sqrt{2}}{2}} {0},
    \vThree {\frac{\sqrt{6}}{6}} {-\frac{\sqrt{6}}{6}} {-\frac{\sqrt{6}}{3}},
    \vThree {-\frac{\sqrt{3}}{3}} {\frac{\sqrt{3}}{3}} {-\frac{\sqrt{3}}{3}}
    \right\}
  \end{displaymath}
\end{enumerate}
Determine the projection of $\mathbf v$ onto the span of the given set of vectors.
\begin{enumerate}[resume]
\item
  \begin{displaymath}
    \vv = \vTwo 9 7
    \qquad
    U =
    \left\{
    \vTwo 7 {-9}
    \right\}
  \end{displaymath}
\item
  \begin{displaymath}
    \vv = \left[\begin{matrix}3\\3\\5\end{matrix}\right]
    \qquad
    U =
    \left\{
    \left[\begin{matrix}-2\\0\\-10\end{matrix}\right]
    \right\}
  \end{displaymath}
\item
  \begin{displaymath}
    \vv = \left[\begin{matrix}-3\\6\\2\end{matrix}\right]
    \qquad
    U =
    \left\{
    \left[\begin{matrix}- \frac{\sqrt{2}}{2}\\\frac{\sqrt{2}}{2}\\0\end{matrix}\right],
    \left[\begin{matrix}\frac{2}{3}\\\frac{2}{3}\\- \frac{1}{3}\end{matrix}\right]
    \right\}
  \end{displaymath}
\end{enumerate}
\section*{True/False}

\trueFalseHeader

\begin{enumerate}
\item
  If $\|\vu\|^2 + \|\vv\|^2 = \|\vu + \vv\|^2$, then $\vu$ and $\vv$
  are orthogonal.
\item
  For a square matrix $A$, vectors in $\col A$ are orthogonal to
  vectors in $\nul A$.
\item
  There can be a linear dependence relationship between vectors in an
  orthogonal set.
\item
  In $\R^4$, any set of vectors that has 5 members cannot be an
  orthogonal set.
\item
  Orthogonal matrices are invertible.
\item
  Orthogonal matrices have determinant 1.
\item
  For an $m \times n$ matrix with orthogonal columns, it may be the
  case that $m < n$.
\item
  The orthogonal projection of $\vy$ onto $\vv$ is the same as the
  orthogonal projection of $\vy$ onto $c \vv$ whenever $c \neq 0$.
\item
  For any vector $\vy \in \R^n$ and any subspace $W$ of $\R^n$, the
  vector $\vy - \proj_W \vy$ is orthogonal to $W$.
\end{enumerate}

\section*{More Difficult Problems}

\begin{enumerate}
\item
  Determine if the following six vectors form an orthogonal set.
  Justify your answer. \textit{Hint:} You do not need to explicitly
  check every pair for orthogonality, you can argue more generally.
  \begin{displaymath}
    \vv_1 = \begin{bmatrix} 1 \\ -2 \\ 1 \\ 0 \\ 0 \\ 0 \end{bmatrix}
    \qquad
    \vv_2 = \begin{bmatrix} 0 \\ 1 \\ 2 \\ 0 \\ 0 \\ 0 \end{bmatrix}
    \qquad
    \vv_3 = \begin{bmatrix} -5 \\ -2 \\ 1 \\ 0 \\ 0 \\ 0 \end{bmatrix}
  \end{displaymath}
  \begin{displaymath}
    \vv_4 = \begin{bmatrix} 0 \\ 0 \\ 0 \\ 3 \\ -3 \\ 0 \end{bmatrix}
    \qquad
    \vv_5 = \begin{bmatrix} 0 \\ 0 \\ 0 \\ 2 \\ 2 \\ -1 \end{bmatrix}
    \qquad
    \vv_6 = \begin{bmatrix} 0 \\ 0 \\ 0 \\ 1 \\ 1 \\ 4 \end{bmatrix}
  \end{displaymath}
\item Project $\vb$ onto $\vspan\{ \va_1, \va_2, \va_3 \}$.
  \begin{displaymath}
    \vb = \vFour 2 5 6 6
    \qquad
    \va_1 = \vFour 1 1 0 {-1}
    \qquad
    \va_2 = \vFour 1 0 1 1
    \qquad
    \va_3 = \vFour 0 {-1} 1 {-1}
  \end{displaymath}
\item Show that any $2 \times 2$ rotation matrix $R_\theta$ is an orthogonal matrix.
\item
  Show that any $3 \times 3$ rotation matrix $R_x^\theta$,
  $R_y^\theta$, $R_z^\theta$ (about the $x$-axis, $y$-axis, and $z$-axis,
  respectively) is an orthogonal matrix.
\item
  Determine the matrix that implements orthogonal projection onto
  $\vspan\{ [2 \ \ 3 \ \ 1]^T \}$.
\item Determine a matrix (in terms of $\vu_1$ and $\vu_2$) that
  implements orthogonal projection onto $\vspan \{ \vu_1, \vu_2 \}$
  where $\vu_1$ and $\vu_2$ are orthogonal.
\item
  Show that for any nonzero vector $\vv \in \R^n$, the set of vectors
  orthogonal to $\vv$ form a subspace of $\R^n$.
\item
  Determine a basis for the set of vectors orthogonal to the following vector.
  \begin{displaymath}
    \vThree 1 4 {-3}
  \end{displaymath}
\item
  Determine a basis for the set of vectors orthogonal to every vector in the solution
  set of the following linear equation.
  \begin{displaymath}
    2x_1 + 3x_2 - 4x_3 + 5x_4 = 0.
  \end{displaymath}
\item
  Consider the following pair of vectors.
  \begin{displaymath}
    \vu = \vFour 1 2 1 {-2}
    \qquad
    \vv = \vFour {-3} {-6} {-2} 2
  \end{displaymath}
  For this problem, we will be using theorem that isn't too difficult
  to show: every vector in $\nul A^T$ is orthogonal to every vector in
  $\col A$.  Use this to determine two linearly independent vector
  with integer entries that are orthogonal to every vector in $\vspan
  \{ \vu , \vv \}$.
\item
  For this problem, we will use another theorem that's more difficult
  to prove: $\rank A + \dim(\nul A^T) = m$ for any matrix $A \in \R^{m
    \times n}$.  Use this to show that $\rank A = \rank A^T$.
\item
  Consider the following vectors.\footnote{This problem goes over the
  \textit{Gram-Schmidt process}, an algorithm for converting
  a basis into an orthogonal basis.}
  \begin{displaymath}
    \vv_1 = \vFour 2 1 0 {-1}
    \qquad
    \vv_2 = \vFour 3 0 0 0
    \qquad
    \vv_3 = \vFour {-4} 9 0 {-11}
    \qquad
    \vu = \vFour 2 {-2} 0 {-5}
  \end{displaymath}
  \begin{enumerate}
  \item
    Find the component of $\vv_2$ orthogonal to $\vv_1$ (that is, find
    the vector $\vz$ such that $\vv_2 = \hat{\vv_2} + \vz$ where
    $\hat{\vv_2}$ is the orthogonal projection of $\vv_2$ onto
    $\vv_1$). We will refer to this as $\vv_2'$ below.
  \item
    Find the component of $\vv_3$ orthogonal to $\vv_1$. We will refer
    to this as $\vv_3'$ below.
  \item
    Find the component of $\vv_3'$ orthogonal to $\vv_2'$. We will
    refer to this as $\vv_3''$ below.
  \item
    Demonstrate that $\{\vv_1, \vv_2', \vv_3''\}$ is an orthogonal
    set.
  \item
    Find $[\mathbf u]_{\cB}$ where $\cB = \{\vv_1, \vv_2', \vv_3''\}$
    without any row reductions.
  \end{enumerate}
\item
  Repeat the previous problem with the following vectors.
  \begin{displaymath}
    \vv_1 = \vFour 1 0 1 1
    \qquad
    \vv_2 = \vFour 0 1 1 2
    \qquad
    \vv_3 = \vFour 1 0 2 {-2}
    \qquad
    \vu = \vFour2 {-2} 1 {-5}
  \end{displaymath}
\item
  \begin{displaymath}
    A =
    \begin{bmatrix}
      1 & 2 & -5 \\
      1 & 1 & 3 \\
      8 & 14 & -6 \\
      1 & 2 & 1
    \end{bmatrix}
    \qquad
    B =
    \begin{bmatrix}
      1 & -4 & -2 & 9 & 0 \\
      0 & 2 & 2 & -6 & 1 \\
      3 & -5 & 1 & 9 & 5 \\
      0 & 1 & 1 & -2 & 1
    \end{bmatrix}
  \end{displaymath}
  \begin{enumerate}
  \item
    \faCalculator \ Determine approximately the matrix that implements
    orthogonal projection onto $\col A$.
  \item
    \faCalculator \ Determine approximately the matrix which
    implements orthogonal projection onto $\col B$.  \textit{Hint:}
    First determine $\rank B$.
  \item
    Compare the matrices from the previous two parts.  Explain what
    this implies about the relationship between $\col A$ and $\col B$.
    Justify your answer.
  \end{enumerate}
\end{enumerate}

\section*{Challenge Problems}

\begin{enumerate}[resume]
\item
  Show that the row vectors of an orthogonal matrix form an
  orthonormal set.
\item
  Show that the product of two orthogonal matrices is orthogonal.
\item
  Show that $\vv \in \nul A^T$ if and only if $\vv$ is orthogonal to every
  vector in $\col A$ for any matrix $A \in \R^{m \times n}$ and vector
  $\vv \in \R^m$.
\item
  Show that $\rank A + \dim(\nul A^T) = m$ for any matrix $A \in \R^{m
    \times n}$.
\end{enumerate}


% ======================================================================
% ======================================================================
% ======================================================================

\chapter{Least Squares and Linear Models}
\section{Basic Exercises}

Solve for all least-squares solutions of the equation $A\vx = \vb$ using the normal equations.
\begin{enumerate}
\item
  \begin{displaymath}
    A = \left[\begin{matrix}-3 & 0\\0 & 3\\3 & -6\end{matrix}\right]
    \qquad
    \vb = \left[\begin{matrix}-3\\8\\3\end{matrix}\right]
  \end{displaymath}
\item
  \begin{displaymath}
    A =
    \begin{bmatrix}
      1 & 2 \\
      0 & -1 \\
      1 & -1
    \end{bmatrix}
    \qquad
    \vb = \vThree 4 2 0
  \end{displaymath}
\item
  \begin{displaymath}
    A = \left[\begin{matrix}1 & 1 & -4\\0 & 1 & -3\\-1 & -2 & 7\end{matrix}\right]
    \qquad
    \vb = \left[\begin{matrix}-7\\-10\\2\end{matrix}\right]
  \end{displaymath}
\item
  \begin{displaymath}
    A = \left[\begin{matrix}-1 & 1\\-3 & 1\\-3 & 0\end{matrix}\right]
    \qquad
    \vb = \left[\begin{matrix}1\\11\\15\end{matrix}\right]
  \end{displaymath}
\item
  \begin{displaymath}
    A =
    \begin{bmatrix}
      1 & 0 & 1 \\
      1 & 0 & 1 \\
      1 & 0 & 1 \\
      1 & 1 & 0 \\
      1 & 1 & 0 \\
      1 & 1 & 0 \\
    \end{bmatrix}
    \qquad
    \vb =
    \begin{bmatrix}
      6 \\ 5 \\ 4 \\ 7 \\ 2 \\ 3
    \end{bmatrix}
  \end{displaymath}
\item
  Determine the least squares linear fit to the following data points.
  Sketch the graphs to verify your answer.  You should set up the
  required matrix equations by hand, but you can use a computer to
  determine the coefficients of your model.
  \begin{displaymath}
    \{(-5, 2), (-2, 1), (0, 0), (2, 4), (6, 6)\}
  \end{displaymath}
\item Determine the least squares quadratic fit to the data points
  from the previous problem. Sketch the graphs to verify your answer.
  You should set up the required matrix equations by hand, but you can
  use a computer to determine the coefficients of your model.
\item Determine the design matrix for the following function model
  using the data points from the previous problem.
  \begin{displaymath}
    f(x) = \beta_12^x + \beta_2x\sin(x) + \beta_3x^2 + 7\beta_4
  \end{displaymath}
\item
  Determine the least-squares fit model for the given data.  Round the
  parameters to the nearest hundredth.
  \begin{displaymath}
    \{(-2, 0), (-1, 5), (0, 13), (1, 9), (2, 5), (3, 0)\}
  \end{displaymath}
\item
  Determine the design matrix for the following function model using
  the data points from the previous problem.
  \begin{displaymath}
    f(x) = \beta_1 \cos x + \beta_2 \sin x + \beta_3 x + \beta_4
  \end{displaymath}
\item Determine an orthogonal diagonalization of the following matrix.
  \begin{displaymath}
    \begin{bmatrix}
      3 & 1 \\
      1 & 3
    \end{bmatrix}
  \end{displaymath}
\end{enumerate}

\section{True/False}

\trueFalseHeader

\begin{enumerate}[resume]
\item
  Given any matrix equation $A\vx = \vb$, there always exists a
  least-squares solution.
\item
  Given any matrix equation $A\vx = \vb$, there always exists a unique
  least-squares solution.
\item
  A least-squares solution of $A\vx = \vb$ is a vector $\hat{\vx}$
  that satisfies $A\hat{\vx} = \hat{\vb}$, where $\hat{\vb}$ is the
  orthogonal projection of $\vb$ onto $\col A$.
\item
  Given a model function $f(x) = \beta_0 + \beta_1^2 x$, we may
  minimize for least squares error by using the least squares solution
  to a matrix equation.
\item
  If $X$ denotes the design matrix for a least squares regression
  problem, then $X^T X$ is invertible.
\item
  There are orthogonally diagonalizable matrices that are not symmetric.
\item
  Every $n \times n$ symmetric matrix must has $n$ distinct eigenvalues.
\item
  For a symmetric matrix, the dimension of each eigenspace is equal to
  the algebraic multiplicity of the corresponding eigenvalue.
\item
  $\|\vx\|^2$ is a quadratic form.
\item
  A positive definite quadratic form $Q(\vx)$ satisfies $Q(\vx) > 0$
  for all $\vx \in \R^n$.
\item
  $\vx^T A \vx$ defines a quadratic form only if $A$ is symmetric.
\end{enumerate}

\section{More Difficult Problems}

\begin{enumerate}[resume]
\item
  Determine a formula for the least-squares solution of $A\vx = \vb$
  given that the columns of $A$ are orthonormal.  Furthermore, explain
  why this solution is unique.
\item
  Let $C$ be an arbitrary $m \times n$ matrix of rank $k$ and let
  $\mathbf b$ be an arbitrary vector in $\R^m$.  Determine the maximum
  size of a linearly independent set of least squares solutions for
  $C\mathbf x = \mathbf b$ where $\vb \not = \vzero$.  Justify your
  answer.  Your solution should be given in terms of $m$, $n$, and
  $k$.
\item \faCalculator
  Consider the following multivariate function model
  \begin{displaymath}
    f(x,y) = \beta_0 + \beta_1 x + \beta_2 y
  \end{displaymath}
  and solve for the real number parameters $\beta_0$, $\beta_1$, and
  $\beta_2$ that minimize least squares error for the following data
  points.
  \begin{displaymath}
    \{(-2,0,0), (0,0,3), (0,2,0), (-2, -3, 3), (1,5,-1)\}.
  \end{displaymath}
  Justify your answer.
\end{enumerate}

\section{Challenge Problems}


% ======================================================================
% ======================================================================
% ======================================================================

\chapter{Quadratic Forms}
\section{Basic Exercises}

Determine the symmetric matrix $A$ such that $Q(\vx) = \vx^T A \vx$.
\begin{enumerate}
\item
  \begin{displaymath}
    Q([x_1 \ x_2 \ x_3]^T)
    = 3x_1^2 + 4 x_1 x_2 + 5 x_2 x_1 - x_1 x_3
  \end{displaymath}
\item
  \begin{displaymath}
    Q([x_1 \ x_2 \ x_3]^T)
    = 3x_1^2 + 3x_2^2 + 5x_3^2 + 6x_1 x_2 + 2x_1 x_3 + 2x_2 x_3
  \end{displaymath}

\end{enumerate}
Determine the quadratic form $Q(\vx)$ such that $Q(\vx) = \vx^T A
\vx$.
\begin{enumerate}[resume]
\item
  \begin{align*}
    A =
    \begin{bmatrix}
      2 & 0 & 0 \\
      0 & 9 & 3 \\
      0 & 3 & 1
    \end{bmatrix}
  \end{align*}
\end{enumerate}
Determine the type of the quadratic form $Q(\vx) = \vx^T A \vx$.
\begin{enumerate}[resume]
\item
  \begin{displaymath}
    A =
    \begin{bmatrix}
      2 & 0 & 0 \\
      0 & 9 & 3 \\
      0 & 3 & 1
    \end{bmatrix}
  \end{displaymath}
\end{enumerate}

\section{True/False}

\trueFalseHeader

\section{More Difficult Problems}

\begin{enumerate}[resume]
\item
  Consider the quadratic form $Q(\vx) = \vx^T A \vx$ where $A$ is
  defined as follows.
  \begin{displaymath}
    \begin{bmatrix}
      1 & 1 & 5 \\
      1 & 5 & 1 \\
      5 & 1 & 1
    \end{bmatrix}
  \end{displaymath}
  Determine the following.
  \begin{enumerate}
  \item
    $\min_{\|\vx\| = 1} Q(\vx)$
  \item
    $\max_{\|\vx\| = 1}  Q(\vx)$
  \item
    $\argmin_{\|\vx\| = 1} Q(\vx)$
  \item
    $\argmax_{\|\vx\| = 1} Q(\vx)$
  \end{enumerate}
  \textit{Hint:} $4$ is an eigenvalue of $A$.
\item
  \faCalculator \ Repeat the previous problem with the following
  quadratic form.
  \begin{displaymath}
    Q([x_1 \ x_2 \ x_3]^T) =
    3x_1^2 + 3x_2^2 + 5x_3^2 + 6x_1 x_2 + 2x_1 x_3 + 2x_2 x_3
  \end{displaymath}
\end{enumerate}

\section{Challenge Problems}


% ======================================================================
% ======================================================================
% ======================================================================

\chapter{Singular Value Decomposition}
\section{Basic Exercises}

For each matrix, determine a singular value decomposition.
\begin{enumerate}
\item
  \begin{displaymath}
    \begin{bmatrix}
      3 & 2 \\
      2 & 3 \\
      2 & -2
    \end{bmatrix}
  \end{displaymath}
\end{enumerate}

\section{True/False}

\trueFalseHeader

\begin{enumerate}[resume]
\item
  An orthogonal diagonalization of a symmetric matrix $A = PDP^T$ is
  also a singular value decomposition of $A$.
\end{enumerate}

\section{More Difficult Problems}

\section{Challenge Problems}


\end{document}
