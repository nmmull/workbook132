Let $G = (V, E)$ be a graph with $n$ vertices $v_1, v_2, \dots, v_n$.
In class we discussed the $n \times n$ adjacency matrix given by
\begin{align*}
A_{ij} =
\begin{cases}
1 & (v_i, v_j) \in E \\
0 & (v_i, v_j) \not \in E
\end{cases}
\end{align*}
There are many other matrices based on graphs. One is the $n \times n$
*degree matrix* given by
\begin{align*}
D_{ij} =
\begin{cases}
\deg(v_i) & i = j\\
0 & i \not = j
\end{cases}
\end{align*}
where $\deg(v)$ is the number of edges adjacent to $v$.  Another is
the $n \times n$ *Laplacian matrix* given by
\begin{align*}
L = D - A
\end{align*}

For the following two graphs, write down
1. its adjacency matrix
2. its degree matrix
3. its Laplacian matrix
4. the number of pivot columns in its Laplacian matrix (you can use a
   computer to solve for this)

** (5 points)
#+ATTR_HTML: :width 300px
[[file:graph1.jpg]]

This graph is made up of two disjoint triangles, one with vertices
$v_1, v_4, v_5$ and one with vertices $v_2, v_3, v_6$.

** (5 points)
#+ATTR_HTML: :width 300px
[[file:graph2.jpg]]

This graph is the same as the previous graph but with an additional
edge connecting $v_3$ and $v_4$.





Consider the following undirected unweighted graph.\footnote{Image generated with Graph Editor (\texttt{https://csacademy.com/app/graph\_editor/})}
For this problem, you may use NumPy or solve by hand.
If you use NumPy, you must include the code you used to compute each part.
\begin{center}
  \includegraphics[scale=0.4]{graph.png}
\end{center}

\begin{enumerate}
\item (4 points) Write down the adjacency matrix $A$ for this graph.
\item (4 points) Compute $A^2$ and $A^3$.
\item (5 points)The \textit{Hadamard product} of two $m \times n$ matrices (denoted by $A \circ B$) is entry-wise multiplication (like `\texttt{*}' in NumPy).
  So $(A \circ B)_{ij} = A_{ij} * B_{ij}$ for any indices $i$ and $j$.
  Compute the value of
  \begin{displaymath}
    \frac 1 6 \left(\mathbf 1^T (A^2 \circ A) \mathbf 1\right)
  \end{displaymath}
  (Recall that $\mathbf 1$ is the all-ones vector.
  In the case of this problem, it must be in $\mathbb R^6$.)
\item (5 points) The \textit{trace} of a $n \times n$ matrix (denoted by $\mathsf{tr}(A)$) is the sum of the entries along it's diagonal. So
  \begin{displaymath}
    \mathsf{tr}(A) = A_{11} + A_{22} + \dots + A_{nn}.
  \end{displaymath}
  Compute the value of $\frac 1 6 \mathsf{tr}(A^3)$.
\end{enumerate}
