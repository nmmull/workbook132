\section*{Basic Exercises}

\begin{align*}
  A = \begin{bmatrix}
    6 & 1 \\
    -1 & -3 \\
    8 & 1
  \end{bmatrix} \quad B = \begin{bmatrix}
    6 & -5 \\
    -6 & -2 \\
    -3 & 4
  \end{bmatrix} \quad C = \begin{bmatrix}
    -4 & 0 \\
    -8 & 5 \\
    -4 & -1
  \end{bmatrix}
\end{align*}
Compute the following expressions.
\begin{enumerate}
\item $A + B$
\item $2A - 4B - 3C$
\item $A^TA + B^TB$
\item $AC^T + CA^T$
\end{enumerate}
Compute the following matrix multiplications, if possible.
If it is not possible, explain why.
\begin{enumerate}[resume]
\item
  \begin{align*}
    \begin{bmatrix}
      -2 & 6 \\
      -2 & -7 \\
      -3 & -2
    \end{bmatrix}
    \begin{bmatrix}
      -4 & 8 & 4 \\
      -10 & -4 & 9
    \end{bmatrix}
  \end{align*}
\item
  \begin{align*}
    \begin{bmatrix} 8 & 6 & -1 & -2 \end{bmatrix}
    \begin{bmatrix} 3 \\ 5 \\ 1 \\ 5 \end{bmatrix}
  \end{align*}
\item
  \begin{align*}
    \begin{bmatrix} -3 \\ 6 \\ 8 \\ 5 \end{bmatrix}
    \begin{bmatrix} -9 & -7 & 0 & -8 \end{bmatrix}
  \end{align*}
\item
  \begin{align*}
    \begin{bmatrix}
      7 & -10 \\
      -6 & 6 \\
      1 & 0 \\
      -4 & 2
    \end{bmatrix}
    \begin{bmatrix}
      -8 & 9 \\
      -10 & 0 \\
      -9 & 1 \\
      -4 & -3
    \end{bmatrix}
  \end{align*}
\item
  \begin{align*}
    \begin{bmatrix}
      2 & 3 & -1 \\
      7 & 4 & 3 \\
      0 & -1 & 2
    \end{bmatrix}
    \begin{bmatrix}
      1 & 4 & 1 \\
      0 & 2 & 1 \\
      0 & 0 & 3
    \end{bmatrix}
  \end{align*}
\item
  \begin{align*}
    \begin{bmatrix}
      1 & 0 & 1 & 0 & 1 \\
      0 & 1 & 0 & 1 & 0 \\
      1 & 0 & 1 & 0 & 1
    \end{bmatrix}
    \begin{bmatrix}
      1 & 2 & 3 \\
      2 & 3 & 4 \\
      3 & 4 & 5 \\
      4 & 5 & 6 \\
      5 & 6 & 7
    \end{bmatrix}
  \end{align*}
\item
  \begin{align*}
    \begin{bmatrix}
      1 & 2 & 3 & 4 & 5 & 6
    \end{bmatrix}
    \begin{bmatrix}
      1 \\ 2 \\ 3 \\ 4 \\ 5 \\ 6
    \end{bmatrix}
  \end{align*}
\item
  \begin{align*}
    \begin{bmatrix}
      1 \\ 2 \\ 3 \\ 4 \\ 5 \\ 6
    \end{bmatrix}
    \begin{bmatrix}
      1 & 2 & 3 & 4 & 5 & 6
    \end{bmatrix}
  \end{align*}
\item
  \begin{displaymath}
    \begin{bmatrix}
      1 & -1 & 4 \\
      3 & 0 & 1 \\
      0 & 1 & 1 \\
      -3 & 1 & 2
    \end{bmatrix}
    \begin{bmatrix}
      2 & 0 & 1 & 1 \\
      0 & 0 & 2 & 3 \\
      1 & 1 & 3 & 5
    \end{bmatrix}
  \end{displaymath}
\item
  \begin{displaymath}
    \begin{bmatrix}
      1 \\ 2 \\ 3 \\ -1
    \end{bmatrix}
    \begin{bmatrix}
      2 & -1 & 2 & 2
    \end{bmatrix}
  \end{displaymath}
\item
  \begin{displaymath}
    \begin{bmatrix}
      1 \\ 1 \\ 1 \\ 1 \\ 1 \\ 1
    \end{bmatrix}^T
    \begin{bmatrix}
      1 & 0 & 1 & 0 & 1 & 0 \\
      0 & 1 & 0 & 1 & 0 & 1 \\
      1 & 0 & 1 & 0 & 1 & 0 \\
      0 & 1 & 0 & 1 & 0 & 1 \\
      1 & 0 & 1 & 0 & 1 & 0 \\
      0 & 1 & 0 & 1 & 0 & 1 \\
    \end{bmatrix}
    \begin{bmatrix}
      1 \\ 1 \\ 1 \\ 1 \\ 1 \\ 1
    \end{bmatrix}
  \end{displaymath}
\item
  \begin{displaymath}
    \begin{bmatrix}
      \mathbf a_1 & \mathbf a_2 & \mathbf a_3 & \mathbf a_4 & \mathbf a_5
    \end{bmatrix}
    \begin{bmatrix}
      0 & 0 & 1 & 0 & 0 \\
      0 & 1 & 0 & 0 & 0 \\
      0 & 0 & 0 & 1 & 0 \\
      1 & 0 & 0 & 0 & 0 \\
      0 & 0 & 0 & 0 & 1
    \end{bmatrix}
  \end{displaymath}
  where $\mathbf a_1$, $\mathbf a_2$, $\mathbf a_3$, $\mathbf a_4$, $\mathbf a_5$ are vectors in $\mathbb R^n$.
\item
  \begin{align*}
    \begin{bmatrix}
      1 & 1 \\
      0 & 1
    \end{bmatrix}^{2023}
  \end{align*}
\item
  \begin{align*}
    \begin{bmatrix}
      \cos2 & -\sin2 & 0 \\
      \sin2 & \cos2 & 0 \\
      0 & 0 & 1
    \end{bmatrix}^{2023}
  \end{align*}
\end{enumerate}
Let $A$ be matrix such that $A^{-1}$ is defined as below.
Use this to determine the solution to the matrix equations of the form $A\mathbf x = \mathbf b_i$, where each $\mathbf b_i$ is defined below.
\begin{enumerate}[resume]
\item
  \begin{align*}
    A^{-1} =
    \begin{bmatrix}
      -10 & -10 & -1 \\
      -2 & -4 & -3 \\
      -8 & -6 & 1
    \end{bmatrix}
    \quad
    \mathbf b_{1} = \begin{bmatrix} -5 \\ 6 \\ -1 \end{bmatrix}
    \quad
    \mathbf b_{2} = \begin{bmatrix} -1 \\ -3 \\ -10 \end{bmatrix}
    \quad
    \mathbf b_{3} = \begin{bmatrix} -6 \\ 4 \\ -4 \end{bmatrix}
  \end{align*}
\end{enumerate}
Determine the inverse of the following matrices.
If the matrix is not invertible, then explain why.
\begin{enumerate}[resume]
\item
  \begin{align*}
    \begin{bmatrix}
      2 & -1 \\
      3 & 3
    \end{bmatrix}
  \end{align*}
\item
  \begin{align*}
    \begin{bmatrix}
      -8 & -2 \\
      -2 & -3
    \end{bmatrix}
  \end{align*}
\item
  \begin{align*}
    \begin{bmatrix}
      1 & 3 \\
      -3 & -2
    \end{bmatrix}
  \end{align*}
\item
  \begin{align*}
    \begin{bmatrix}
      0 & 0 & 1 \\
      0 & 2 & 0 \\
      3 & 0 & 0
    \end{bmatrix}
  \end{align*}
\item
  \begin{align*}
    \begin{bmatrix}
      1 & 0 & -2 \\
      -3 & 1 & 4 \\
      2 & -3 & 4
    \end{bmatrix}
  \end{align*}
\item
  \begin{align*}
    \begin{bmatrix}
      1 & -1 & 1 \\
      -3 & 4 & -4 \\
      0 & 0 & 1
    \end{bmatrix}
  \end{align*}
\item
  \begin{align*}
    \begin{bmatrix}
      1 & 1 & 0 & -1 \\
      -2 & -1 & -1 & 0 \\
      1 & 1 & 0 & 0 \\
      -2 & -2 & 0 & 1
    \end{bmatrix}
  \end{align*}
\item
  \begin{align*}
    \begin{bmatrix}
      1 & 0 & 0 \\
      2 & 1 & 0 \\
      3 & 4 & 1 \\
    \end{bmatrix}
  \end{align*}
\item
  \begin{align*}
    \begin{bmatrix}
      \cos \theta & -\sin \theta \\
      \cos \theta + \sin \theta & \cos \theta - \sin \theta
    \end{bmatrix}
  \end{align*}
  (Simplify your solution as much as possible.)
\end{enumerate}
Determine the matrix in $\mathbb R^{4 \times 4}$ that implements the following row operations, in order from top to bottom.
That is, determine a matrix $A$ such that $AB$ is the result of applying the following row operations, from top to bottom, to $B$.
\begin{enumerate}[resume]
\item
  \begin{align*}
    R_{2} &\gets R_{2} - R_{3} \\
    R_{1} &\leftrightarrow R_{4} \\
    R_{1} &\gets R_{1} + 3R_{2} \\
    R_{3} &\gets R_{3} - 5R_{4} \\
    R_{4} &\gets -2R_{4}
  \end{align*}
\end{enumerate}
Determine the inverse of the following transformation, if it exists.
Your solution should be in the form of a transformation, as given below.
\begin{enumerate}[resume]
\item
  \begin{align*}
    \begin{bmatrix}
      x_{1} \\
      x_{2} \\
      x_{3}
    \end{bmatrix} \mapsto \begin{bmatrix}
      x_{1} + x_{3} \\
      -3x_{1} + x_{2} - 4x_{3} \\
      x_{1} + 2x_{2}
    \end{bmatrix}
  \end{align*}
\end{enumerate}
Suppose that $A$ is a matrix in $\mathbb R^{3 \times 3}$ such that the following sequence of row operations (from top to bottom) transforms $A$ into the identity matrix.
Determine the inverse of $A$.
You should do this without determining $A$ first.
\begin{enumerate}[resume]
\item
  \begin{align*}
    R_{1} &\leftrightarrow R_{2} \\
    R_{1} &\gets R_{1} - R_{2} \\
    R_{3} &\gets R_{3} - 3R_{2} \\
    R_{1} &\gets R_{1} + 3R_{2} \\
    R_{3} &\gets R_{3} - 5R_{1} \\
    R_{1} &\leftrightarrow R_{2}
  \end{align*}
\end{enumerate}

\section*{True/False}

\trueFalseHeader

\begin{enumerate}[resume]
\item
  For all matrices $A$ and $B$ in $\mathbb{R}^{m \times n}$ and $C$ in $\mathbb R^{n \times m}$, we have $(A + B + C^T)^T = C + B^T + A^T$.
\item
  For all matrices $A$ and $B$ such that $AB$ is defined, we have $AB \not = BA$.
\item
  If $A\mathbf x = \mathbf b$ has a unique solution for every vector $\mathbf b$ in the span of the columns of $A$, then $A$ is invertible.
\item
  For any matrix $A$ and vector $\mathbf v$, if $\mathbf v^T A$ is defined, then $A$ is single column.
\item
  For any matrices $A$ and $B$, if $A^{-1} = B^{-1}$, then $A = B$.
\item
  For any matrices $A$ and $B$, if there is a unique matrix $X$ such that $AX = B$, then $A$ is invertible.
\item
  If $A$ and $B$ are invertible, then so is $A + B$.
\item
  If $A$ and $B$ are invertible, then so is $BA$.
\item
  If $A$ and $B$ in $\mathbb R^{n \times n}$ are invertible, then so is $\begin{bmatrix}A & B \\ B & A\end{bmatrix}$, the matrix in $\mathbb R^{2n \times 2n}$ gotten by stacking copies of $A$ and $B$.
\item
  For any matrices $A$ and $B$, if there is a unique matrix $X$ such that $AX = B$, then $A$ is invertible.
\item
  If $A$ and $B$ are symmetric and $AB = BA$ then $AB$ is symmetric.
\item
  For any matrix $A \in \mathbb R^{n \times n}$, if the columns of
  $A^3$ span all of $\mathbb R^n$, then the columns of $A$ are
  linearly independent.
\item
  For any matrix $A \in \mathbb R^{n \times n}$, the matrix $A + A^T$ is symmetric.
\item
  If $AB$ and $BA$ are symmetric, then $A$ and $B$ are symmetric.
\item
  If $A \in \mathbb R^{n \times n}$ has zeros along its diagonal, then
  $A$ is not invertible.
\item
  If $A \in \mathbb R^{n \times n}$ has a row of all zeros, then $A$
  is not invertible.
\item
  If $A \in \mathbb R^{2 \times 2}$ and $A^{-1}$ has integer entries
  then the determinant of $A$ is $1$.
\item
  For any square matrices $A$ and $B$, if $AB = I$, then $AB = BA$.
\item
  The \textbf{Hadamard product} of two matrices is defined as
  \begin{align*}
    (A \circ B)_{ij} = A_{ij} B_{ij}
  \end{align*}
  In other words, $A$ and $B$ are multiplied entry-wise.  For any
  invertible matrices $A$ and $B$, if $A \circ B$ is invertible, then
  $(A \circ B)^{-1} = A^{-1} \circ B^{-1}$.
\end{enumerate}

\section*{More Difficult Problems}

\begin{enumerate}[resume]
\item
  Let $A$ be as defined below.  Is it possible to write the
  inverse of $A$ as a power of $A$?  If so, determine the smallest positive integer $n$ such that $A^n = A^{-1}$.
  \begin{align*}
    A = \begin{bmatrix} 1 & 1 \\ -1 & 0 \end{bmatrix}
  \end{align*}
\item Determine the smallest positive integer $n$ such that $A^n = A^{-1}$.
  \begin{align*}
    A =
    \begin{bmatrix}
      \cos\frac{\pi}{9} & -\sin\frac{\pi}{9} & 0 \\
      \sin\frac{\pi}{9} & \cos\frac{\pi}{9} & 0 \\
      0 & 0 & 1
    \end{bmatrix}
  \end{align*}
\item
  Compute the following matrix expression.
  Your answer should be a single matrix with entries given in terms of $n$.
  \begin{align*}
    \begin{bmatrix}
      1 & 0 & 2 \\
      0 & 1 & 0 \\
      0 & 0 & 1
    \end{bmatrix}^{n}
    \begin{bmatrix}
      1 & 0 & 0 \\
      0 & 1 & 0 \\
      0 & -3 & 1
    \end{bmatrix}^{-n}
  \end{align*}
\item
  Suppose that $A$ and $B$ are invertible matrices such that $AB^TXA^{-1}B = I$ for some matrix $X$.
  Determine $X$ in terms of $A$ and $B$.
\item
  Let $A$, $B$, and $C$ such that $A = A^{-1}$ and $C = C^T$ and
  \begin{align*}
    A(C^{-1}(AB)^T)^TC
  \end{align*}
  is well-defined.
  Simplify this expression using the algebraic properties of matrix operations.
\item
  Determine a matrix in $\mathbb R^{2 \times 2}$ with all nonzero entries that is equal to its inverse.
  \textit{Hint.} Use the closed-form equation for the inverse of a $2 \times 2$ matrix.

\end{enumerate}

\section*{Challenge Problems}

\begin{enumerate}[resume]
\item
  Determine \textit{three} matrices in $\mathbb R^{2 \times 2}$ that satisfy the following equation.
  In particular, you must demonstrate that each matrix in your solution satisfies the equation.
  \begin{align*}
    X^2 +
    \begin{bmatrix}
      3 & 0 \\
      0 & 1
    \end{bmatrix}
    X
    +
    \begin{bmatrix}
      2 & 0 \\
      0 & 0
    \end{bmatrix}
    =
    \begin{bmatrix}
      0 & 0 \\
      0 & 0
    \end{bmatrix}
  \end{align*}
\item
  Determine the reduced echelon form of the matrix
  \begin{align*}
    \begin{bmatrix}
      a & b & 1 & 0 \\
      c & d & 0 & 1
    \end{bmatrix}
  \end{align*}
  in terms of $a$, $b$, $c$, $d$. Show your work.
\item
  Determine two invertible matrices $A$ and $B$ such that $AB^{-1} = -BA^{-1}$.
\item
  Let $A$ and $B$ be two invertible matrices in $\mathbb R^{n \times n}$ such that $AB^{-1} = -BA^{-1}$.
  Determine the inverse of the following matrix in $\mathbb R^{2n \times 2n}$.
  \begin{align*}
    \begin{bmatrix}
      A & B \\
      B & A
    \end{bmatrix}
  \end{align*}
\end{enumerate}
