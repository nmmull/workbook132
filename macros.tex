
\DeclareMathOperator{\col}{\mathrm{Col}}
\DeclareMathOperator{\row}{\mathrm{Row}}
\DeclareMathOperator{\nul}{\mathrm{Nul}}
\DeclareMathOperator{\rank}{\mathrm{rank}}
\DeclareMathOperator{\proj}{\mathrm{proj}}
\DeclareMathOperator{\vspan}{\mathrm{span}}
\DeclareMathOperator{\argmin}{\mathrm{argmin}}
\DeclareMathOperator{\argmax}{\mathrm{argmax}}

\newcommand{\R}{\mathbb R}
\newcommand{\cB}{\mathcal B}

\newcommand{\va}{\mathbf a}
\newcommand{\vb}{\mathbf b}
\newcommand{\vc}{\mathbf c}
\newcommand{\ve}{\mathbf e}
\newcommand{\vo}{\mathbf o}
\newcommand{\vu}{\mathbf u}
\newcommand{\vv}{\mathbf v}
\newcommand{\vw}{\mathbf w}
\newcommand{\vx}{\mathbf x}
\newcommand{\vy}{\mathbf y}
\newcommand{\vz}{\mathbf z}
\newcommand{\vzero}{\mathbf 0}

\newcommand{\vTwo}[2]{
  \ensuremath{
    \begin{bmatrix}
      #1 \\ #2
    \end{bmatrix}
  }
}
\newcommand{\vThree}[3]{
  \ensuremath{
    \begin{bmatrix}
      #1 \\ #2 \\ #3
    \end{bmatrix}
  }
}
\newcommand{\vFour}[4]{
  \ensuremath{
    \begin{bmatrix}
      #1 \\ #2 \\ #3 \\ #4
    \end{bmatrix}
  }
}
\newcommand{\vFive}[5]{
  \ensuremath{
    \begin{bmatrix}
      #1 \\ #2 \\ #3 \\ #4 \\ #5
    \end{bmatrix}
  }
}
\newcommand{\vSix}[6]{
  \ensuremath{
    \begin{bmatrix}
      #1 \\ #2 \\ #3 \\ #4 \\ #5 \\ #6
    \end{bmatrix}
  }
}

\newcommand{\trueFalseHeader}{
  Determine if each statement is \textbf{true} or \textbf{false} and
  justify your answer.  In particular, if the statement is false,
  provide a counterexample if possible.
}
