\section*{Basic Exercises}

For each of the following linear transformations, determine its domain,
codomain, and the matrix that implements it (i.e., determine the
matrix $A$ such that the given transformation is $\vx \mapsto A \vx$).
\begin{enumerate}
\item
  \begin{displaymath}
    \begin{bmatrix}
      x_1 \\ x_2 \\ x_3
    \end{bmatrix}
    \mapsto
    \begin{bmatrix}
      x_1 + x_2 \\
      -2x_1 - x_3 \\
      x_1 + x_3
    \end{bmatrix}
  \end{displaymath}
\item
  \begin{displaymath}
    \begin{bmatrix}
      x_1 \\ x_2 \\ x_3 \\ x_4
    \end{bmatrix}
    \mapsto
    \begin{bmatrix}
      x_2 \\ x_1 \\ x_3 + x_4
    \end{bmatrix}
  \end{displaymath}
\item
  \begin{displaymath}
    \begin{bmatrix} x_1 \\ x_2 \\ x_3 \end{bmatrix}
    \mapsto
    \begin{bmatrix}
      5x_1 + 7x_2 + 10x_3 \\
      2x_1 - x_2 + 4x_3 \\
      -3x_2
    \end{bmatrix}
  \end{displaymath}
\item
  \begin{align*}
    \begin{bmatrix} x_{1} \\ x_{2} \\ x_{3} \\ x_{4} \\ x_{5} \end{bmatrix}
    \mapsto
    \begin{bmatrix}
      -9x_{1} + 4x_{2} + 5x_{3} - 3x_{4} + 2x_{5} \\
      3x_{1} + 8x_{2} + 5x_{3} - 5x_{4} + 7x_{5} \\
      -x_{1} + 5x_{2} - x_{3} + 4x_{4} + 8x_{5}
    \end{bmatrix}
  \end{align*}
\item
  \begin{align*}
    \begin{bmatrix}
      x_1 \\ x_2 \\ x_3 \\ x_4
    \end{bmatrix}
    \mapsto
    \begin{bmatrix}
      -2x_2 + x_3 + x_4 \\ x_1 + x_3 \\ -3x_3 - 3x_4 \\ 9x_4
    \end{bmatrix}
  \end{align*}
\item
  \begin{align*}
    \left[\begin{matrix}x_{1}\\x_{2}\end{matrix}\right]
    \mapsto
    \begin{bmatrix}- 10 x_{1} - 4 x_{2} \\8 x_{1} \\10 x_{1} - 9 x_{2} \\- x_{1} + 7 x_{2}\end{bmatrix}
  \end{align*}
\item
  \begin{align*}
    \left[\begin{matrix}x_{1}\\x_{2}\\x_{3}\\x_{4}\end{matrix}\right]
    \mapsto
    \begin{bmatrix}7 x_{1} - 5 x_{2} - 7 x_{3} - 8 x_{4} \\6 x_{1} + x_{2} + 2 x_{3} - 8 x_{4}\end{bmatrix}
  \end{align*}
\end{enumerate}
Let $T$ be a linear transformation with the following input-output
behavior.
\begin{align*}
  T(\vv_{1}) = \begin{bmatrix} -6 \\ 3 \\ -2 \\ -10 \end{bmatrix}
  \qquad
  T(\vv_{2}) = \begin{bmatrix} -5 \\ 1 \\ -2 \\ 9 \end{bmatrix}
  \qquad
  T(\vv_{3}) = \begin{bmatrix} 8 \\ -7 \\ 6 \\ 6 \end{bmatrix}
\end{align*}
Compute the following.
\begin{enumerate}[resume]
\item
  $T(5\vv_3)$
\item
  $T(\vv_1 + \vv_2)$
\item
  $T(-3\vv_{1} - \vv_{2} - 2\vv_{3})$
\item
  $T(2\vv_1 + 2\vv_2 - 4\vv_3)$
\end{enumerate}
Determine if the vector $\mathbf v$ is in the range of the matrix
transformation $T$ given by $\mathbf x \mapsto A \mathbf x$, where $A$
and $\mathbf v$ are defined below.  If $\mathbf v$ is in the range of
$T$, then determine a vector whose image under $T$ is $\mathbf v$.
Furthermore, state whether the vector you determined is unique, i.e.,
determine if there is another vector whose image under $T$ is $\mathbf
v$.
\begin{enumerate}[resume]
\item
\begin{align*}A = \left[\begin{matrix}3 & -3\\2 & 1\end{matrix}\right] \quad \mathbf v = \left[\begin{matrix}21\\8\end{matrix}\right]\end{align*}
\item
  \begin{align*}
    A = \begin{bmatrix}
      1 & -1 & 2 \\
      2 & -1 & 3 \\
      -2 & 2 & -3
    \end{bmatrix} \quad \mathbf v =
    \begin{bmatrix}
      6 \\
      12 \\
      -10
    \end{bmatrix}
  \end{align*}
\item
\begin{align*}A = \left[\begin{matrix}2 & 6 & 6 & -30\\2 & 6 & 3 & -21\\-1 & -3 & -1 & 9\end{matrix}\right] \quad \mathbf v = \left[\begin{matrix}0\\9\\-2\end{matrix}\right]\end{align*}
\end{enumerate}


Let $T$ be a linear transformation with the following input-output
behavior.
\begin{align*}
  T
  \left(
  \begin{bmatrix}
    -3 \\ 9 \\ 5 \\ 4
  \end{bmatrix}
  \right)
  =
  \begin{bmatrix}
    -4 \\ 3 \\ 4
  \end{bmatrix}
  \qquad
  T
  \left(
  \begin{bmatrix}
    -9 \\ -5 \\ 0 \\ -7
  \end{bmatrix}
  \right)
  =
  \begin{bmatrix}
    0 \\ 1 \\ 2
  \end{bmatrix}
  \qquad
  T
  \left(
  \begin{bmatrix}
    9 \\ -1 \\ -2 \\ 4
  \end{bmatrix}
  \right)
  =
  \begin{bmatrix}
    3 \\ -3 \\ 7
  \end{bmatrix}
\end{align*}
Compute the following.
\begin{enumerate}[resume]
\item \faCalculator \
  \begin{align*}
    T
    \left(
    \begin{bmatrix}
      30 \\ 0 \\ -7 \\ 22
    \end{bmatrix}
    \right)
  \end{align*}
\item \faCalculator \
  \begin{align*}
    T
    \left(
    \begin{bmatrix}
      1 \\ 0 \\ 0 \\ 0
    \end{bmatrix}
    \right)
  \end{align*}
\end{enumerate}
Determine the matrix that implements the linear transformation $T$,
given its input-output behavior.
\begin{enumerate}[resume]
\item
  \begin{align*}
    T\left(\begin{bmatrix} 1 \\ -2 \end{bmatrix}\right)
    = \begin{bmatrix} -3 \\ -1 \end{bmatrix}
    \qquad
    T\left(\begin{bmatrix} 2 \\ -3 \end{bmatrix}\right)
    = \begin{bmatrix} -4 \\ -1 \end{bmatrix}
  \end{align*}
\item
  \begin{align*}
    T\left(\begin{bmatrix} 1 \\ -2 \\ -3 \end{bmatrix}\right)
    = \begin{bmatrix} 1 \\ 1 \\ 1 \end{bmatrix}
    \qquad
    T\left(\begin{bmatrix} 2 \\ -3 \\ -4 \end{bmatrix}\right)
    = \begin{bmatrix} 1 \\ 2 \\ 3 \end{bmatrix}
    \qquad
    T\left(\begin{bmatrix} 2 \\ -4 \\ -5 \end{bmatrix}\right)
    = \begin{bmatrix} 2 \\ 1 \\ 3 \end{bmatrix}
  \end{align*}
\item
  \begin{align*}
    T\left(\begin{bmatrix} 2 \\ 1 \\ 0 \end{bmatrix} \right)
    = \begin{bmatrix} 3 \\ -3 \\ 4 \end{bmatrix}
    \qquad
    T\left(\begin{bmatrix} 1 \\ 1 \\ 0 \end{bmatrix} \right)
    = \begin{bmatrix} 8 \\ 0 \\ 2 \end{bmatrix}
    \qquad
    T\left(\begin{bmatrix} 2 \\ 1 \\ 2 \end{bmatrix} \right)
    = \begin{bmatrix} -1 \\ -1 \\ -1 \end{bmatrix}
  \end{align*}
\item
  \begin{align*}
    T \left(\begin{bmatrix} -7 \\ -3 \end{bmatrix} \right)
    = \begin{bmatrix} -4 \\ 3 \\ 4 \\ 0 \\ 1 \end{bmatrix}
    \qquad
    T \left(\begin{bmatrix} 9 \\ 4 \end{bmatrix} \right)
    = \begin{bmatrix} 13 \\ 0 \\ 0 \\ 8 \\ 3 \end{bmatrix}
  \end{align*}
\end{enumerate}
For each matrix, determine if its transformation is (a) one-to-one but
not onto, (b) onto but not one-to-one, (c) both, or (d)
neither. Justify your answer.
\begin{enumerate}[resume]
\item
  \begin{align*}
    \begin{bmatrix}
      1 & 2 & 1 & 3 \\
      1 & 3 & 1 & 3 \\
      1 & 5 & 1 & 4
    \end{bmatrix}
  \end{align*}
\item
  \begin{align*}
    \begin{bmatrix}
      1 & 1 & 2 \\
      2 & 3 & 5 \\
      -1 & -3 & -3
    \end{bmatrix}
  \end{align*}
\item
  \begin{align*}
    \begin{bmatrix}
      5 & 0 & 5 \\
      0 & 0 & 0 \\
      4 & 0 & 7
    \end{bmatrix}
  \end{align*}
\item
  \begin{align*}
    \begin{bmatrix}
      1 & 2 \\
      0 & 1 \\
      -2 & -2 \\
      1 & 3 \\
      1 & 4
    \end{bmatrix}
  \end{align*}
\item
  \begin{align*}
    \begin{bmatrix}
      1 \\ 1 \\ 1 \\ 1 \\ 1 \\ 1
    \end{bmatrix}
  \end{align*}
\item
  \begin{align*}
    \begin{bmatrix}
      2 & 0 & 0 \\
      0 & 1 & 3 \\
      0 & 0 & 3 \\
      0 & 0 & 3
    \end{bmatrix}
  \end{align*}
\item
  \begin{align*}
    \begin{bmatrix}
      1 & 2 & -1 & 6 & 9 \\
      2 & 5 & -5 & 13 & 22 \\
      -3 & -4 & -3 & -16 & -19
    \end{bmatrix}
  \end{align*}
\item
  \begin{align*}
    \begin{bmatrix}
      2 & 1 & 3 & 0 \\
      1 & 4 & 3 & 0 \\
      0 & 0 & 1 & 0
    \end{bmatrix}
  \end{align*}
\item
  \begin{align*}
    \begin{bmatrix}
      1 & -2 & 3 & -4 \\
      0 & 5 & -6 &  7 \\
      0 & 0 & -8 & 9 \\
      0 & 0 & 0 & -10
    \end{bmatrix}
  \end{align*}
\item
  \begin{align*}
    \begin{bmatrix}
      12 & 45 & -3 & 20 & 1 \\
      0 & 24 & 121 & 0 & -47 \\
      0 & 0 & 252 & 44 & 46 \\
      0 & 0 & 0 & 21 & -44 \\
      0 & 0 & 0 & 0 & 11
    \end{bmatrix}
  \end{align*}
\item
  \begin{align*}
    \begin{bmatrix}
      1 & 2 & 3 & 4 \\
      -13 & 6 & -39 & 4 \\
      2 & 9 & 6 & 7 \\
      5 & 4  & 15 & 16 \\
      0 & 1 & 0 & 65 \\
      1 & 1 & 1 & 1 \\
      -12 & 3 & -36 & 33 \\
      0 & 0 & 0 & 0
    \end{bmatrix}
  \end{align*}
\item
\end{enumerate}
Draw the image of the unit square under the following linear
transformations.
\begin{enumerate}[resume]
\item
  \begin{align*}
    \begin{bmatrix} x_1 \\ x_2 \end{bmatrix}
    \mapsto
    \begin{bmatrix}
      -2 & -2 \\
      -1 & 0
    \end{bmatrix}
    \begin{bmatrix} x_1 \\ x_2 \end{bmatrix}
  \end{align*}
\item
\begin{displaymath}
  \vTwo {x_1} {x_2}
  \mapsto
  \begin{bmatrix}
    1 & 2 \\
    3 & 4
  \end{bmatrix}
  \vTwo {x_1} {x_2}
\end{displaymath}
\item
\begin{displaymath}
  \vTwo {x_1} {x_2}
  \mapsto
  \begin{bmatrix}
    1 & 2 \\
    -2 & 1
  \end{bmatrix}
  \vTwo {x_1} {x_2}
\end{displaymath}
\item The transformation $T$ which rotates vectors by 90 degrees about
  the origin and then reflects across the line $y = x$.
\end{enumerate}

\section*{True/False}

\trueFalseHeader

\begin{enumerate}[resume]
\item
  If $\vx \mapsto A \vx$ is one-to-one and onto, then $A$ is a square
  matrix.
\item
  If $\vx \mapsto A \vx$ is neither one-to-one or onto, and the
  reduced echelon form of $A$ only has $0$s and $1$s, then one of the
  columns of $A$ is $\vzero$.
\item
  For any matrix $A \in \R^{m \times n}$ where $m > n$, it is not
  possible for the transformation $\vx \mapsto A \vx$ to be
  one-to-one.
\item
  For any matrix $A$, if the matrix transformation $\vx \mapsto A \vx$
  is onto, then it is also one-to-one.
\item
  If $A$ is an $m \times n$ matrix and $m < n$, then the range of
  $\vx \mapsto A\vx$ is $\R^m$.
\end{enumerate}

\begin{enumerate}[resume]
\item
  If $T : \R^m \to \R^n$ is a linear transformation and $T(\vv) = A
  \vv$ for all vectors $\vv$ in the domain of $T$, then $A$ is an $m
  \times n$ matrix.
\item
  The matrix that implements a linear transformation is unique.
\end{enumerate}

\section*{More Difficult Problems}

\begin{enumerate}[resume]
\item
  Let $A$ be a matrix with $n$ rows and 6 columns.  Each row of $A$
  contains the \textbf{unweighted} percentage scores (out of 100) of
  one student on 4 homework assignments (columns 1 through 4) a
  midterm exam (column 5) and a final exam (column 6).
  \begin{displaymath}
    \begin{matrix}
      H_1 & H_2 & H_3 & H_4 & M & F
    \end{matrix}
  \end{displaymath}
  \begin{displaymath}
    \begin{bmatrix}
      p_{11} & p_{12} & p_{13} & p_{14} & p_{15} & p_{16} \\ p_{21} &
      p_{22} & p_{23} & p_{24} & p_{25} & p_{26} \\ \vdots & \vdots &
      \vdots & \vdots & \vdots & \vdots \\ p_{n1} & p_{n2} & p_{n3} &
      p_{n4} & p_{n5} & p_{n6} \\
    \end{bmatrix}
  \end{displaymath}
  All homework assignments are worth the same amount.  Let $T$ denote
  the linear transformation implemented by this matrix.
  \begin{enumerate}
  \item
    Suppose that homework assignments account for \textbf{50} percent
    of the final grade, the midterm exam accounts for \textbf{20}
    percent and the final exam accounts for \textbf{30} percent.  Find
    a vector $\vv$ such that $T(\vv)$ is the vector whose
    $i^\text{th}$ entry is the final percentage grade of the
    $i^\text{th}$ student.  For example, if the $i^\text{th}$ student
    recieved $90$ percent on every homework assignment, $85$ percent
    on the midterm, and $92$ percent on the final, then the
    $i^\text{th}$ entry of the output vector should be $90 * 0.5 + 85
    * 0.2 + 92 * 0.3$.
  \item
    Find a vector $\vv$ such that $T(\vv)$ is the vector
    whose $i^\text{th}$ entry is the unweighted homework grade for
    student $i$. For the same example as above, the $i^\text{th}$
    entry would be $90$.
  \end{enumerate}
\item
  Consider the following linear transformation $T$.
  \begin{displaymath}
    \begin{bmatrix}
      x_1 \\ x_2 \\ x_3
    \end{bmatrix}
    \mapsto
    \begin{bmatrix}
      x_1 + 2x_2 - x_3 \\
      x_3 \\
      2x_3
    \end{bmatrix}
  \end{displaymath}
  Determine a set of linearly independent vectors which span the range
  of $T$.
\item
  Determine the matrix which implements the linear tranformation $T :
  \R^3 \to \R^3$ that reflects vectors across the $x_1x_2$-plane.
\item
  Determine the matrix which implements the linear transformation $T :
  \R^2 \to \R^4$ that repeats the input vector, e.g.,
  \begin{align*}
    T\left(\begin{bmatrix} 1 \\ 2 \end{bmatrix} \right)
    = \begin{bmatrix} 1 \\ 2 \\ 1 \\ 2 \end{bmatrix}
  \end{align*}
\item
  Determine the matrix which implements the linear transformation $T :
  \R^5 \to \R^5$ which adds to each entries all its preceding entries,
  e.g.,
  \begin{displaymath}
    T\left(\vFive 1 2 3 4 5 \right) =
    \vFive 1 3 6 {10} {15}
  \end{displaymath}
\item
  Determine the matrix which implements the linear transformation $T :
  \R^2 \to \R^2$ that reflects vectors across the line $y = x$.
\item
  Determine the matrix which implements the linear transformation $T :
  \R^2 \to \R^2$ that reflects vectors across the line $y =
  \tan(\frac{3\pi}{8})x$.
\item
  Determine the matrix which implements the linear transformation $T : \R^4
  \to \R^4$ that swaps the first and second entries of its input,
  e.g.,
  \begin{align*}
    T \left( \begin{bmatrix} 1 \\ 2 \\ 3 \\ 4 \end{bmatrix} \right)
    = \begin{bmatrix} 2 \\ 1 \\ 3 \\ 4 \end{bmatrix}
  \end{align*}
\item
  Determine the matrix which implements the linear transformation $T :
  \R^2 \to \R^2$ that reflects vectors across the line $y =
  2x$. (\textit{Hint:} The vector $\vTwo 2 {-1}$ is perpendicular to
  the line $y = 2x$)
\item
  Consider the following transformation.
  \begin{align*}
    \begin{bmatrix}
      x \\ y \\ z
    \end{bmatrix}
    \mapsto
    \begin{bmatrix}
      (x^3 + y^3 + z^3)^{1 / 3} \\ y + z
    \end{bmatrix}
  \end{align*}
  \begin{enumerate}
  \item
    Demonstrate that the above transformation is homogeneous.
  \item
    Determine if the above transformation is not linear.
    Justify your answer.
  \end{enumerate}
\item
  Show that the transformation $T: \R^4 \to \R^4$ given by
  \begin{displaymath}
    \begin{bmatrix}
      v_1 \\ v_2 \\ v_3 \\ v_4
    \end{bmatrix}
    \mapsto
    \begin{bmatrix}
      \min(v_1, 100) \\
      \min(v_2, 100) \\
      \min(v_3, 100) \\
      \min(v_4, 100) \\
    \end{bmatrix}
  \end{displaymath}
  is not linear.
\item
  Suppose that $T : \R^3 \to \R^3$ is the linear transformation which
  reflects vectors across the $xy$ plane (i.e., across the plane given
  by the linear equation $z = 0$) and that $S : \R^3 \to \R^3$ the
  transformation which rotates vectors around $\vspan\{[1 \ \ 1
    \ \ 0]^T\}$ by $180$ degrees.  Determine the matrix which
  implements $S \circ T$, the composition of $S$ and $T$ (recall that
  $(S \circ T)(\vv) = S(T(\vv))$).
\item
  Considering the transformation $T$ implemented by the following
  matrix.
  \begin{displaymath}
    \begin{bmatrix}
      \cos2 & 0 & -\sin2 \\
      0 & 1 & 0 \\
      \sin2 & 0 & \cos2
    \end{bmatrix}
  \end{displaymath}
  Describe geometrically what $T$ does.  Then find a vector $\vv$
  whose span is not changed by this transformation (i.e.,
  $\vspan\{\vv\} = \vspan\{T(\vv)\}$).
\item
  Consider the following linear transformations.
  \begin{align*}
    S
    \left(
    \vTwo {x_1} {x_2}
    \right)
    =
    \begin{bmatrix}
      x_1 \\ 2 x_2 \\ x_1 - x_2
    \end{bmatrix}
    \qquad
    T
    \left(
    \begin{bmatrix}
      x_1 \\ x_2 \\ x_3
    \end{bmatrix}
    \right)
    =
    \begin{bmatrix}
      x_1 + x_2 + x_3 \\
      -x_1 - 2x_2 - 3x_3
    \end{bmatrix}
  \end{align*}
  \begin{enumerate}
  \item
    Determine the matrix that implements $S \circ T$, the composition
    of $S$ and $T$.  That is, determine the matrix $A$ such that the
    transformation $\mathbf x \mapsto S(T(\mathbf x))$ is equivalent
    to the transformation $\mathbf x \mapsto A \mathbf x$.
  \item
    Determine a general form solution that describes the preimages of
    the following vector $\mathbf b$ under the linear transformation
    $S \circ T$.  That is determine a general form solution where
    $S(T(\mathbf x)) = \mathbf b$ if and only if $\mathbf x$ is in the
    solution set described by your general form solution.
    \begin{align*}
      \mathbf b =
      \begin{bmatrix}
        1 \\ 8 \\ -3
      \end{bmatrix}
    \end{align*}
  \end{enumerate}
\end{enumerate}

\section*{Challenge Problems}

\begin{enumerate}[resume]
\item
  Determine the matrices which implement the linear transformations
  that rotates vectors in $\R^3$ 120 degrees about
  \begin{align*}
    \vspan\left\{\begin{bmatrix}1 \\ 1 \\ 1\end{bmatrix}\right\}
  \end{align*}
  You should determine \textit{two} matrices, one for clockwise
  rotation and the other for counterclockwise rotation.
\item
  Determine the matrices which implement the linear transformations
  that reflects vectors across the plane defined by the linear
  equation $x + y + z = 0$, and then rotates vectors 60 degrees about
  \begin{align*}
    \vspan\left\{\begin{bmatrix}1 \\ 1 \\ 1\end{bmatrix}\right\}
  \end{align*}
  You should determine \textit{two} matrices, one for clockwise
  rotation and the other for counterclockwise rotation.
\item
  Determine the matrix which implements the linear transformation $T :
  \R^3 \to \R^3$ that reflects vectors across the plane defined by the
  linear equation $x + y + z = 0$.
\item
  Consider the following $\R^3$ rotation matrices.
  \begin{displaymath}
    A =
    \begin{bmatrix}
      \cos 45^\circ & -\sin 45^\circ & 0 \\
      \sin 45^\circ & \cos 45^\circ & 0 \\
      0 & 0 & 1
    \end{bmatrix}
    \qquad
    B =
    \begin{bmatrix}
      1 & 0 & 0 \\
      0 & \cos 180^\circ & -\sin 180^\circ \\
      0 & \sin 180^\circ & \cos 180^\circ
    \end{bmatrix}
  \end{displaymath}
  The matrix $A$ rotates vectors around the $x_3$-axis by 45 degrees,
  and $B$ rotates vectors around the $x_1$-axis by 180 degrees.
  \begin{enumerate}
  \item
    Determine $A^{-1}$.
  \item
    Calculate $ABA^{-1}$.
  \item
    Describe what the transformation implemented by $ABA^{-1}$ does
    geometrically.
  \end{enumerate}
\end{enumerate}
